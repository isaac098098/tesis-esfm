%%% Signal processing
\newpage
\section{Procesamiento de señales}

Una \emph{señal} es una descripción de un fenómeno que evoluciona en el tiempo o espacio; el \emph{procesamiento de señales} se refiere a cualquier operación manual o mecánica que modifique, analice o manipule de otra forma la información contenida en una señal. Considérese, por ejemplo, la temperatura ambiente; se puede medir su evolución coon el tiempo de muchas formas y los datos resultantes representan una ``señal'' de temperatura. Algunas operaciones sobre esta señal se pueden hacer incluso a mano, por ejemplo calcular la temperatura promedio en un mes o graficar la señal en una hoja de papel.

\bigskip

El adjetivo ``digital'' proviene de \emph{digitus}, palabra latina para dedo. En este contexto, se refiere a un paradigma en el que el mundo físico se puede describir usando únicamente números enteros.
El \emph{procesamiento digital de señales} es por tanto una rama del procesamiento de señales en la cual todo, incluido el tiempo, es descrito en términos de números enteros. En el procesamiento de digital de señales, la representación abstracta subyacente siempre es el conjunto de números naturales, independientemente de la naturaleza de la señal \cite{Prand1}.

\begin{definition}
Más específicamente, una \emph{señal} es cualquier función $x : \mathbb{Z} \longrightarrow \mathbb{C}$.
\end{definition}

\begin{remark}
Es necesario establecer las dos convenciones siguientes a lo largo del capítulo, siguiendo la notación estándar en la literatura de procesamiento de señales:
\begin{itemize}
\item El valor de una señal $x$ en un índice entero $n$ se denotará como $x[n]$, con corchetes en vez de paréntesis.
\item A partir de ahora se le denotará $j$ a la unidad imaginaria, es decir, $j = \sqrt{-1}$.
\item Nuevamente, durante todo el capítulo se supondrá que $r$ es un entero positivo arbitrario pero fijo.
\end{itemize}
\end{remark}
% \subsection{Señales simétricas}

En este capítulo se estudiará una clase particular de señales periódicas, llamadas señales \emph{simétricas} o \emph{pares}, con las herramientas del capítulo anterior. Estas señales no son más que funciones aritméticas pares definidas en los enteros.

\subsection{Transformada Discreta de Fourier}

\begin{definition}[Señal periódica]
Desde luego, una señal periódica con periodo $r \in \mathbb{N}$ es una señal para la cual $x[n] = x[n + k r]$ para cada $n,k \in \mathbb{Z}$.
\end{definition}

Una señal periódica con periodo $r$ contiene toda su información en un periodo, en el cuál toma $r$ valores complejos. Considérese pues un arreglo con estos valores, es decir, un elemento de $\mathbb{C}^r$.
\bigskip

Sea $W_r = e^{-j(2 \pi / r)}$ y considérese el producto interno $\langle x,y \rangle = \sum_{n=1}^{r} \overline{x[n]} y[n]$ en $\mathbb{C}^r$. Se sabe, véase por ejemplo \cite{Prand1}, que el conjunto $\{ w_k \}_{k=1}^r,$ donde \begin{equation*}
    w_k = (W_r^{-k}, W_r^{-2 k}, \ldots, W_r^{-(r-1) k}, 1) \in \mathbb{C}^r, w=1,\ldots,r,
\end{equation*}
es una base ortogonal de $\mathbb{C}^r$. En efecto,
\begin{equation*}
    \langle w_m, w_n \rangle = \sum_{i=1}^{r} \overline{W_r^{-m i}} W_r^{-n i} = \sum_{i=1}^{r} W_r^{(m-n)i} = \begin{cases}
        \hfil r & \text{si } m=n \\[7pt]
        \hfil \displaystyle\frac{1-W_r^{(m-n)r}}{1-W_r^{m-n}} = 0 & \text{en otro caso}.
    \end{cases}
\end{equation*}

Por otro lado, si $x \in \mathbb{C}^r$, existen por tanto $X(k) \in \mathbb{C}, k=1,\ldots,r$, tales que $r x = \sum_{k=1}^{r} X[k] w_k$. Pero
\begin{equation*}
    \langle w_k, x \rangle = \left\langle w_k, \frac{1}{r} \sum_{n=1}^{r} X[n] w_n \right\rangle = \frac{1}{r} \sum_{n=1}^{r} \left\langle w_k, X[n] w_n \right\rangle = \frac{X[k]}{r} r = X[k].
\end{equation*}

En consecuencia,
\begin{equation}\label{eq:analysis}
    X[k] = \sum_{n=1}^{r} \overline{W_r^{-n k}} x[n] = \sum_{n=1}^{r} x[n] W_r^{n k}, \forall k=1,\ldots,r
\end{equation}
y
\begin{equation}\label{eq:synthesis}
    x[n] = \frac{1}{r} \sum_{k=1}^{r} X[k] w_k[n] = \frac{1}{r} \sum_{k=1}^{r} X[k] W_r^{- n k}, \forall n=1,\ldots,r.
\end{equation}

La ecuación \eqref{eq:analysis} se conoce como fórmula de \emph{análisis} o \emph{coeficientes de Fourier discretos} y la ecuación \eqref{eq:synthesis} como fórmula de \emph{síntesis} o \emph{reconstrucción} del elemento $x \in \mathbb{C}^r$.
\bigskip

Considere la fórmula de síntesis. Dado que $W_r^{(n + i r) k} = W_r^{n k}$ para cada $i \in \mathbb{Z}$, entonces $x[n + i r] = x[n]$ para todos $i \in \mathbb{Z}$, $n=1,\ldots,r$, de manera que se puede extender $x \in \mathbb{C}^r$ a una señal periódica a todo $\mathbb{Z}$ de forma natural. Si $\tilde{x}$ es una señal periódica y se calcula la fórmula de reconstrucción con los valores de su periodo, dicha extensión coincide con $\tilde{x}$ en todo $\mathbb{Z}$, así que se puede escribir
\begin{equation}\label{eq:analysis1}
    \tilde{X}[k] = \sum_{n=1}^{r} \overline{W_r^{-n k}} \tilde{x}[n] = \sum_{n=1}^{r} \tilde{x}[n] W_r^{n k}, \forall k \in \mathbb{Z}
\end{equation}
y
\begin{equation}\label{eq:synthesis1}
    \tilde{x}[n] = \frac{1}{r} \sum_{k=1}^{r} \tilde{X}[k] w_k[n] = \frac{1}{r} \sum_{k=1}^{r} \tilde{X}[k] W_r^{- n k}, \forall n \in \mathbb{Z},
\end{equation}

donde la fórmula \eqref{eq:analysis} se ha extendido implicitamente  a todos los enteros. Dicha extensión, la ecuación $\eqref{eq:analysis1}$, se llama \emph{transformada discreta de Fourier} de la señal $\tilde{x}$. La ecuación \eqref{eq:synthesis1} también será llamada fórmula de \emph{síntesis} o \emph{reconstrucción} de la señal $\tilde{x}$.

\subsection{Señales simétricas}

Las señales simétricas son un análogo a las funciones aritméticas pares estudiadas en el capítulo anterior.

\begin{definition}[Señal simétrica]
Una señal $x$ se dice \emph{simétrica módulo} $r$ si $x[n] = x[(n,r)]$ para cada $n \in \mathbb{Z}$.
\end{definition}

Toda señal simétrica módulo $r$ es periódica módulo $r$ y la demostración es idéntica a la de la \Cref{prop:mod->per}. Una de las principales diferencias entre las señales periódicas y las simétricas, es que las últimas en general tienen una imagen más pequeña, más específicamente, la cardinalidad de su imagen no puede exceder $d(r)$.

\begin{proposition}\label{prop:fou3}
Si $x : \mathbb{Z} \longrightarrow \mathbb{C}$ es una señal simétrica módulo $r$, entonces $| x (Z) | \le d(r)$, donde $d(r)$ es el número de divisores de $r$.
\end{proposition}

\begin{proof}
Para cada divisor positivo $d$ de $r$, sea
\begin{equation*}
    S_d = \left\{ \left( \frac{r}{d} \right) \alpha \std (\alpha,d) = 1, 1 \le \alpha \le d \right\}.
\end{equation*}
Por la proposición ??, se tiene que $\bigcup_{d \mid r} S_d = \{ 1,2,\ldots,r \}$.
\bigskip

Además, si $(r/d)\alpha \in S_d$ entonces
\begin{equation}\label{eq:res1}
    \left( \left( \frac{r}{d} \alpha \right),r \right) = \left( \left( \frac{r}{d} \right),r \right) = \frac{r}{d},
\end{equation}
pues $(\alpha,d)=1$ y $d \mid r$. Más aún, si $x$ es simétrica y $n, m \in S_e$ entonces $\alpha[n]=\alpha[m]$. En efecto, se debe tener que $n = (r/d) \alpha$ y $m = (r/d) y$ con $(\alpha,d)=(\beta,d)=1$ y $1 \le \alpha,\beta \le d$. Luego, como $x$ es simétrica,
\begin{equation*}
    x[n] = x \left[ \frac{r}{d} \alpha \right] = x \left[ \left( \frac{r}{d} \alpha, r \right) \right] = x \left[ \frac{r}{d} \right]
\end{equation*}
y
\begin{equation*}
    x[m] = x \left[ \frac{r}{d} \beta \right] = x \left[ \left( \frac{r}{d} \beta, r \right) \right] = x \left[ \frac{r}{d} \right],
\end{equation*}
por tanto $x[n]=x[m]$. Como $x$ es periódica con periodo $r$, hace falta conocer sus valores únicamente en los enteros $1,\ldots,r$. Así,
\begin{equation*}
    \{ x[1], \ldots, x[r] \} = \bigcup_{d \mid r} \left\{ x \left [\left( \frac{r}{d} \right) \alpha \right] \std (\alpha,d)=1, 1 \le \alpha \le d \right\} = \bigcup_{d \mid r} \{ x[r/d] \},
\end{equation*}
donde el último conjunto tiene a lo sumo $d(r)$ elementos.
\end{proof}

Los conjuntos $S_d$ de la proposición anterior de llaman \emph{sistemas de residuos} de los divisores $d$ de $r$ \cite{Sam1}. Al igual que con las sumas de Ramanujan, cada señal par se puede expresar como combinación lineal de señales indicadoras sobre los sistemas de residuos.

\begin{definition}
Si $d$ es un divisor de $r$, se define $h_{r,d} : \{ 1,\ldots,r \} \longrightarrow \mathbb{C}$ como
\begin{equation*}
    h_{r,d}[n] = \begin{cases}
        \hfil 1 & \text{si } n \in S_d \\
        \hfil 0 & \text{en otro caso.}
    \end{cases}
\end{equation*}
\end{definition}

\begin{proposition}
Para cada $n \in \{ 1,2,\ldots,r \}$ se tiene $h_{r,d}[n]=h_{r,d}[(n,r)]$.
\end{proposition}

\begin{proof}
Sea $d$ un divisor positivo de $r$ y sea $n \in \{ 1,2,\ldots,r \}$. Basta probar que $n \in S_d$ si y sólo si $(n,r) \in S_d$.
\bigskip

Si $n \in S_d$, entonces $n=(r/d) x$ con $(x,d)=1$ y $1 \le x \le d$. Además, como $(n,r) \mid n$ y $(n,r) \mid r$, existen $a,b \in \mathbb{N}$ tales que $n = (n,r) a$  y $r = (n,r) b$. Luego $(n,r) a = (r/d) x$ y por tanto $(n,r) a d = (n,r) b x$, así que $a d = b x$. Dado que $(a,b)=(n/(n,r), r/(n,r))=1$ y $a \mid b x$, entonces $a \mid x$ y por tanto $x = a q$, para algún $q \in \mathbb{N}$. Se sigue que $(n,r) = (r/d) q$ y además $1 \le q \le a q = x \le d$. Luego $(n,r) \in S_d$.
\bigskip

Recíprocamente, si $(n,r) \in S_d$, escríbase $(n,r) = (r/d) y$ con $(y,d)=1$ y $1 \le y \le d$. Como $(n,r) \mid r$ entonces $(r/d) y \mid r$, luego $r = (r/d) y q$ para algún $q \in \mathbb{N}$ y por tanto $r d = r y q$, así que $d = y q$, es decir $y \mid d$. Se debe tener por tanto que $y=(y,d)=1$, luego $(n,r)=(r/d)$ y como $n=(n,r)a$ para algún $a \in \mathbb{N}$, entonces $n = (r/d)a$. Además, como $n \le r$, entonces $(r/d) a \le r$, luego $r a \le r d$, es decir, $1 \le a \le d$ y por tanto $n \in S_d$.
\end{proof}

Las funciones $h_{r,d}$ se pueden extender de forma natural a una función simétrica a todo $\mathbb{Z}$ definiendo $h_{r,d}[n] = h_{r,d}[(n,r)]$ para cada $n \in \mathbb{Z}$ y abusando de la notación se le seguirá denotando $h_{r,d}$. Sin embargo, en lo que sigue será suficiente considerar únicamente su restricción a $\{ 1,2,\ldots,r \}$.

\begin{proposition}\label{prop:fou2}
Cualquier señal simétrica $x$ módulo $r$ se puede escribir como
\begin{equation}\label{eq:res2}
    x[n] = \sum_{d \mid r} x \left[ \frac{r}{d} \right] h_{r,d}[n],
\end{equation}
para todo $n \in \mathbb{Z}$.
\end{proposition}

\begin{proof}
Sin pérdida de generalidad supóngase que $n \in \{ 1,2,\ldots,r \}$. Entonces existe un único $e \mid r$ tal que $n \in S_e$, es decir, tal que $n = (r/e) \alpha$, con $(e,\alpha)=1$ y $1 \le \alpha \le e$. Luego
\begin{align*}
    \sum_{d \mid r} x \left[ \frac{r}{d} \right] h_{r,d}[n] = \sum_{\substack{d \mid r \\ n \in S_d}} x \left[ \frac{r}{d} \right] & = x \left[ \frac{r}{e} \right] = x \left[ \left( \frac{r}{e},r \right) \right] \\
                                                            & = x \left[ \left( \frac{r}{e} \alpha,r \right) \right] = x [(n,r)] = x[n]
\end{align*}
por \eqref{eq:res1} y por ser $x$ simétrica.
\end{proof}

La siguiente proposición muestra que las sumas de Ramanujan aparecen naturalmente como los coeficientes de Fourier clásicos en la fórmula de síntesis \eqref{eq:synthesis1} para las señales $h_{r,d}$.

\begin{proposition}
Los coeficientes de Fourier de la señal $h_{r,d}$ están dados por
\begin{equation*}
    H_{r,d}[k] =  c_d[k].
\end{equation*}
donde $c_d$ es la suma de Ramanujan módulo $d$.
\end{proposition}

\begin{proof}
Utilizando \eqref{eq:analysis} para calcular los coeficientes de $h_{r,d}$ en \eqref{eq:synthesis1}, se tiene que:
\begin{equation}\label{eq:res3}
\begin{split}
    H_{r,d}[k] = \sum_{n=1}^{r} h_{r,d}(n) W_r^{n k} = \sum_{\substack{n=1 \\ n \in S_d}}^{r} W_r^{n k} & = \sum_{\substack{n=1 \\ n=(r/d) x \\ 1 \le x \le d \\ (x,d)=1}}^{r} W_r^{n k} \\
    &= \sum_{\substack{x = 1 \\ (x,d)=1}}^{d} W_d^{x k} = \sum_{\substack{x = 1 \\ (x,d)=1}}^{d} W_d^{-x k}  = c_d[k],
\end{split}
\end{equation}
pues $e^{-i(2 \pi /r) n k}=e^{-i (2 \pi /d) x k}$ si $n=(r/d)x$.
\end{proof}

\begin{proposition}\label{prop:fou4}
Si $x$ es una señal simétrica, sus coeficientes de Fourier discretos están dados por
\begin{equation*}
   X[k] = \sum_{d \mid r} x \left[ \frac{r}{d} \right] c_d[k].
\end{equation*}
\end{proposition}

\begin{proof}
Utilizando las ecuaciones ecuaciones \eqref{eq:analysis}, \eqref{eq:res2} y \eqref{eq:res3} se tiene que
\begin{align*}
    X[k] = \sum_{n=1}^{r} x[n] W_r^{n k} = \sum_{n=1}^{r} \sum_{d \mid r} x \left[ \frac{r}{d} \right] & h_{r,d}[n] W_r^{n k} = \sum_{d \mid r}x \left[ \frac{r}{d} \right] \sum_{n=1}^{r} h_{r,d}[n] W_r^{n k} \\
                                         & = \sum_{d \mid r} x \left[ \frac{r}{d} \right] H_{r,d}[k] = \sum_{d \mid r} x \left[ \frac{r}{d} \right] c_d[k].
\end{align*}
\end{proof}

\begin{remark}
Dado que $c_d[k] = c_d[(k,d)]$, la proposición anterior implica que la transformada discreta de Fourier de una señal simétrica módulo $r$ también es simétrica módulo $r$. Además tal ecuación define una extensión natural de $c_d$ a todo $\mathbb{Z}$.
\end{remark}

\begin{proposition}
Si $x$ es una señal simétrica módulo $r$, entonces
\begin{equation*}
    x[n] = \frac{1}{r} \sum_{d \mid r} X \left[ \frac{r}{d} \right] c_d[n], \forall n \in \mathbb{Z}.
\end{equation*}
\end{proposition}

\begin{proof}
Considere la transformada de Fourier discreta $X$ de $x$. Dado que $X$ también es simétrica módulo $r$, la \cref{prop:fou2} implica que
\begin{equation*}
    X[n] = \sum_{d \mid r} x \left[ \frac{r}{d} \right] h_{r,d}[n], \forall n \in \mathbb{Z},
\end{equation*}
luego, usando la fórmula de reconstrucción \eqref{eq:synthesis1}, se tiene
\begin{align*}
    x[n] = \frac{1}{r} \sum_{k=1}^{r} X[k] W_r^{-n k} & = \frac{1}{r} \sum_{k=1}^{r} \sum_{d \mid r} X \left[ \frac{r}{d} \right] h_{r,d} [k] W_r^{-n k} \\
         & = \frac{1}{r} \sum_{d \mid r} X \left[ \frac{r}{d} \right] \sum_{k=1}^{r} h_{r,d}[k] W_r^{-n k} = \frac{1}{r} \sum_{d \mid r} X \left[ \frac{r}{d} \right] c_d[n],
\end{align*}
para todo $n \in \mathbb{Z}$.
\end{proof}

\begin{example}
Como ejemplo concreto, considere una señal $x$ arbitraria y simétrica módulo $r=10$. Por la \cref{prop:fou3}, $x$ puede tomar a lo sumo $d(10)=4$ valores y estos estarán agrupados en los sistemas de residuos de los divisores de $10$.
\bigskip

Específicamente, se tiene $S_1 = \{ 10 \}, S_2 = \{ 5 \}, S_5 = \{ 2, 4, 6, 8 \}, S_{10} = \{ 1, 3, 5, 7 \}$. Por tanto, $x[2]=x[4]=x[6]=x[8]$ y $x[1]=x[3]=x[5]=x[7]$. Luego los coeficientes $X[k]$, dados por \ref{prop:fou4}, son
\begin{align*}
    X[1] = X[3] = X[5] = X[7] & = \sum_{d \mid 10} x \left[ \frac{10}{d} \right] c_d[1] = x[10] - x[5] - x[2] + x[1] \\
    X[2] = X[4] = X[6] = X[8] & = \sum_{d \mid 10} x \left[ \frac{10}{d} \right] c_d[2] = x[10] + x[5] - x[2] - x[1] \\
    X[5] & = \sum_{d \mid 10} x \left[ \frac{10}{d} \right] c_d[5] = x[10] - x[5] + 4 x[2] - 4 x[1] \\
    X[10] & = \sum_{d \mid 10} x \left[ \frac{10}{d} \right] c_d[10] = x[10] + x[5] + 4 x[2] + 4 x[1]. \\
\end{align*}
El cálculo de estos coeficientes se reduce entonces a calcular $c_d[k]$ para cada $d \mid r$ y $k \mid r$. A diferencia de la transformada discreta de Fourier clásica, éstos cálculos no requieren la implementación de las funciones $\sin$ y $\cos$ en un microprocesador. Las sumas de Ramanujan se puede implementar o bien usando la \cref{prop:ram4}, implementando sólo la función de Möbius, una una suma condicional y el máximo común divisor, o bien utilizando la fórmula cerrada sin sumatorias \ref{lem:holder} (fórmula de Hölder), pero implementando la función de Möbius, la función $\varphi$ de Euler y el máximo común divisor.
\end{example}

\begin{example}
Suponga que se quiere obtener una señal simétrica $x$ módulo $3$ tal que $x[1]=x[2]=1$ y $x[3]=3$. Calculando sus coeficientes de Fourier, se tiene
\begin{align*}
    X[1] = X[2] & = \sum_{d \mid 3} x \left[ \frac{3}{d} \right] c_d[1] = x[3] - x[1] = 2 \\
    X[2] & = \sum_{d \mid 3} x \left[ \frac{3}{d} \right] c_d[3] = x[3] + 2 x[1] = 5.
\end{align*}
Utilizando la fórmula de reconstrucción, se puede escribir
\begin{equation*}
x[n] = \frac{1}{3} \sum_{d \mid 3} X \left[ \frac{3}{d} \right] c_d[n] = \frac{1}{3} \left( 5 c_1[n] + 2 c_3[n] \right), \forall n \in \mathbb{Z}.
\end{equation*}

\begin{figure}
\centering
    \begin{tikzpicture}
        \begin{axis}[
            width=\textwidth,
            height=8.0cm,
            ymin=-3.5,
            ymax=3.5,
            xmin=-12,
            xmax=12,
            tick style={draw=none},
            axis x line=middle,
            axis line style={-}
        ]
            \addplot+[black, mark options={fill=black}, ycomb] table [col sep=comma] {plots/plot1.csv};
            \addplot[black, thick] coordinates {(-12,3.5) (12,3.5)};
            \addplot[black, thick] coordinates {(-12,-3.5) (12,-3.5)};
        \end{axis}
    \end{tikzpicture}
\caption{$x[n] = \frac{1}{3} (5 c_1[n] + 2 c_3 [n])$}
\label{fig:plot1}
\end{figure}

\begin{figure}
\centering
    \begin{tikzpicture}
        \begin{axis}[
            width=\textwidth,
            height=8.0cm,
            ymin=-6,
            ymax=6,
            xmin=-12,
            xmax=12,
            tick style={draw=none},
            axis x line=middle,
            axis line style={-}
        ]
            \addplot+[black, mark options={fill=black}, ycomb] table [col sep=comma] {plots/plot2.csv};
            \addplot[black, thick] coordinates {(-12,6) (12,6)};
            \addplot[black, thick] coordinates {(-12,-6) (12,-6)};
        \end{axis}
    \end{tikzpicture}
\caption{$X[n] = 3 c_1[k] + c_3[k]$}
\label{fig:plot2}
\end{figure}

Se puede calcular también su transformada discreta de Fourier usando la \cref{prop:fou4} para todo entero $k$,
\begin{equation*}
    X[k] = \sum_{d \mid 3} x \left[ \frac{3}{d} \right] c_d[k] = 3 c_1[k] + c_3[k], \forall k \in \mathbb{Z}.
\end{equation*}
La gráfica de esta señal se muestra en la \Cref{fig:plot2}.
\end{example}
