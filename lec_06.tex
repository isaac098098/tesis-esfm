%%% Funciones pares
\newpage
\section{Funciones pares}

Al estudiar el espacio de funciones aritméticas se puede hacer una analogía con la teoría de Fourier del análisis para funciones definidas en todo el plano real o complejo, para la cuál se necesitará la noción de periodicidad. En este capítulo se considerarán dos clases de funciones aritméticas que capturan esta noción y se probará que son equivalentes. También se expondran resultados análogos a los de análisis respecto a funciones periódicas.

\begin{remark}
Durante todo el capítulo se supondrá que $r$ es un entero positivo arbitrario pero fijo.
\end{remark}

\begin{definition}[Función par]
Una función aritmética se dice \textbf{par} $\Mod{r}$ si $f(n)=f((n,r))$, donde $(m,r)$ es el máximo común divisor de $n$ y $r$, para cada $n \in \mathbb{N}$.
\end{definition}

\begin{definition}[Función periódica]
Una función aritmética se dice \textbf{periódica} con periodo $r$ (o periódica $\Mod{r}$) si $m, n \in \mathbb{N}$ y $m \equiv n \pmod{r}$ implica que $f(m)=f(n)$.
\end{definition}

La siguiente proposición es una consecuencia inmediata de las definiciones anteriores.

\begin{proposition}
Toda función par $\Mod{r}$ es periódica con periodo $r$.
\end{proposition}
\begin{proof}
Si $m \equiv n \pmod{r}$ entonces $r \mid m-n$, por tanto existe $q \in \mathbb{Z}$ tal que $m-n=q r$. Por demostrar que $(n,r)=(m,r)$. En efecto, como $(n,r) \mid n$ y $(m,r) \mid r$, entonces $(m,r) \mid n+qr=m$, luego $(n,r) \mid (m,r)$. Análogamente, se tiene que $(m,r) \mid m$ y $(m,r) \mid r$, por lo que $(m,r) \mid m-q r=n$, luego $(m,r) \mid (n,r)$. Se sigue que $(n,r)=(m,r)$ y por tanto $f(n)=f((n,r))=f((m,r))=f(m)$.
\end{proof}

\subsection{Sumas de Ramanujan}

\begin{definition}[Sumas de Ramanujan]
Se define la función aritmética $c_r$ como
\begin{equation*}
    c_r(n) = \sum_{d \mid (n,r)} \mu \left( \frac{r}{d} \right) d.
\end{equation*}
Esta función será referida como la suma de Ramanujan módulo $r$ o simplemente suma de Ramanujan cuando no haya riesgo de confusión.
\end{definition}

\begin{proposition}
Algunas propiedades de la sumas de Ramanujan son las siguientes:
\begin{enumerate}[label=\textnormal{(\arabic*)},ref=\textnormal{\arabic*}]
\item $c_1 = \mathbf{1}$
\item $c_r(1) = \mu(r)$
\item $c_r(n) \le \max \{ \sigma(r), \sigma(n) \}$
\item \label{it:ram1} $c_r(n)$ es una función multiplicativa de $r$
\item Si $p$ es primo y $m$ es un entero positivo, entonces
    \begin{equation*}
        c_{p^m}(n) = \begin{cases}
            \hfil p^m - p^{m-1} & \text{si } p^m \mid n \\
            \hfil -p^{m-1} & \text{si } p^{m-1} \text{ pero } p^m \centernot\mid n \\
            \hfil 0 & \text{si } p^{m-1} \centernot\mid n.
        \end{cases}
    \end{equation*}
\end{enumerate}
\end{proposition}

\begin{proof}
\begin{enumerate}[label=\textnormal{(\arabic*)}]
\item Para cada $n \in \mathbb{N}$ se tiene que $(n,1)=1$ y por tanto
    \begin{equation*}
        c_1(n) = \sum_{d \mid (n,1)} \mu \left( \frac{1}{d} \right) d = \mu(1) 1 = 1.
    \end{equation*}
\item De manera similar,
    \begin{equation*}
        c_r(1) = \sum_{d \mid (1,r)} \mu \left( \frac{r}{d} \right) d = \mu(r) 1 = \mu(r).
    \end{equation*}
\item Por definición se tiene que $\sigma(k) = \sum_{d \mid k} d$. Además $\mu(k) \le 1$ para todo $k \in \mathbb{N}$, luego
    \begin{equation*}
        c_r(n) = \sum_{d \mid (n,r)} \mu \left( \frac{r}{d} \right) d \le \sum_{d \mid (n,r)} d = \sum_{\substack{d \mid n \\ d \mid r}} d \le \sum_{d \mid n} d, \sum_{d \mid r} d \le \max \{ \sigma(n),\sigma(r) \}.
    \end{equation*}
\item Defínase
    \begin{equation*}
        \eta_r(n) = \begin{cases}
            \hfil r & \text{si } r \mid n \\
            \hfil 0 & \text{en otro caso}.
        \end{cases}
    \end{equation*}
Se tiene que la función $\eta_\Box(n)$ es multiplicativa para $n$ fijo. En efecto, si $r,s \in \mathbb{N}$ son tales que $(r,s)=1$, entonces
\begin{equation*}
    \eta_{r s}(n) = \begin{cases}
        \hfil r s & \text{si } r s \mid n \\
        \hfil 0 & \text{en otro caso,}
    \end{cases}
\end{equation*}
pero $r s \mid n$ si y sólo si $r \mid n$ y $s \mid n$. En efecto, si $r s \mid n$ es claro que $r \mid n$ y $s \mid n$. Supóngase que $r \mid n$ y $s \mid n$, de tal manera que existen $q_1, q_2 \in \mathbb{Z}$ tales que $n=r q_1=s q_2$. Como $(r,s)=1$, también existen $x, y \in \mathbb{Z}$ tales que $1=r x + s y$, luego $n=n r x + n s y$, por lo que $n= r s (q_2 x + q_1 y)$, es decir, $r s \mid n$. Luego, si $r s \mid n$, entonces
\begin{equation*}
    \eta_{r s}(n) = r s = \eta_r(n) \eta_s(n),
\end{equation*}
y si $r s \centernot\mid$ entonces $r \centernot\mid n$ y $s \centernot\mid n$, por lo que
\begin{equation*}
    \eta_{r s}(0) = 0 = \eta_r(n) \eta_s(n).
\end{equation*}

Por otro lado, se tiene que
\begin{equation*}
    \sum_{d \mid r} \mu \left( \frac{r}{d} \right) \eta_d(n) = \sum_{\substack{d \mid r \\ d \mid n}} \mu \left( \frac{r}{d} \right) d = \sum_{d \mid (n,r)} \mu \left( \frac{r}{d} \right) d = c_r(n),
\end{equation*}
es decir, $c_\Box(n)=\mu*\eta_\Box(n)$. Luego $c_\Box(n)$ debe ser multiplicativa para $n$ fijo, por ser producto de funciones multiplicativas.
\item Tenemos los siguientes casos:
\begin{itemize}
\item Si $p^m \mid n$, entonces $(n,p^m)=p^m$, luego
    \begin{equation*}
        c_{p^m}(n) = \sum_{d \mid p^m} \mu \left( \frac{p^m}{d} \right) d = \mu(1) p^m + \mu(p) p^{m-1} = p^m - p^{m-1},
    \end{equation*}
    pues $\mu(p^i)=0$ para toda $i > 1$.
\item Si $p^{m-1} \mid n$ pero $p^m \centernot\mid n$, entonces $(n,p^m)=p^{m-1}$. En efecto, se tiene que $p^{m-1} \mid p^m$ y además $p^{m-1} \mid n$ por hipótesis. Si $e \in \mathbb{Z}$ es tal que $e \mid p^{m}$ y $e \mid n$, entonces $e=p^i$, para algún $0 \le i \le m-1$, pues $p^m \centernot\mid n$, por tanto $e \mid p^{m-1}$. Esto prueba que $(p^m,n)=p^{m-1}$, así
    \begin{equation*}
        c_{p^m}(n) = \sum_{d \mid p^{m-1}} \mu \left( \frac{p^m}{d} \right) d = \mu(p) p^{m-1} = -p^{m-1}
    \end{equation*}
    
\item Finalmente, si $p^{m-1} \centernot\mid n$, entonces $p^m \centernot\mid n$. Además, $(n,p^m) \mid p^m$, por tanto $(n,p^m)=p^i$ para algún $0 \le i \le m$. Más aún, por la hipótesis se debe tener que $0 \le i \le m-2$. Luego
    \begin{equation*}
        c_{p^m}(n) = \sum_{d \mid p^i} \mu \left( \frac{p^m}{d} \right) d = \mu(p^m)1+ \mu(p^{m-1}) p + \cdots + \mu(p^{m-i}) p^i = 0,
    \end{equation*}
    pues $i \le m-2$ implica que $2 \le m-i$ y por tanto $\mu(p^i)=0$.
\end{itemize}
\end{enumerate}
\end{proof}

Del la demostración del \cref{it:ram1} se puede rescatar el siguiente corolario, usando la inversión de Möbius (\cref{cor:mob1}).

\begin{corollary}
Para cada $n \in \mathbb{N}$ fijo se tiene
\begin{equation*}
    \sum_{d \mid r} c_d(n) = \eta_r(n) = \begin{cases}
        \hfil r & \text{si } r \mid n \\
        \hfil 0 & \text{en otro caso}.
    \end{cases}
\end{equation*}
\end{corollary}

Las sumas de Ramanujan gozan de la siguiente propiedad de ``ortogonalidad''.

\begin{lemma}\label{lem:ram4}
Si $r$ y $s$ dividen a $k$, entonces
\begin{equation*}
    \sum_{d \mid k} c_r(k/d) c_d(k/s) = \begin{cases}
        \hfil k & \text{si } r = s \\
        \hfil 0 & \text{en otro caso}.
    \end{cases}
\end{equation*}
\end{lemma}
\begin{proof}
Si $r$ y $s$ dividen a $k$, entonces
\begin{equation} \label{eq:ram2}
\begin{split}
\sum_{d \mid k} c_r (k/d) c_d (k/s) &= \sum_{d \mid k} c_d (k/s) \sum_{d' \mid (k/d,r)} \mu (r/d') d' \\
                                                                              &= \sum_{d \mid k} c_d (k/s) \sum_{\substack{d' \mid r \\ d' \mid k/d}} \mu(r/d') d' \\
                                                                              &= \sum_{\substack{d \mid k \\ d' \mid r \\ d' \mid k/d}} c_d (k/s) \mu (r/d') d' \\
                                                                              &= \sum_{\substack{d \mid k/d' \\ d' \mid r \\ d' \mid r}} c_d(k/s) \mu(r/d')  d' \\
                                                                              &= \sum_{\substack{d' \mid r \\ d' \mid k}} \mu(r/d') d' \sum_{d \mid k/d'} c_d(k/s) \\
                                                                              &= \sum_{d' \mid (k,r)} \mu(r/d) d' \eta_{k/d'} (k/s), \text{por el corolario anterior} \\
                                                                              &= \sum_{d' \mid r} \mu(r/d) d' \eta_{k/d'} (k/s),
\end{split}
\end{equation}
dado que $(k,r)=r$ por ser $r$ divisor de $k$ y dado que los conjuntos $\{ d,d' \in \mathbb{N} \std d \mid k, d' \mid r, d' \mid k/d \}$ y $\{ d,d' \in \mathbb{N} \std d \mid k/d', d' \mid r, d' \mid k \}$ son iguales. En efecto, si $d \mid k$ entonces $k/d$ es un entero, lueg $d' \mid k/d$ implica que $k/d=d' q'$, luego $k=d' q' d$, por tanto $d \mid k/d'$ y $d' \mid k$.
\bigskip

Recíprocamente, si $d' \mid k$ entonces $k/d'$ es un entero, luego $d \mid k/d'$ implica que $k/d'=dq$, por tanto $k=d q d'$, por tanto $d \mid k$ y $d' \mid k/d$.
\bigskip

Si $s \centernot\mid r$ entonces $s \centernot\mid d'$ y por tanto $k/d' \centernot\mid k/s$. En efecto, pues si $s \mid d'$, como $d' \mid r$ entonces se tendría que $s \mid r$ por transitividad. Además, si $k/d' \mid k/s$ se tendría que $s \mid d'k$. Luego la suma \eqref{eq:ram2} se anula si $s \centernot\mid r$ y en particular si $r \ne s$, pues en este caso se tiene que $\eta_{k/d'}(k/s)=0$ para cada $d' \mid r$. 
\bigskip

Si $s \mid r$ entonces la suma \eqref{eq:ram2} es igual a
\begin{equation*}
\begin{split}
    \sum_{\substack{d' \mid r \\ k/d' \mid k/s}} \mu(r/d') d' \frac{k}{d'} &= \sum_{\substack{d' \mid r \\ k/d' \mid k/s}} \mu(r/d') k \\
    &= \sum_{\substack{d' \mid r \\ s \mid d'}} \mu(r/d') k \\
    &= k \sum_{\substack{d' \mid r \\ d'=se}} \mu(r/se) \\
    &= k \sum_{e \mid r/s} \mu(r/se) \\
    &= k \sum_{se \mid r} \mu(r/se) = \begin{cases}
        \hfil k & \text{si } r=s \\
        \hfil 0 & \text{en otro caso,}
    \end{cases}
\end{split}
\end{equation*}
pues $k/d' \mid k/s$ si y sólo si $s \mid d'$.
\end{proof}

\begin{lemma}
Si $d \mid r$ entonces $c_d(n)=c_d((n,r))$.
\end{lemma}
\begin{proof}
Si $d \mid r$ entonces $(n,d)=((n,r),d)$. En efecto, dado que $(n,d) \mid n$ y $(n,d) \mid d$, entonces $(n,d) \mid n$, $(n,d) \mid d$ y $(n,d) \mid r$, por lo que $(n,d) \mid (n,r)$ y $(n,d) \mid d$, es decir, $(n,d) \mid ((n,r),d)$. Recíprocamente se tiene que $((n,r),d) \mid n$ y $((n,r),d) \mid d$, así que $((n,r),d) \mid (n,d)$. Se sigue que $(n,d)=((n,r),d)$. Luego
\begin{equation*}
    c_d(n) = \sum_{e \mid (n,d)} \mu(d/e) e = \sum_{e \mid ((n,r),d)} \mu(d/e) e = c_d((n,r)).
\end{equation*}
\end{proof}

\begin{corollary}
La suma de Ramanujan módulo $r$ es par $\Mod{r}$.
\end{corollary}

El lema anterior permite probar uno de los resultados importantes de este capítulo, el cuál establece la existencia de una expansión finita de cualquier función par $\Mod{r}$, con sumas de Ramanujan como coeficientes.

\begin{theorem}
Toda función $f$ par $\Mod{r}$ tiene una expansión de la forma
\begin{equation}\label{eq:ram3}
    f(n) = \sum_{d \mid r} \alpha(d) c_n(n),
\end{equation}
y recíprocamente, toda función aritmética de esta forma es par $\Mod{r}$. Los coeficientes $\alpha(d)$ están dados por
\begin{equation*}
    \alpha(d) = \frac{1}{r} \sum_{e \mid r} f \left( \frac{r}{e} \right) c_e \left( \frac{r}{d} \right),
\end{equation*}
o por la fórmula equivalente,
\begin{equation*}
    \alpha(d) = \frac{1}{r \phi(d)} \sum_{m=1}^{r} f(m) c_d(m),
\end{equation*}
donde $\phi$ es la función phi de Euler.
\end{theorem}
\begin{proof}
Es claro que toda función de la forma \eqref{eq:ram3} es par $\Mod{r}$, pues por el lema anterior si $d \mid r$ entonces $c_d(n)=c_d((n,r))$. Nótese que
\begin{equation*}
\begin{split}
    \sum_{d \mid r} \alpha(d) c_d(n) &= \sum_{d \mid r} \left( \frac{1}{r} \sum_{e \mid r} f \left( \frac{r}{e} \right) c_e \left( \frac{r}{d} \right) \right) c_d(n) \\
                                     &= \frac{1}{r} \sum_{e \mid r} f \left( \frac{r}{e} \right) \sum_{d \mid r} c_e \left( \frac{r}{d} \right) c_d(n) \\
                                     &= \frac{1}{r} \sum_{e \mid r} f \left( \frac{r}{e} \right) \sum_{d \mid r} c_e \left( \frac{r}{d} \right) c_d((n,r)) \\
                                     &= \frac{1}{r} f \left( \frac{r}{q} \right) r = f((n,r)) = f(n),
\end{split}
\end{equation*}
por el \Cref{lem:ram4}, donde $r=(n,r) q$, para algún $q \in \mathbb{N}$ y donde la última igualdad se cumple por ser $f$ par $\Mod{r}$.
\bigskip

Por otro lado, de la demostración de la \cref{eq:gauss1} se puede rescatar el hecho de que el conjunto $\{ 1,2,\ldots,r \}$ es igual a $\bigcup_{e \mid r} \{ rx/e \std (x,e)=1, 1 \le x \le e \}$ y todos los conjuntos son disjuntos a pares, por tanto
\begin{equation*}
\begin{split}
    \frac{1}{r \phi(d)}\sum_{m=1}^{r} f(m) c_d(m) &= \frac{1}{r \phi(d)}\sum_{e \mid r} \sum_{\substack{(x,e)=1 \\ 1 \le x \le e}} f \left( \frac{rx}{e} \right) c_d \left( \frac{rx}{e} \right) \\
                               &= \frac{1}{r \phi(d)}\sum_{e \mid r} \sum_{\substack{(x,e)=1 \\ 1 \le x \le e}} f \left( \left( \frac{rx}{e},r \right) \right) c_d \left( \left( \frac{rx}{e},r \right) \right) \\
                               &= \frac{1}{r \phi(d)}\sum_{e \mid r} \sum_{\substack{(x,e)=1 \\ 1 \le x \le e}} f \left( \frac{r}{e} \right) c_d \left( \frac{r}{e} \right) \\
                               &= \frac{1}{r \phi(d)}\sum_{e \mid r} f \left( \frac{r}{e} \right) c_d \left( \frac{r}{e} \right) \phi(e) \\
                               &= \frac{1}{r \phi(d)}\sum_{e \mid r} f \left( \frac{r}{e} \right) c_e \left( \frac{r}{d} \right) \phi(d) \\
                               &= \frac{1}{r}\sum_{e \mid r} \left( \frac{r}{e} \right) c_e \left( \frac{r}{d} \right)
\end{split}
\end{equation*}
por ser $f$ par $\Mod{r}$. Además $(rx/e,r)=r/e$, pues $(x,e)=1$ implica que $(r/e)(x,e)=r/e$, y como $r/e$ es un entero positivo, entonces $(rx/e,r)=r/e$. Y la penúltima igualdad se cumple por el {\renewcommand\ttdefault{cmtt}\texttt{cor:mcd1}} y la fórmula de Hölder.
\end{proof}
