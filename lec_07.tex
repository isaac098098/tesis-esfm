%%% Divisibilidad
\newpage
\appendix
\section{Divisibilidad}

\begin{proposition}
Si $r \in \mathbb{N}$, $r=e q_1$, $r= d q_2$, $d=(q_1,d)k_1$ y $e=(q_2,e) k_2$ entonces $k_1=k_2$.
\end{proposition}
\begin{proof}
Nótese que $q_1,q_2$ enteros positivos, además $(q_1 q_2, r) = (q_1 q_2, r)$, luego $(q_1 q_2,e q_1)=(q_1 q_2,d q_2)$ y por tanto $q_1 (q_2,e)=q_2 (q_1,d)$ por la proposición anterior. Luego, dado que $r=(q_2,e)k_2 q_1=(q_1,d)k_1 q_2$, la ley de cancelación implica que $k_1 = k_2$.
\end{proof}

\begin{corollary}\label{cor:mcd1}
Si $r \in \mathbb{N}$, $e \mid r$ y $d \mid r$ con $e,r \in \mathbb{N}$, entonces
\begin{equation*}
    d/\left( r/e,d \right) = e/\left( r/d,e \right).
\end{equation*}
\end{corollary}

\begin{theorem}[Lema de Euclides]
Si $a \mid bc$ y $(a,b)=1$ entonces $a \mid c$.
\end{theorem}
\begin{proof}
Si $(a,b)=1$, podemos escribir $1=a s+b t$, donde $s,t\in \mathbb{Z}$. Luego $c=a(s c)+b c(t)$ y como $a \mid a$ y $a \mid bc$ por hipótesis, entonces $a \mid c$.
\end{proof}
