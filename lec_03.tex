%%% Una norma
\subsection{Una norma para funciones aritméticas}

\begin{definition}
Sea $\mathcal{A}$ el conjunto de todas las funciones aritméticas. Se define
\begin{align*}
    \mathcal{N} : \mathcal{A} & \longrightarrow \mathbb{N} \cup \{0\} \\
	   f & \longmapsto \mathcal{N}(f)=
	   \begin{cases}
           \hfil 0 & \textnormal{si } f=\mathbf{0} \\ \hfil
           \min \left\{n : f(n)\neq 0\right\} & \textnormal{si } f \neq \mathbf{0}.
       \end{cases}
\end{align*}
\end{definition}

\begin{proposition}
La función $\mathcal{N}$ definida anteriormente tiene las siguientes propiedades:
\begin{enumerate}[label=\textnormal{(\roman*)}]
    \item $\mathcal{N}(f)=0 \iff f=0, \forall f \in \mathcal{A}$.
    \item $\mathcal{N}(f*g)=\mathcal{N}(f)\mathcal{N}(g), \forall f,g \in \mathcal{A}$.
    \item $\min \{\mathcal{N}(f),\mathcal{N}(g)\}\leq \mathcal{N}(f+g), \forall f,g \in \mathcal{A}$. 
    \item Si $\mathcal{N}(f) \ne \mathcal{N}(g)$, entonces $\mathcal{N}(f+g)=\min \{ \mathcal{N}(f),\mathcal{N}(g) \}$.
\end{enumerate}
\end{proposition}
\begin{proof}
({\scshape \romannumeral 1}) Si $f=0$ por definición se tiene que $\mathcal{N}(f)=0$. Si $f \neq 0$, entonces $\min \left\{n : f(n) \neq 0\right\}\neq 0$, i.e. $\mathcal{N}(f)\geq 1\neq 0$. Por tanto $\mathcal{N}(f)=0$, implica que $f=0$.
\bigskip

({\scshape \romannumeral 2}) Si $f=0$ o $g=0$, entonces $\mathcal{N}(f)=0$ o $\mathcal{N}(g)=0$. Además $(f*g)(n)=\sum_{d \mid n} f(n)g(n/d)=0, \:\forall\: n \in\mathbb{N}$, es decir, $\mathcal{N}(f*g)=0=\mathcal{N}(f)\mathcal{N}(g)$. Supongamos pues que $f \neq 0$ y $g \neq 0$. Sean $a=\mathcal{N}(f)$ y $b=\mathcal{N}(g)$. Afirmamos que $a b=\min \left\{n : (f*g)(n) \ne 0\right\}=m$.
\bigskip

En efecto, se tiene
\begin{align*}
	(f*g)(a b) &= \sum_{d \mid a b} f(d)g(a b/d) \\
			   &= \sum_{\substack{d \mid a b \\ a \leq d}} f(d)g(a b/d), \textnormal{ pues }f(d)=0, \:\forall\: d<a \\
			   &= \sum_{\substack{d \mid a b \\ a \leq d \\ a b/d \leq b}} f(d) g(a b/d), \textnormal{ pues }a \leq d \implies a b/d \leq b \\
			   &= \sum_{a=d} f(d)g(a b/d), \textnormal{ pues }g(d)=0, \:\forall\: d<b \\
			   &= f(a)g(b)\neq 0.
\end{align*}
Luego $m \leq a b$ por elección de $m$. Si $m < a b$, entonces
\begin{equation*}
	(f * g)(m) = \sum_{d \mid m} f(d) g(m/d) = \sum_{\substack{d \mid m \\ b \le m/d}} f(d) g(m/d) = \sum_{\substack{d \mid m \\ d<a}} f(d) g(m/d) = 0,
\end{equation*}
pues $b \le m/d$ implica que $d<a$ y $f(d)=0$. Pero esto contradice la elección de $m$. Por tanto, $m=a b$.

\bigskip

({\scshape \romannumeral 3}) Sin pérdida de generalidad se puede suponer que $a \le b$, de tal manera que $\min \{ a,b \} = a$. Si $n<a$, entonces $(f+g)(n) = f(n) + g(n) = 0$, por lo que
\begin{equation*}
    \min \{ \mathcal{N}(f),\mathcal{N}(g) \}=a \le \min \{ n \std (f+g)(n) \ne 0 \}=\mathcal{N}(f+g).
\end{equation*}

({\scshape \romannumeral 4}) Supóngase de nuevo sin pérdida de generalidad que $a<b$. Entonces
\begin{equation*}
	(f+g)(a) = f(a) + g(a) = f(a) + 0 = f(a) \ne 0,
\end{equation*}
por tanto, $N(f+g) \le a = \min \{ a,b \}=\min \{ \mathcal{N}(f),\mathcal{N}(g) \}$. El resultado se sigue ahora del punto ({\scshape \romannumeral 3}).

\end{proof}
\begin{theorem}
$\mathcal{A}$ es un dominio entero. 
\end{theorem}
\begin{proof}
Si $f,g \in \mathcal{A}$ y $f*g=0$, entonces $\mathcal{N}(f*g)=0 \implies \mathcal{N}(f)\mathcal{N}(g)=0 \implies \mathcal{N}(f)=0$ o $\mathcal{N}(g)=0 \implies f=0$ o $g=0$.
\end{proof}

\begin{remark}
Como ocurre en cualquier anillo con identidad, el conjunto de elementos invertibles forma un grupo respecto a la operación de multiplicación, en este caso, respecto a la convolución de Dirichlet. Este grupo se denotará $(\mathcal{A}^*,*)$ o simplemente como $\mathcal{A}^*$ cuando no haya riesgo de confusión.
\end{remark}

\begin{proposition}\label{prop:str1}
$f \in \mathcal{A}^*$ si y sólo si $\mathcal{N}(f)=1$.
\end{proposition}
\begin{proof}
Si $f(1) \ne 0$, defínase
\begin{equation}\label{eq:inv1}
    \begin{split}
	f^{-1}(1) &= \frac{1}{f(1)} \\
	f^{-1}(n) &= -\frac{1}{f(1)} \sum_{\substack{d \mid n \\ d \ne n}} f \left( \frac{n}{d} \right) f^{-1}(d), \hspace{0.2cm} n>1.
\end{split}
\end{equation}

Entonces la ecuación recursiva \eqref{eq:inv1} define a $f^{-1}$ de tal forma que $f*f^{-1} = I$, pues $f(1)f^{-1}(1)=1$ y si $n>1$, entonces
\begin{align*}
	(f*f^{-1})(n) = \sum_{d \mid n} f(d)f^{-1}\left( \frac{n}{d} \right) &= f(1)f^{-1}(n) + \sum_{\substack{d \mid n \\ d \ne n }} f(d)f^{-1}\left( \frac{n}{d} \right) \\
	&= f(1)f^{-1}(n)-f(1)f^{-1}(n) = 0
\end{align*}
es decir, $f*f^{-1}=I$.
\bigskip

Si se supone ahora que $f$ es invertible, entonces, en particular, se tiene que $(f*f^{-1})(1)=1$,  y por tanto $f(1) \ne 0$, es decir, $\mathcal{N}(f)=1$.
\end{proof}

\begin{proposition}
Si $\mathcal{N}(f)=p$ para algún número primo $p$, entonces $f$ es irreducible en $\mathcal{A}$.
\end{proposition}
\begin{proof}
Como $p \ne 0$ y $p \ne 1$, entonces $f$ no es cero ni es una unidad. Además, si $f=g*h$ para algunas funciones $g,h \in \mathcal{A}$, entonces  $g,h \ne 0$,  pues en caso contrario $f=0$ y en consecuencia $\mathcal{N}(f)=0 \ne p$, así que $\mathcal{N}(g)$ y $\mathcal{N}(h)$ son enteros positivos. Luego $p = \mathcal{N}(f) = \mathcal{N}(g*h) = \mathcal{N}(g)\mathcal{N}(h)$ y como $p$ es primo, entonces $\mathcal{N}(g)=1$ o bien $\mathcal{N}(h)=1$, es decir, $g$ o $h$ es unidad. Como $g$ y $h$ fueron arbitrarios, entonces $f$ debe ser irreducible en $\mathcal{A}$.
\end{proof}

\begin{theorem}\label{thm:chain1}
No puede existir una sucesión $\{ f_i \}_{i \in \mathbb{N}}$ de funciones aritméticas con la propiedad de que $f_1 \ne 0$ y $f_i=f_{i+1}*g_{i+1}$ y $g_{i+1}$ no es unidad, para cada $i \in \mathbb{N}$.
\end{theorem}
\begin{proof}
Asuma que dicho conjunto existe. Nótese que $\mathcal{N}(f_1) = \mathcal{N}(f_2) \mathcal{N}(g_2) > \mathcal{N}(f_2)$, pues $g_2$ no es unidad. De hecho se tiene
\begin{equation*}
    \mathcal{N}(f_1) > \mathcal{N}(f_2) > \cdots > \mathcal{N}(f_i), \forall i \in \mathbb{N}.
\end{equation*}
Más aún, como cada uno de éstos números son enteros, entonces
\begin{equation*}
    \mathcal{N}(f_1) \ge \mathcal{N}(f_2) + 1 \ge \cdots \ge \mathcal{N}(f_{i+1}) + i, \forall i \in \mathbb{N}
\end{equation*}
En particular, como $\mathcal{N}(f_1) \ge 1$ por ser $f_1 \ne \mathbf{0}$, se tiene que $\mathcal{N}(f_1) \ge \mathcal{N}\left(f_{\mathcal{N}(f_1)+1}\right) + \mathcal{N}(f_1)$, luego $0 \ge \mathcal{N}\left(f_{\mathcal{N}(f_1)+1}\right)$ y por tanto $\mathcal{N}(f_{\mathcal{N}(f_1)+1}) = \mathbf{0}$, en consecuencia $f_{\mathcal{N}(f_1)+1} = \mathbf{0}$, lo cual no es posible, pues de ser así $f_1 = f_{\mathcal{N}(f_1)+1} \cdot g_{\mathcal{N}(f_1)+1} \cdots g_2 = \mathbf{0}$, contradiciendo la hipótesis.
\end{proof}

El teorema anterior permite probar que cualquier elemento no cero y no unidad de $\mathcal{A}$ se puede expresar como producto finito de elementos irreducibles de $A$.

\begin{proposition}
Si $f \in \mathcal{A}\setminus (\mathcal{A}^* \cup \{ \mathbf{0} \})$, entonces $f$ es producto finito de elementos irreducibles en $\mathcal{A}$.
\end{proposition}
\begin{proof}
Como $f \ne \mathbf{0}$, en lo que sigue de esta demostración se debe tener que todas las funciones involucradas son distintas de cero. Se probará primero que $f$ tiene un factor irreducible. En efecto, si $f$ es irreducible, entonces no hay nada que probar. Supóngase que este no es el caso y por tanto $f=f_1*g_1$, donde $f_1$ y $g_1$ no son unidades. Si $f_1$ es irreducible hemos concluido. En caso contrario, se tiene $f_1=f_2*g_2$, donde $f_2$ y $g_2$ no son unidades. De manera inductiva se tiene una sucesión de funciones $\{ f_i \}_{i \in \mathbb{N}}$ tal que $f_i=f_{i+1}*g_{i+1}$, donde $g_{i+1}$ no es unidad para cada $i \in \mathbb{N}$, lo cual es imposible por el teorema anterior. En consecuencia, este proceso debe terminar y debe existir $M \in \mathbb{N}$ tal que $f_M=f_{M+1}*g_{M+1}$, donde $f_{M+1}$ es irreducible y $f_{M+1} \mid f$.
\bigskip

Se probará ahora el resultado principal. Escribiendo $f_{M+1}=p_1$, se tiene que $f=p_1*q_1$, con $p_1$ irreducible. Si $q_1$ es una unidad, entonces $f$ es irreducible y ya terminamos. Si $q_1$ no es unidad, nuevamente, $q_1$ debe tener un factor irreducible, es decir, $q_1=p_2*q_2$, donde $p_2$ es irreducible, y por tanto no es unidad. Si $q_2$ es unidad, entonces $q_1$ es irreducible y $f=p_1*q_1$ es la factorización buscada. Si este proceso nunca terminara, de forma inductiva se tendría una sucesión $\{ q_i \}_{i \in \mathbb{N}}$ tal que $q_i=p_{i+1}*q_{i+1}$, con $p_i$ no unidad, para cada $i \in \mathbb{N}$, lo que contradice de nuevo el Teorema \ref{thm:chain1}. En consecuencia, el proceso eventualmente termina y por tanto existe $N \in \mathbb{N}$ tal que $q_N=p_{N+1}*q_{N+1}$, donde $q_{N+1}$ es unidad y $p_{N+1}$ es irreducible. Luego
\begin{equation*}
    f=p_1*p_2*\cdots*p_N*q_N,
\end{equation*}
donde $p_1,\ldots,p_{N}$ y $q_N$ son irreducibles.
\end{proof}

Habiendo llegado tan lejos, se puede conjeturar que el dominio $\mathcal{A}$ es un dominio de factorización única. Esta sospecha es, de manera sorprendente, acertada. Sin embargo, la demostración de este hecho no es tan sencilla como la de la proposición anterior.

\begin{theorem}
$\mathcal{A}$ es un dominio de factorización única.
\end{theorem}
\begin{proof}
El hecho de que toda función aritmética se puede escribir como producto de funciones aritméticas irreducibles ha quedado en evidencia en la proposición anterior. Una demostración de la unicidad de dicha factorización se puede encontrar en \cite[18, p. 985]{Ca1959}. Ahí se prueba que el anillo de series de potencias formales en un conjunto numerable de variables $\{ x_1,x_2,\ldots \}$ es un dominio de factorización única. El resultado se sigue del hecho de que este anillo es isomorfo al anillo de funciones aritméticas mediante el isomorfismo
\begin{align*}
    P : \mathcal{A} & \longrightarrow \mathbb{C}[[x_1,x_2,\ldots]] \\
    P(f) & \longmapsto \sum_{n \in \mathbb{N}} f(n) x_1^{\alpha_1}\cdots x_{\nu}^{\alpha_{\nu}},
\end{align*}
donde $n=p_1^{\alpha_1}\cdots p_{\nu}^{\alpha_{\nu}}$ es la factorización en primos de $n$. Se tiene que $P(f+g)=P(f)+P(g)$ y $P(f*g)=P(f)P(g)$, donde la multiplicación de dos series de este tipo se realiza agrupando términos ``semejantes'', es decir, monomios iguales. Otra demostración de este hecho se puede encontrar en \cite{Nish1}. Ambas demostraciones utilizan el hecho de que los anillos de series de potencias formales en un número finito de variables $\mathbb{C}[[x_1,\ldots,x_n]]$ son dominios de factorización única, para cada $n \in \mathbb{N}$.
\end{proof}

\begin{corollary}\label{cor:fac1}
Todo elemento irreducible en $\mathcal{A}$ es primo en $\mathcal{A}$.
\end{corollary}

Siendo $\mathcal{A}$ un dominio de factorización única, cabe preguntarse si también es un dominio de ideales principales. La siguiente proposición muestra que este no es el caso.

\begin{proposition}
$\mathcal{A}$ no es un dominio de ideales principales.
\end{proposition}
\begin{proof}
Considere $f=(0,1,0,\ldots)$ y $g=(0,0,1,0,\ldots)$. Se tiene que $\mathcal{N}(f)=2$ y $\mathcal{N}(g)=3$, ambos números primos. Afirmamos que $I$ es un máximo común divisor de $f$ y $g$. Claro que $I \mid f$ y $I \mid g$. Si $h \in \mathcal{A}$ es tal que $h \mid f$ y $h \mid g$, entonces $f=h k_1$ y $g=h k_2$, con $h,k_1,k_1 \in \mathcal{A}\setminus \{ \mathbf{0} \}$. Luego $2=\mathcal{N}(h)\mathcal{N}(k_1)<3=\mathcal{N}(h)\mathcal{N}(k_2)$, en consecuencia, $1 \le \mathcal{N}(k_1)<\mathcal{N}(k_2)$, así que necesariamente $\mathcal{N}(k_2)=3$ y $\mathcal{N}(h)=1$. Luego $h$ es unidad, es decir $h \mid I$. Esto prueba que $I$ es máximo común divisor de $f$ y $g$.
\bigskip

Si $\mathcal{A}$ fuera un dominio de ideales principales por \cite[\S III.3, Thm. 3.11.(ii), p. 140]{Hun1}, existirían $s,t \in \mathcal{A}$ tales que $I=f*s+g*t$, en particular, $1=I(1)=f(1)s(1)+g(1)t(1)=0$, lo cual es imposible.
\end{proof}

\begin{theorem}
$\mathcal{A}$ es un anillo local.
\end{theorem}
\begin{proof}
Por \cite[\S III.4, Thm. 4.13.(iii), p. 147]{Hun1}, basta probar que los elementos no invertibles de $\mathcal{A}$ forman un ideal de $\mathcal{A}$. En efecto, se tiene que $\mathbf{0} \in \mathcal{A}\setminus \mathcal{A}^*$. Si $f \in \mathcal{A}\setminus \mathcal{A}^*$ y $g \in \mathcal{A}$, entonces $f(1)=0$, en consecuencia $(f*g)(1)=f(1)g(1)=0$, es decir, $f*g \in \mathcal{A}\setminus \mathcal{A}^*$. Además, si $h \in \mathcal{A}\setminus \mathcal{A}^*$, entonces $h(1)=0$ y por tanto $f(1)-h(1)=0$, es decir $f-h \in \mathcal{A}\setminus \mathcal{A}^*$. Esto prueba que $\mathcal{A}\setminus \mathcal{A}^*$ es un ideal de $\mathcal{A}$.
\end{proof}

\begin{proposition}
Si $f \in \mathcal{A}$ es tal que $f(1) = 0$ y $f(p) \ne 0$ para alǵun primo $p$, entonces $f$ es irreducible en $\mathcal{A}$.
\end{proposition}
\begin{proof}
Por hipótesis $f$ no es cero y no es unidad. Supongamos que $f=g*h$. Si $g$ y $h$ no fueran unidades se tendría que $g(1)=0$ y $h(1)=0$, por tanto, $f(p)=(g*h)(p)=g(1)h(p)+g(p)h(1)=0$, lo que contradice la hipótesis. En consecuencia alguna de las funciones $g$ o $h$ es unidad.
\end{proof}

Un colorario de la proposición anterior y el \Cref{cor:fac1} es el siguiente.

\begin{corollary}
Si $f \in \mathcal{A}$ es tal que $f(1) = 0$ y $f(p) \ne 0$ para algún primo $p$, entonces $f$ es un elemento primo de $\mathcal{A}$.
\end{corollary}

El \Cref{thm:chain1} afirma que el anillo $\mathcal{A}$ satisface una forma débil de condición de la cadena ascendente, ¿satisfará también la condición de la cadena descendente? Es decir, ¿es $\mathcal{A}$ artiniano? El siguiente contraejemplo desacredita esta observación.

\begin{proposition}
$\mathcal{A}$ no es un anillo artiniano.
\end{proposition}
\begin{proof}
Para cada $n \in \mathbb{N}$ defínase $I_n = \{ f \in \mathcal{A} \std \mathcal{N}(f) \ge n \} \cup \{ \mathbf{0} \}$. Se tiene lo siguiente:\begin{enumerate}[label=\textnormal{(\arabic*)}]
\item $I_n$ es un ideal de $\mathcal{A}$, para cada $n$. En efecto, por definición se tiene $I_n \ne \emptyset$. Si $f,g \in I_n$, entonces $\mathcal{N}(f) \ge n$ y $\mathcal{N}(g) \ge n$, luego $\mathcal{N}(f-g) \ge \min \{ \mathcal{N}(f), \mathcal{N}(-g) \} = \min \{ \mathcal{N}(f),\mathcal{N}(g) \} \ge n$, luego $f-g \in I_n$.
\bigskip

Además, si $h \in \mathcal{A}$, se tienen dos casos. Si $h=\mathbf{0}$, entonces $f*h=0 \in I_n$. Si $h \ne \mathbf{0}$, entonces $\mathcal{N}(f) \ge 1$, de tal manera que $\mathcal{N}(f*g)=\mathcal{N}(f)\mathcal{N}(h) \ge \mathcal{N}(f) \ge n$, es decir, $f*h \in I_n$. Esto prueba que $I_n$ es un ideal de $\mathcal{A}$.
\item $I_{n+1} \subset I_n$, para cada $n \in \mathbb{N}$, pues $\mathcal{N}(f) \ge n+1$, implica que $\mathcal{N}(f) \ge n$.
\item $I_n \not \subset I_{n+1}$, para cada $n \in \mathbb{N}$, pues considere $f \in \mathcal{A}$ definida como
\begin{equation*}
    f(k) = \begin{cases}
        \hfil 1 & \text{si } k=n \\
        \hfil 0 & \text{en otro caso.}
    \end{cases}
\end{equation*}
Entonces $\mathcal{N}(f) = n < n+1$, es decir, $f \in I_n$, pero $f \notin I_{n+1}$.
\end{enumerate}
Se tiene pues una cadena descendente de ideales distintos entre sí de $\mathcal{A}$, luego $\mathcal{A}$ no es artiniano.
\end{proof}

\subsection{Funciones multiplicativas}

\begin{definition}[Función multiplicativa]\label{def:mul1}
Se dice que una función aritmética $f$ es \textbf{multiplicativa} si no es idénticamente cero y para todo $m,n \in \mathbb{N}$, $(m,n)=1$, implica que $f(m n)=f(m)f(n)$.
\end{definition}

\begin{remark}
Se denota al conjunto de funciones multiplicativas como $\mathcal{M}$. En general si $f$ y $g$ son funciones multiplicativas, entonces $f-g$ no es necesariamente una función multiplicativa, sin embargo, $f*g$ sí lo es.
\end{remark}

\begin{lemma}
Si $(a,b)=1$ y $d \in \mathbb{N}$, entonces $(a b,d)=(a,d)(b,d)$.
\end{lemma}
\begin{proof}
Escríbanse $(a,d)=a x+d y$ y $(b,d)=b s+d t$, para algunos $x,y,s,t \in \mathbb{Z}$. Entonces
\begin{equation*}
	(a,d)(b,d) = a b x s + a x d t + d y b s + d y d t = a b (xs) + d(a x t + y b s + y d t),
\end{equation*}
por tanto, $(a b,d) \mid (a,d)(b,d)$. 
\bigskip

Por otro lado, escríbase $1=a z + b w$, para algunos $z,w \in \mathbb{Z}$. Entonces $d=d a z + d b w$. Además, como $a=(a,d)m$, $b=(b,d)n$, $d=(b,d)p$ y $d=(a,d)q$ para algunos $m,n,p,q \in \mathbb{Z}$, entonces
\begin{equation*}
    d = (a,d)(b,d) (p m z + q n w),
\end{equation*}
es decir, $(a,d)(b,d) \mid d$. Dado que $a b=(a,d)(b,d)m n$, entonces $(a,d)(b,d) \mid a b$ y en consecuencia $(a,d)(b,d) \mid (a b,d)$. Se sigue que $(a b,d)=(a,d)(b,d)$.
\end{proof}

\begin{lemma}\label{lem:div1}
Si $(a,b)=1$, $a_1,\ldots,a_l$ son todos los divisores positivos de $a$ y $b_1,\ldots,b_m$ son todos los divisores positivos de $b$, entonces $\{ d>0 \std d \mid a b \}=\{ a_i b_j \std i=1,\ldots,l, \hspace{0.2cm} j=1,\ldots,m \}$.
\end{lemma}
\begin{proof}
Si $a_i$, $b_j$ son divisores de $a$ y $b$, respectivamente, entonces existen $s,t \in \mathbb{Z}$ tales que $a=a_i s$ y $b=b_j t$, luego $a b=a_i b_j s t$, es decir, $a_i b_j \mid a b$. Recíprocamente, si $d$ es un divisor de $a b$, entonces $(a b,d)=d$, pero por el lema anterior $(a b,d)=(a,d)(b,d)$, luego $d=(a,d)(b,d)$, donde $(a,d)$ es un divisor positivo de $a$ y $(b,d)$ es un divisor positivo de $b$.
\end{proof}

\begin{theorem}\label{thm:sb1}
$(\mathcal{M},*)$ es un subgrupo de $(\mathcal{A}^*,*)$.
\end{theorem}
\begin{proof}
Si $f \in \mathcal{M}$, entonces $f \ne \mathbf{0}$ y existe $N \in \mathbb{N}$ tal que $f(N) \ne 0$, luego $f(N)=f(1 \cdot N)=f(1)f(N)$ y en consecuencia $1=f(1)$, es decir, $f \in \mathcal{A}^*$. Esto prueba que $\mathcal{M} \subset \mathcal{A}^*$.
\bigskip

Claro que el conjunto $\mathcal{M}$ es no vacío, pues $I \in \mathcal{M}$. Veamos que la operación $*$ es cerrada en $\mathcal{M}$. Sean $f,g$ funciones multiplicativas, sean $a,b \in \mathbb{N}$ tales que $(a,b)=1$ y sean $a_1,\ldots,a_l$ y $b_1,\ldots,b_m$ todos los divisores positivos de $a$ y $b$, respectivamente. Entonces $(a_i,b_j)=1$, para cada $i=1,\ldots,l$ y para cada $j=1,\ldots,m$, luego
\begin{align*}
	(f*g)(a)(f*g)(b) &= \left[ \sum_{i=1}^{l} f(a_i)g\left( \frac{a}{a_i} \right) \right]\left[ \sum_{j=1}^{m} f(b_j)g \left( \frac{b}{b_i} \right) \right] \\
					 &= \sum_{i=1}^{l} \sum_{j=1}^{m} f(a_i)g \left( \frac{a}{a_i} \right)f(b_i)g \left( \frac{b}{b_i} \right) \\
					 &= \sum_{i=1}^{l} \sum_{j=1}^{m} f(a_i b_j)g \left( \frac{a b}{a_i b_j} \right) \\
					 &= \sum_{d \mid a b}f(d)g \left( \frac{a b}{d} \right) = (f*g)(a b)
\end{align*}
por el \Cref{lem:div1}.
\bigskip

\thispagestyle{easter4}

Como ya se probó al inicio de esta demostración, si $f$ es multiplicativa, entonces $f(1)=1$, por lo que existe $f^{-1}$. Veamos que $f^{-1} \in \mathcal{M}$. Para esto construiremos, a partir de $f$, una función multiplicativa $g$ con la propiedad de que $f*g=I$, con lo que quedará demostrado que $f^{-1}$ es multiplicativa por la unicidad de la inversa. Se procede definiendo a $g$ de forma gradual:

\begin{enumerate}[label=\textnormal{(\arabic*)},ref=\textnormal{\arabic*}]
\item $g(1)=1$.
\item Para cada primo $p$ se define $g(p)=-f(p)$. De tal manera que
	\begin{equation*}
		(f*g)(p) = \sum_{d \mid p} f(p)g \left( \frac{p}{d} \right) = f(1)g(p) + f(p)g(1) = -f(p) + f(p) = 0.
	\end{equation*}
\item \label{it:mul1} Para cada $a \in \mathbb{N}$ y para cada primo $p$ se define, recursivamente,
	\begin{equation*}
	    g(p^a)=-f(p)g(p^{a-1})-\cdots-f(p^a)g(1)
	\end{equation*}
	de tal manera que 
	\begin{align*}
		(f&*g)(p^a) = \sum_{d \mid p^a} f(d)g \left( \frac{p^a}{d} \right) = f(1)g(p^a)+f(p)g(p^{a-1})+\cdots+f(p^a)g(1) \\
				   &= -f(p)g(p^{a-1})-\cdots-f(p^a)g(1)+f(p)g(p^{a-1})+\cdots+f(p^a)g(1) = 0.
	\end{align*}
\item Se define
	\begin{equation*}
	    g \left( \prod p_i^{a_i} \right) = \prod g(p_i^{a_i}).
	\end{equation*}
	para cualquier producto finito de potencias de primos, con $p_i \ne p_j$ si $i \ne j$. La función $g$ ha quedado entonces definida para cualquier entero positivo.
\item $g$ es multiplicativa, pues si $a=p_1^{\alpha_1}\cdots p_m^{\alpha_m}$ y $b=q_1^{\beta_1}\cdots q_l^{\beta_l}$ son tales que $(a,b)=1$, entonces $p_i \ne q_j$, luego
	\begin{align*}
		g(a b) = g(p_1^{\alpha_1}\cdots p_m^{\alpha_m} q_1^{\beta_1}\cdots q_l^{\beta_l}) &= g(p_1^{\alpha_1})\cdots g(p_m^{\alpha_m})g(q_1^{\beta_1})\cdots g(q_l^{\beta_l}) \\
																				 &= g(p_1^{\alpha_1}\cdots p_m^{\alpha_m})g(q_1^{\beta_1}\cdots q_l^{\beta_l}) = g(a)g(b)
	\end{align*}
\item Como la operación $*$ es cerrada en $\mathcal{M}$, entonces $f*g$ es multiplicativa.
\item Si $n>1$ y $n=p_1^{\alpha_1}\cdots p_l^{\alpha_l}$ es su factorización en primos, entonces
	\begin{equation*}
		(f*g)(n) = (f*g)(p_1^{\alpha_1})\cdots (f*g)(p_l^{\alpha_l}) = 0
	\end{equation*}
	donde la primera igualdad se cumple por ser $f*g$ multiplicativa y la segunda por el inciso \eqref{it:mul1}. Además, $(f*g)(1)=f(1)g(1)=1$. En consecuencia $f*g=I$.
\item Se sigue que $g=f^{-1}$ y como $g$ es multiplicativa, entonces $f^{-1}$ también lo es.
\end{enumerate}
\end{proof}

\begin{corollary}
Si $f*g$ es multiplicativa y $g$ es multiplicativa, entonces $f$ también lo es.
\end{corollary}
\begin{proof}
Como $g$ es multiplicativa, entonces existe $g^{-1}$ y también es multiplicativa, luego $f=(f*g)*g^{-1}$ es multiplicativa por ser producto de funciones multiplicativas.
\end{proof}

\subsection{Isomorfismos entre grupos de funciones aritméticas}

Se denotará como $\mathcal{A}_\mathbb{R}$ al conjunto de funciones aritméticas real valuadas, es decir, $\mathcal{A}_{\mathbb{R}}=\{ f \in \mathcal{A} \std f(n) \in \mathbb{R}, \forall n \in \mathbb{N} \}$. Asimismo, se define $P=\{ f \in \mathcal{A} \std f(1)>0 \}$. Es fácil verificar que $(\mathcal{A}_{\mathbb{R}},+)$ y $(P,*)$ son subgrupos de $(\mathcal{A},+)$ y de $(\mathcal{A}^*,*)$, respectivamente. Más aún, estos grupos son isomorfos.

\begin{lemma}\label{prop:iso1}
$(\mathcal{A}_{\mathbb{R}},+) \cong (P,*)$.
\end{lemma}
\begin{proof}
El isomorfismo buscado es
\begin{align*}
    L : (P,*) & \longrightarrow (\mathcal{A}_{\mathbb{R}},+) \\
    f & \longmapsto L f
\end{align*}
donde $L f(1)=\log(f(1))$ y $L f(n)=\sum_{d \mid n} \log(d)f(d)f^{-1}(n/d)$ para $n>1$. Se tiene que $L$ es en efecto un homomorfismo, pues para $n=1$ se tiene
\begin{equation*}
    L (f*g)(1) = \log (f*g) (1) = \log(f(1)g(1)) = \log f(1) + \log g(1) = L f(1) + L g(1).
\end{equation*}
Para el caso $n>1$, nótese primero que para cualquier $n \in \mathbb{N}$,
\begin{align*}
	\log(n)(f*g)(n) &= \log(n) \sum_{d \mid n} f(d)g \left( \frac{n}{d} \right) \\
					&= \sum_{d \mid n} f(d) g \left( \frac{n}{d} \right) \left[ \log \frac{n}{d} + \log d \right] \\
					&= \sum_{d \mid n} f(d) g \left( \frac{n}{d} \right) \log \left( \frac{n}{d} \right) + \sum_{d \mid n} f(d) g \left( \frac{n}{d} \right) \log(d) \\
					&= (f*(\log \cdot g))(n)+((\log \cdot f)*g)(n),
\end{align*}
es decir, $\log \cdot (f*g) = f*(\log \cdot g)+(\log \cdot f)*g$. Multiplicando por $(f*g)^{-1}=f^{-1}*g^{-1}$ a ambos lados de la ecuación, se tiene que
\begin{equation*}
	(\log \cdot (f*g))*(f*g)^{-1} = (\log \cdot g)*g^{-1} + (\log \cdot f)*f^{-1},
\end{equation*}
es decir, $L(f*g)=L f+L g$ y en particular para $n>1$. Esto prueba que $L$ es un homomorfismo.
\bigskip

$L$ también es suprayectivo, pues si $f \in \mathcal{A}_{\mathbb{R}}$, defínase $g(1)=\exp(f(1))$. Entonces $L g(1)=\log g(1)=\log \exp(f(1))=f(1)$, pues $f(1) \in \mathbb{R}$. Además, como $g(1)>0$ existe $g^{-1}$ y se define recursivamente, para $n>1$,
\begin{equation*}
    g(n) = \frac{1}{\log(n) g^{-1}(1)} \left[ f(n) - \sum_{\substack{d \mid n \\ d \ne 1,n}} \log(d) g(d) g^{-1} \left( \frac{n}{d} \right)\right].
\end{equation*}
Esta ecuación implica que
\begin{align*}
	f(n) = g(n)\log(n)g^{-1}(1)+g(1)\log(1)g^{-1}(n)&+\sum_{\substack{d \mid n \\ d \ne 1,n}} \log(d)g(d)g^{-1} \left( \frac{n}{d} \right) \\
		 &= \sum_{d \mid n} \log(d)g(d)g^{-1} \left( \frac{n}{d} \right) = L g(n).
\end{align*}
En consecuencia, $L g(n)=f(n), \forall n \in \mathbb{N}$, es decir, $L g=f$.
\bigskip

Finalmente, se tiene que $L$ es inyectivo. En efecto, si $L(f)=L(g)$, entonces $L(f)-L(g)=0$, pero $-L g=L g^{-1}$ por ser $L$ un homomorfismo, luego $L f+L g^{-1}=L(f*g^{-1})=0$. Para $n=1$ esto implica que $\log(f*g^{-1}(1))=0$ y por tanto $(f*g^{-1})(1)=1$. Si $n=2$, entonces
\begin{equation*}
    L(f*g^{-1})(2) = \log(1)(f*g^{-1})(1)(f*g^{-1})^{-1}(2) + \log(2)(f*g^{-1})(2)(f*g^{-1})(1) = 0,
\end{equation*}
pero $\log(1)=0$, por tanto $\log(2)(f*g^{-1})(2)(f*g^{-1})(1)=0$ y dado que $(f*g^{-1})(1) \ne 0$, entonces $(f*g^{-1})(2)=0$. Supóngase que $(f*g^{-1})(d)=0$, para cada $1<d<n$. Entonces $L(f*g)=0$ implica que
\begin{equation*}
	\log(n)(f*g^{-1})(n)(f*g^{-1})(1) + \sum_{\substack{d \mid n \\ d \ne 1,n}} \log(d)\underbrace{(f*g^{-1})(d)}_{0}(f*g^{-1})^{-1} \left( \frac{n}{d} \right) = 0,
\end{equation*}
pues $\log(1)=0$, por tanto, $\log(n)(f*g^{-1})(n)(f*g^{-1})(1)=0$ y por tanto $(f*g^{-1})(n)=0$. Esto prueba que para cada $n>1$, $(f*g^{-1})=0$. Así pues, se tiene que $f*g^{-1}=I$, por tanto, $f=g$.
\end{proof}

Se denota $\mathcal{A}^\prime=\{ f \in \mathcal{A}_{\mathbb{R}} \std f(n) = 0,\forall n \ne p^\alpha, p \text{ primo y } \alpha \in \mathbb{N} \}$. La siguiente proposición es una caracterización de las funciones multiplicativas respecto al conjunto $\mathcal{A}^\prime$ y al isomorfismo $L$.

\begin{proposition}
$f \in \mathcal{M}$ si y sólo si $L f \in \mathcal{A}^\prime$.
\end{proposition}
\begin{proof}
Supóngase primero que $f$ es multiplicativa. Entonces $f(1)=1$, por tanto, $L f(1)=\log f(1)=\log 1 = 0$. Si $N>1$ no es potencia de ningún primo, entonces $N=m n$, con $(m,n)=1$ y $n,m>1$. Luego
\begin{align*}
	L f(N) = L f(m n) &= \sum_{d \mid m n} \log(d)f(d)f^{-1} \left( \frac{m n}{d} \right) \\
					  &= \sum_{d \mid m}\sum_{e \mid n} f(d)f(e)f^{-1}\left( \frac{m}{d} \right)f^{-1}\left( \frac{n}{e} \right)(\log(d)+\log(e)) \\
					  &= \sum_{d \mid m} \log(d)f(d)f^{-1}\left( \frac{m}{d} \right)\sum_{e \mid n} f(e)f^{-1}\left( \frac{n}{e} \right) \\
					  &+ \sum_{e \mid n} \log(e)f(e)f^{-1}\left( \frac{n}{e} \right)\sum_{d \mid m}f(d)f^{-1}\left( \frac{m}{e} \right) \\
					  &= L f(m)\sum_{e \mid n} f(e)f^{-1}\left( \frac{n}{e} \right)+L f(n) \sum_{d \mid m}f(d)f^{-1}\left( \frac{m}{d} \right) \\
					  &= L f(m)I(n) + L f(n)I(m) = 0,
\end{align*}
pues $m,n>1$. Luego $f \in \mathcal{A}^\prime$. 
\bigskip

Recíprocamente, supóngase que $Lf \in \mathcal{A}^\prime$. En particular se tilene que $Lf(1)=0$ y por tanto $f(1)=1$. Se definirá una función multiplicativa $g$ y se probará que coincide con $f$.
\begin{enumerate}[label=\textnormal{(\arabic*)},ref=\textnormal{\arabic*}]
    \item Se define $g(1)=1=f(1)$.
    \item Para cada primo, se define
        \begin{equation*}
            g(n) = \prod_{p \mid n} f(p^\nu),
        \end{equation*}
        donde $\nu := \max \{ \alpha \std p^\alpha \mid n \}$.
    \item $g$ es multiplicativa, pues $(m,n)=1$ implica que
        \begin{equation*}
            g(m n) = \prod_{p \mid m n} f(p^\nu) = \prod_{p \mid n} f(p^\nu) \prod_{p \mid n} f(p^\nu) = g(m) g(n).
        \end{equation*}
    \item \label{it:mul2} $g$ coincide con $f$ en todas las potencias de primos, pues si $q$ es un primo y $\alpha \in \mathbb{N}$,
        \begin{equation*}
            g(q^{\alpha}) = \prod_{p \mid q^{\alpha}} f(p^\nu) = f(q^{\alpha}).
        \end{equation*}
    \item $g^{-1}$ coincide con $f^{-1}$ en todas las potencias de primos, pues si $q$ es primo,
        \begin{equation*}
            g^{-1}(q) = - \sum_{\substack{d \mid q \\ d \ne q}} g \left( \frac{q}{d} \right) g^{-1}(d) = -g(q)g^{-1}(1) = -g(q) = -f(q) = f^{-1}(q),
        \end{equation*}
        por el \cref{it:mul2}. Además, de forma recursiva se tiene que
        \begin{align*}
            g^{-1}(q^\alpha) &= -[g(q^{\alpha-1})g^{-1}(1)+\cdots+g(q)g^{-1}(q^{\alpha-1})] \\
                             &= -[f(q^{\alpha-1})f^{-1}(1)+\cdots+f(q)f^{-1}(q^{\alpha-1})] = f^{-1}(q^\alpha),
        \end{align*}
        donde $g^{-1}$ coincide con $f^{-1}$ en $1,q,q^2,\ldots,q^{\alpha-1}$.
    \item El punto anterior implica que $Lf(q^\alpha)=Lg(q^\alpha)$ para todo primo $q$ y para todo $\alpha \in \mathbb{N}$, pues
        \begin{equation*}
            Lf(q^\alpha) = \sum_{d \mid p^\alpha} \log(d) f(d)f^{-1}\left( \frac{n}{d} \right) = \sum_{d \mid p^\alpha} \log(d) g(d)g^{-1}\left( \frac{n}{d} \right) = Lg(q^\alpha).
        \end{equation*}
        Además, como $g$ es multiplicativa, entonces $Lg(n)=0$ para todo $n$ no potencia de algún primo, por la primera parte de esta demostración. Luego, por hipótesis se tiene que $Lf(n)=0=Lg(n)$ para todo $n$ no potencia de algún primo, así que de hecho $Lf(n)=Lg(n)$ para todo $n$, es decir, $Lf=Lg$ y como la aplicación $L$ es inyectiva, entonces $f=g$. Luego $f$ es multiplicativa, pues $g$ lo es.
\end{enumerate}
\end{proof}

\begin{lemma}\label{prop:iso2}
$(\mathcal{M},*) \cong (\mathcal{A}^\prime,+)$.
\end{lemma}
\begin{proof}
Es fácil ver que $(\mathcal{A}^\prime,+)$ es un subgrupo de $(\mathcal{A}_{\mathbb{R}},+)$ y que $(\mathcal{M},*)$ es un subgrupo de $(P,*)$. Luego la restricción del homomorfismo a $L$ a $(\mathcal{M},*)$ sigue siendo un isomorfismo y su imagen es $(\mathcal{A}^\prime,+)$ por la proposición anterior.
\end{proof}

\begin{lemma}\label{prop:iso3}
$(\mathcal{A}_{\mathbb{R}},+) \cong (\mathcal{A}^\prime,+)$.
\end{lemma}
\begin{proof}
Sea
\begin{align*}
    \phi : (\mathcal{A}_{\mathbb{R}},+) & \longrightarrow (\mathcal{A}^\prime,+) \\
    f & \longmapsto F,
\end{align*}
donde $F$ es la función definida como $F(n)=f(p_n), \forall n \in \mathbb{N}$ y $p_n$ es el $n-$ésimo término en la sucesión de potencias de primos en orden ascendente.
\bigskip

Se tiene que $\phi$ es un homomorfismo, pues $\phi(f+g)(n)=(f+g)(p_n)=f(p_n)+g(p_n)=\phi(f)(n)+\phi(g)(n), \forall n \in \mathbb{N}$, luego $\phi(f+g)=F+G$. Se también tiene que $\phi$ es inyectivo, pues si $f,g \in (\mathcal{A}^\prime,+)$ son tales que $\phi(f)=\phi(g)$, entonces $f(p_n)=g(p_n), \forall n \in \mathbb{N}$, además $f(n)=g(n)=0$ si $n$ no es potencia de algún primo, de manera que $f(n)=g(n), \forall n \in \mathbb{N}$, es decir, $f=g$.
\bigskip

Finalmente se tiene que $\phi$ es suprayectivo, pues si $F \in (\mathcal{A}^\prime,+)$, defínase $f(p_n)=F(n), \forall n \in \mathbb{N}$ y $f(n)=0$ para toda $n$ no potencia de algún primo. Entonces $f \in (\mathcal{A}_{\mathbb{R}},+)$ y $\phi(f)(n)=f(p_n)=F(n), \forall n \in \mathbb{N}$, es decir, $\phi(f)=F$.
\end{proof}

\begin{lemma}\label{prop:iso4}
$(\mathcal{A}_{\mathbb{R}},+) \cong (\mathcal{A}_1,+)$, donde $\mathcal{A}_1 = \{ f \in \mathcal{A} \std f(1) \in \mathbb{R} \}$.
\end{lemma}
\begin{proof}
Es claro que $(\mathcal{A}_1,+)$ es un grupo aditivo. Defínase
\begin{align*}
    \psi : (\mathcal{A}_{\mathbb{R}},+) & \longrightarrow (\mathcal{A}_1,+) \\
    f & \longmapsto F
\end{align*}
donde $F$ es la función definida como $F(n)=f(2n-2)+i f(2n-1), \forall n>1$ y $F(1)=f(1)$. Se tiene que $\psi$ es un homomorfismo, pues $\psi(f+g)(1)=(f+g)(1)=f(1)+g(1)=\psi(f)(1)+\psi(g)(1)$, además,
\begin{align*}
    \psi(f+g)(n) &= (f+g)(2n-2)+i (f+g)(2n-1) \\
                 &= f(2n-2)+g(2n-2)+i f(2n-1)+i g(2n-1) \\
                 &= [f(2n-2)+i f(2n-1)] + [g(2n-2)+i g(2n-1)] \\
                 &= \psi(f)(n)+\psi(g)(n),
\end{align*}
luego $\psi(f+g)=\psi(f)+\psi(g)$.
\bigskip

El homomorfismo $\psi$ es también inyectivo, pues si $f,g \in (\mathcal{A}_{\mathbb{R}},+)$ son tales que $\psi(f)=\psi(g)$, entonces $f(n),g(n) \in \mathbb{R}, \forall n \in \mathbb{N}$ y además $f(2n-2)+i f(2n-1)=g(2n-2)+i g(2n-1)$, por tanto $f(2n-2)=g(2n-2)$ y $f(2n-1)=g(2n-1)$ y $f(1)=g(1)$, así que $f(n)=g(n), \forall n \in \mathbb{N}$, es decir $f=g$.
\bigskip

Finalmente, $\psi$ también es suprayectivo, pues dada $F \in (\mathcal{A}_1,+)$, se puede escribir $F=F_1+i F_2$, donde $F_1,F_2 \in \mathcal{A}_{\mathbb{R}}$. Defínase $g(1)=F(1)$ y 
{\everymath{\displaystyle}
    \begin{equation*}
        g(n) = \begin{dcases}
            \hfil F_1 \left( \frac{n}{2}+1 \right) & \text{si } n \text{ es par } \\
            \hfil F_2 \left( \frac{n+1}{2} \right) & \text{si } n \text{ es impar y } n>1.
        \end{dcases}
    \end{equation*}
}

Entonces $g \in \mathcal{A}_{\mathbb{R}}$, $\psi(g)(1)=g(1)=F(1)$ y $\psi(g)(n)=g(2n-2)+i g(2n-1)=F_1(n)+i F_2(n)=F(n)$ para cada $n>1$, es decir, $\psi(g)=F$.
\end{proof}

\begin{lemma}\label{prop:iso5}
$(\mathcal{A}_{\mathbb{R}},+) \cong (\mathcal{A},+)$.
\end{lemma}
\begin{proof}
Defínase
\begin{align*}
    \gamma : (\mathcal{A}_{\mathbb{R}},+) & \longrightarrow (\mathcal{A},+) \\
    f & \longmapsto F,
\end{align*}
donde $F$ es la función definida como $F(n)=f(2n-1)+if(2n), \forall n \in \mathbb{N}$. Se tiene que $\gamma$ es un homomorfismo, pues
\begin{align*}
    \gamma(f+g)(n) &= (f+g)(2n-1)+i (f+g)(2n) \\
                   &= f(2n-1)+g(2n-1)+i f(2n)+i g(2n) \\
                   &= [f(2n-1)+i f(2n)]+[g(2n-1)+i g(2n)] \\
                   &= \gamma(f)(n)+\gamma(g)(n),
\end{align*}
por tanto, $\gamma(f+g)=\gamma(f)+\gamma(g)$.
\bigskip

El homomorfismo $\gamma$ es también inyectivo, pues si $f,g \in \mathcal{A}_{\mathbb{R}}$ son tales que $\gamma(f)=\gamma(g)$, entonces $\gamma(f)(n)=\gamma(g)(n)$ para cada $n$, luego $f(2n-1)+i f(2n)=g(2n-1)+ i g(2n)$, por tanto $f(2n-1)=g(2n-1)$ y $f(2n)=g(2n)$ para cada $n$, en consecuencia $f(n)=g(n), \forall n \in \mathbb{N}$, es decir, $f=g$.
\bigskip

Finalmente, se tiene que $\gamma$ es suprayectivo, pues si $F \in \mathcal{A}$, se puede escribir $F=F_1+i F_2$, con $F_1, F_2 \in \mathcal{A}_{\mathbb{R}}$. Defínase
{\everymath{\displaystyle}
    \begin{equation*}
        f(n) = \begin{dcases}
            \hfil F_1 \left( \frac{n+1}{2} \right) & \text{si }n \text{ es impar } \\
            \hfil F_2 \left( \frac{n}{2} \right) & \text{si }n \text{ es par}.
        \end{dcases}
    \end{equation*}
}

Entonces, $\gamma(f)(n)=f(2n-1)+i f(2n)=F_1(n)+i F_2(n)=F(n)$, para cada $n$, es decir, $f \in \mathcal{A}_{\mathbb{R}}$ es tal que $\gamma(f)=F$.
\end{proof}

El resultado principal de esta sección es el siguiente, corolario de los lemas \ref{prop:iso1}, \ref{prop:iso2}, \ref{prop:iso3}, \ref{prop:iso4} y \ref{prop:iso5}.

\begin{theorem}
Los grupos $(\mathcal{A}_{\mathbb{R}},+)$, $(P,*)$, $(\mathcal{M},*)$, $(\mathcal{A}^\prime,+)$, $(\mathcal{A}_1,+)$ y $(\mathcal{A},+)$ son todos isomorfos. 
\end{theorem}
