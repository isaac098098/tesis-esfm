%%% Código

\section{Código}\label{sec:code}

El siguiente programa se utilizó para generar la \Cref{fig:plot6}. El archivo ejecutable se usa como \texttt{./div [int1] [int2]}, donde \texttt{int1} e \texttt{int2} son el intervalo para la gráfica. El programa genera una tabla en formato \texttt{.csv} que puede ser leída por cualquier software de graficación.

{
\footnotesize
\renewcommand\ttdefault{cmtt}
\begin{adjustwidth}{12mm+2mm}{2mm}
\begin{lstlisting}
#include <stdio.h>
#include <stdlib.h>
#include <math.h>

// maximo comun divisor

int mcd(int a, int b) {
    return b == 0 ? a > 0 ? a : -a : mcd(b, a % b);
}

// suma sobre divisores

int dsum(int n) {
    int sum=0;
    for(int i=1; i<=n; i++)
        if((n % i) == 0)
            sum+=i;
    return sum;
}

// funcion de mobius

int mob(int n) {
    int p=0;
    if (n == 1) {
        return 1;
    }
    if (n%2 == 0) {
        n=n/2;
        p++;
        if (n % 2 == 0)
            return 0;
    }
    for (int i=3; i <= n; i+=2) {
        if (n%i == 0) {
            n = n/i;
            p++;
            if (n % i == 0)
                return 0;
        }
    }
    return (p % 2 == 0) ? 1 : -1;
}

// suma de ramanujan modulo r

int ram(int r, int n) {
    int sum=0;
    int nr=mcd(n,r);
    for(int d=1; d<=nr; d++)
        if(nr % d == 0)
            sum+=mob(r/d)*d;
    return sum;
}

// programa principal

int main(int argc, char** argv) {
    FILE* div=fopen("div.csv", "w");

    for(int j = atoi(argv[1]); j <= atoi(argv[2]); j++)
        fprintf(div, "%d, %1.5f\n", j, 2.0*ram(1,j) - 2.0*ram(2,j) - 1.0*ram(3,j) + 1.0*ram(6,j));
    fclose(div);
    return 0;
}

\end{lstlisting}
\end{adjustwidth}
}
\bigskip


Por ejemplo, al ejecutar \texttt{./div -10 10} para la \Cref{fig:plot6}, el programa generó el siguiente archivo llamado \texttt{div.csv}.
\bigskip

{
\footnotesize
\renewcommand\ttdefault{cmtt}
\begin{adjustwidth}{12mm+2mm}{2mm}
\begin{lstlisting}[numbers=none]
-10, 0.00000
-9, 0.00000
-8, 0.00000
-7, 6.00000
-6, 0.00000
-5, 6.00000
-4, 0.00000
-3, 0.00000
-2, 0.00000
-1, 6.00000
0, 0.00000
1, 6.00000
2, 0.00000
3, 0.00000
4, 0.00000
5, 6.00000
6, 0.00000
7, 6.00000
8, 0.00000
9, 0.00000
10, 0.00000
\end{lstlisting}
\end{adjustwidth}
}
