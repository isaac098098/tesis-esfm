%%% Estructura del anillo
\newpage
\thispagestyle{empty}
\
\newpage
\pagenumbering{arabic}
\section{Estructura del anillo de funciones aritméticas}

\begin{definition}
Se entenderá como \textbf{función aritmética} a cualquier función $f : \mathbb{N} \longrightarrow \mathbb{C}$. Se denota al conjunto de todas las funciones aritméticas como $\mathcal{A}$.
\end{definition}

\begin{definition}[Función constante]
La función aritmética constante de valor $c \in \mathbb{C}$ en $\mathbb{N}$ será escrita en negritas como $\mathbf{c}$. Por ejemplo, $\mathbf{1}(n)=1, \forall n \in \mathbb{N}$.
\end{definition}

La siguiente función aritmética, conocida como función de Möbius, es de importancia central en la teoría de números. Aunque a primera vista su definición parece más bien artificial, se verá que aparece naturalmente al derivar propiedades del producto de Dirichlet.

\begin{definition}[Función de Möbius]
La función $\mu$ de Möbius está definida por $\mu(1)=1$ y dada $n=p_1^{\alpha_1}p_2^{\alpha_2}\cdots p_k^{\alpha_k}$ la factorización de $n$ en primos, entonces
\begin{equation*}
	\mu(n) =
		\begin{cases}
			(-1)^k & \text{si} \: \alpha_1=\alpha_2=\cdots=\alpha_k=1 \\ \hfil
			0 & \text{en otro caso.}
		\end{cases}
\end{equation*}
\end{definition}

Una primera forma natural de operar funciones aritméticas es haciéndo su suma o multiplicación puntual, obteniéndo otra función aritmética.

\begin{definition}
Si $f,g \in \mathcal{A}$, definimos la \textbf{suma} de $f$ y $g$ como la función \begin{align*}
    f+g : \mathbb{N} & \longrightarrow \mathbb{C} \\
    n & \longmapsto f(n)+g(n)
\end{align*}
y el \textbf{producto} de $f$ y $g$ como la función
\begin{align*}
    fg : \mathbb{N} & \longrightarrow \mathbb{C} \\
    n & \longmapsto f(n)g(n).
\end{align*}
\end{definition}

Es fácil verificar que para cualesquiera funciones aritméticas $f$ y $g$,
\begin{enumerate}[label=\textnormal{(\roman*)}]
\item $f+\mathbf{0}=\mathbf{0}+f=f$ 
\item $f\mathbf{1}=\mathbf{1}f=f$ 
\item $f+g=g+f$
\item $fg=gf$.
\end{enumerate}

\subsection{Convolución de Dirichlet}

\begin{definition}
Sean $f$ una función aritmética, $n\in\mathbb{N}$ y $d_1,d_2,\cdots,d_k$ todos los divisores positivos de $n$. Se define 
\begin{equation*}
	\sum_{d \mid n} f(d)=f(d_1)+f(d_2)+\cdots+f(d_k).
\end{equation*}
\end{definition}

\begin{definition}[Convolución de Dirichlet]
Si $f$ y $g$ son funciones aritméticas, definimos la \textbf{convolución de Dirichlet} o \textbf{producto de Dirichlet} de $f$ y $g$, como la función aritmética $f*g$ dada por 
\begin{equation*}
	(f*g)(n)=\sum_{d \mid n} f(d)g\left(\frac{n}{d}\right),\:\forall \: n\in\mathbb{N}.
\end{equation*}
\end{definition}

Para ver que la operación $*$ es asociativa, conmutativa y distributiva respecto a la suma son necesarios algunos lemas.

\begin{lemma}
Si $k\in\mathbb{N}$, $D\subset \mathbb{N}$ y $f,g: \{1,\ldots,k\} \longrightarrow D$ son dos funciones biyectivas y estrictamente crecientes, entonces $f=g$.
\end{lemma}
\begin{proof}
Como $D\subset \mathbb{N}$ es finito, se pueden ordenar los elementos de $D$. Sea $D=\{d_1,\ldots,d_k\}$, donde $d_1<d_2<\cdots<d_k$. Se tiene que $d_1$ es el elemento mínimo de $D$. Sin embargo, $f(1)\leq f(i)$ y $g(1)\leq g(i),\:\forall \: i=1,\ldots,k$ y como $f$ y $g$ son suprayectivas, entonces $f(1)\leq d_1$ y $g(1)\leq d_1$, además $d_1\leq f(1)$ y $d_1\leq g(1)$ por ser $d_1$ el elemento mínimo de $D$. Luego $f(1)=d_1=g(1)$.
\bigskip

%Ahora, tenemos que $d_2$ es el elemento mínimo de $D \setminus \{d_1\}$. Notemos que $f(2),g(2)\in D \setminus \{d_1\}$. En efecto, si $f(2)=d_1$ o $g(2)=d_1$, entonces $f(2)=f(1)$ o $g(2)=g(1)$ y como ambas funciones son inyectivas, entonces $2=1$, lo cual es absurdo. En consecuencia $d_2\leq f(2)$ y $d_2\leq g(1)$. Por otro lado, como $d_2\in D$, deben existir $i_1,i_2\in \{1,\ldots,k\}$ tales que $f(i_1)=d_2$ y $g(i_2)=d_2$. Más aún, $2\leq i_1$ y $2\leq i_2$, pues en caso contrario se tendría $i_1=1$ o $i_2=1$ y por tanto $f(1)=d_2$ o $g(1)=d_2$, es decir, $d_1=d_2$, lo cual es falso por hipótesis. Luego $f(2)\leq f(i_1)=d_2$ y $g(2)\leq f(i_2)=d_2$ y por tanto $f(2)=d_2=g(2)$.
%\bigskip

Supóngase que $f(i)=d_i=g(i),\:\forall \: i=1,\ldots,n$ y $n+1\leq k$. Si $n+1=k$, como $f$ y $g$ son biyectivas, necesariamente $f(n+1)=d_{n+1}=g(n+1)$. Supóngase pues que $n+1<k$. Se tiene que $d_{n+1}$ es el elemento mínimo del conjunto $D \setminus \{1,\ldots,d_n\}$. Además, $f(n+1),g(n+1)\in D\setminus \{1,\ldots,d_n\}$. En efecto, pues si $f(n+1)=d_{i_1}$ o $g(n+1)=d_{i_2}$, para algunos $i_1,i_2\in \{1,\ldots,n\}$, entonces $f(n+1)=f(i_1)$ y $g(n+1)=g(i_2)$ por hipótesis de inducción y por inyectividad se tendría que $n+1=i_1\leq n$ o $n+1=i_2\leq n$, lo cual es absurdo. En consecuencia $f(n+1),g(n+1)\in D \setminus \{1,\ldots,d_n\}$ y por tanto $d_{n+1}\leq f(n+1)$ y $d_{n+1}\leq g(n+1)$. 
\bigskip

Por otra parte, se tiene por suprayectividad que existen $j_1,j_2\in \{1,\ldots,k\}$ tales que $f(j_1)=d_{n+1}$ y $g(j_2)=d_{n+1}$, más aún, $n+1\leq j_1$ y $n+1\leq j_2$, pues en caso contrario se tendría que $j_1<n$ o $j_2<n$, es decir, $f(j_1)<f(n)$ o $g(j_2)<g(n)$, es decir, $d_{n+1}<d_n$, lo que contradice la hipótesis. Luego $f(n+1)\leq f(j_1)=d_{n+1}$ y $g(n+1)\leq g(j_2)=d_{n+1}$. Se sigue finalmente que $f(n+1)=d_{n+1}=g(n+1)$.
\end{proof}

\begin{lemma}\label{lemma:div1}
Si $n\in\mathbb{N}$ y $d_1=1<d_2<\cdots<d_{k-1}<d_k=n$ son todos los divisores positivos de $n$, entonces $d_i d_{k+1-i}=n,\:\forall \: i=1,\ldots,k$.
\end{lemma}
\begin{proof}
Sea $D=\{d_1,\ldots,d_k\}$ y consideremos las funciones $f: \{1,\ldots,k\} \longrightarrow D$ definida como $f(i)=d_i,\:\forall \: i=1,\ldots,k$ y $g: \{1,\ldots,k\} \longrightarrow D$ definida como $g(i)=n/d_{k+1-i},\:\forall \: i=1,\ldots,k$. Es fácil ver que $f$ y $g$ cumplen las condiciones del lema anterior y por tanto $f(i)=g(i),\:\forall \: i=1,\ldots,k$, es decir, $d_i d_{k+1-i}=n,\:\forall \: i=1,\ldots,k$.
\end{proof}

\begin{proposition}\label{prop:dir1}
Si $f$ y $g$ son funciones aritméticas, $n\in\mathbb{N}$ y $d_1<\cdots<d_k$ son todos los divisores positivos de $n$, entonces 
\begin{equation*}
	(f*g)(n)=\sum_{i=1}^{k} f(d_i)g(d_{k+1-i})=f(d_1)g(d_k)+\cdots+f(d_k)g(d_1).
\end{equation*}
\begin{proof}
Se sigue de la definición de $(f*g)(n)$ y del \Cref{lemma:div1}.
\end{proof}
\end{proposition}

\begin{definition}[Función identidad]\label{def:str2}
Se define a la función identidad $I$ como
\begin{equation*}
	I(n) =
		\begin{cases}
			\hfil 1 & \text{si} \: n=1 \\ 
			\hfil 0 & \text{si} \: n>1,
		\end{cases}
\end{equation*}
para cada $n\in\mathbb{N}$.
\end{definition}

La siguiente proposición muestra que la función $I$ actúa como la identidad bajo la convolución de Dirichlet, entre otras propiedades algebraicas.

\begin{proposition}
Si $f,g$ y $h$ son funciones aritméticas, entonces se verifica lo siguiente:

\begin{enumerate}[label=\textnormal{(\roman*)}]
	\item $(f*g)*h=f*(g*h)$
	\item $f*I=I*f=f$
	\item $f*(g+h)=(f*g)+(f*h)$
	\item $f*g=g*f$
\end{enumerate}
\end{proposition}
\begin{proof}
Sea $n\in\mathbb{N}$, sean $d_1=1<d_2<\cdots<d_k=n$ todos los divisores positivos de $n$ y para cada $i=1,\ldots,k$ sean $c_{i,1}<c_{i,2}<\cdots<c_{i,m_i}$ los divisores positivos de $d_i$.
\bigskip

({\scshape \romannumeral 1}) Se tiene
\begin{equation}\label{eqn:sum1}
	((f*g)*h)(n)=\sum_{i=1}^{k} \sum_{j=1}^{m_i} f(c_{i,j})g(c_{i,m_i+1-j})h(d_{k+1-i})
\end{equation}
y 
\begin{equation}\label{eqn:sum2}
	(f*(g*h))(n)=\sum_{i=1}^{k} \sum_{j=1}^{m_{k+1-i}} f(d_i)g(c_{m_{k+1-i},j})h(c_{m_{k+1-i},m_{k+1-i}+1-j}).
\end{equation}

Defínanse los conjuntos

\begin{align*}
	& \mathcal{C} = \left\{f(c_{i,j})g(c_{i,m_i+1-j})h(d_{k+1-i}) \: \mid i=1,\ldots,k,j=1,\ldots,m_i \: \right\} \\
	& \\
	& \mathcal{D} = \left\{f(d_i)g(c_{m_{k+1-i},j})h(c_{m_{k+1-i},m_{k+1-i}+1-j}) \: \mid i=1,\ldots,k,j=1,\ldots,m_i \: \right\},
\end{align*}

y $\mathcal{E} = \left\{f(a)f(b)f(c) \mid a,b,c\in\mathbb{N} \textrm{ y } a b c=n\right\}$. Se tiene que $\mathcal{C}=\mathcal{E}$ y $\mathcal{D}=\mathcal{E}$.
\bigskip

En efecto, si $f(c_{i,j})g(c_{i,m_i+1-j})h(d_{k+1-i})$, entonces $c_{i,j}c_{i,m_i+1-j}d_{k+1-i}=d_i d_{k+1-i}=n$, aplicando dos veces el \Cref{lemma:div1}. Recíprocamente, si $a,b,c\in\mathbb{N}$ son tales que $a b c = n$, entonces $c \mid n$, por tanto $c=d_j$, para algún $j=1,\ldots,k$, es decir, $c=d_{k+1-i}$ para $i=k+1-j$ con $i=1,\ldots,k$. Notemos entonces que por el \Cref{lemma:div1}, necesariamente se debe tener $a b=d_i$, por lo que $a=c_{i,j}$, para algún $j=1,\ldots,m_i$ y aplicando el lema de nuevo se debe tener que $b=c_{i,m_i+1-j}$. En consecuencia $f(a)g(b)h(c)=f(c_{i,j})g(c_{i,m_i+1-j})h(d_{k+1-i})\in \mathcal{C}$. Se sigue pues que $\mathcal{C}=\mathcal{E}$. Similarmente se demuestra que $\mathcal{D}=\mathcal{E}$.
\bigskip

Se tiene pues que $\mathcal{C}=\mathcal{D}$ y como las sumas \eqref{eqn:sum1} y \eqref{eqn:sum2} se extienden sobre los conjuntos $\mathcal{C}$ y $\mathcal{D}$, entonces deben coincidir, es decir, $((f*g)*h)(n)=(f*(g*h))(n)$.
\bigskip

({\scshape \romannumeral 2}) Como $1<d_i,\:\forall \: i=2,\ldots,k$, entonces $I(d_i)=0,\:\forall \: i=2,\ldots,k$, luego por la \Cref{prop:dir1} se tiene que

\begin{align*}
	(f*I)(n) &= \sum_{i=1}^{k} f(d_i)I(d_{k+1-i})=f(d_1)I(d_k)+\cdots+f(d_k)I(d_1) \\
			 &= f(d_k)I(d_1)=f(n)I(1)=f(n)\cdot 1=f(n)
\end{align*}
Y
\begin{align*}
	(I*f)(n) &= \sum_{i=1}^{k} I(d_i)f(d_{k+1-i})=I(d_1)f(d_k)+\cdots+I(d_k)f(d_1) \\
			 &= I(d_1)f(d_k)=I(n)f(1)=1\cdot f(n)=f(n)
\end{align*}
\bigskip

({\scshape \romannumeral 3}) Se tiene
\begin{align*}
	(f*(g+ & h)) (n) = \sum_{i=1}^{k} f(d_i)(g+h)(d_{k+1-i}) = \sum_{i=1}^{k} f(d_i)[g(d_{k+1-i})+h(d_{k+1-i})] \\
		   & = \sum_{i=1}^{k} f(d_i)g(d_{k+1-i})+\sum_{i=1}^{k} f(d_i)h(d_{k+1-i}) = (f*g)(n)+(f*h)(n).
\end{align*}
\bigskip

({\scshape \romannumeral 4}) La conmutatividad de la convolución de Dirichlet es clara, pues 
\begin{equation*}
	(f*g)(n) = \sum_{i=1}^{k} f(d_i)g(d_{k+1-i}) = \sum_{i=1}^{k} g(d_i)f(d_{k+1-i})=(g*f)(n).
\end{equation*}
\end{proof}

Se puede definir también la bien conocida multiplicación por escalares en el conjunto de funciones aritméticas, análoga a la del espacio euclideo $\mathbb{R}^n$.

\begin{definition}
Dados $c \in \mathbb{C}$ y $f \in \mathcal{A}$, se define $c f \in \mathcal{A}$ como la función
\begin{align*}
    c f : \mathbb{N} & \longrightarrow \mathbb{C} \\
    n & \longmapsto c f(n).
\end{align*}
\end{definition}

\begin{proposition}
El grupo abeliano $(\mathcal{A}, +)$ junto con la multiplicación por escalares definida anteriormente constituyen un espacio vectorial sobre el campo $\mathbb{C}$, donde el elemento neutro aditivo es la función $\mathbf{0}$. De ahora en adelante, este espacio vectorial será llamado simplemente el espacio de las funciones aritméticas.
\end{proposition}

\begin{proposition}
Si $c \in \mathbb{C}$ y $f, g \in \mathcal{A}$, entonces $c(f*g)=(c f)*g=f*(c g)$.
\end{proposition}
\begin{proof}
Se tiene que para cualquier $n \in \mathbb{N}$,
\begin{equation*}
    (c(f*g))(n) = c(f*g)(n) = c \sum_{d \mid n} f(n) g \left( \frac{n}{d} \right) = \sum_{d \mid n} (c f(n)) g \left( \frac{n}{d} \right) = ((c f)*g)(n).
\end{equation*}
Luego $c(f*g)=(c f)*g$. Las demás igualdades se prueban similarmente.
\end{proof}

\begin{corollary}\label{cor:est1}
El anillo $(\mathcal{A},+,*)$ es un álgebra conmutativa con identidad sobre el campo $\mathbb{C}$.
\end{corollary}
