\documentclass{beamer}
\usetheme{metropolis}
\title{Anillo de funciones aritméticas}
\author{José Luis Juanico}
\date{Julio 11, 2025}
\institute{ESFM-IPN}

\begin{document}

\maketitle
\pagestyle{empty}

\section{Convolución de Dirichlet}
\begin{frame}{Funciones aritméticas}
    Dedekind (1857): si $F(n) = \sum f(d)$, donde $d$ recorre todos los divisores positivos de $n$, entonces
    \begin{equation*}
        f(n) = F(n) - \sum F \left( \frac{n}{a} \right) + \sum F \left( \frac{n}{a b} \right) - \sum F \left( \frac{n}{a b c} \right) + \cdots,
    \end{equation*}
    donde las sumas recorre las combinaciones de primos $a, b, \ldots$ de $n$.
\end{frame}

\begin{frame}{Funciones aritméticas}
    Laguerre (1863) expresó esto en la notación
    \begin{equation*}
        f(n) = \sum_{d \mid n} \mu \left( \frac{n}{d} \right) F(d)
    \end{equation*}
    y fue el primero en usarla. Donde $\mu(1) = 1$, $\mu(p_1 p_2 \cdots p_k) = (-1)^k$ y $\mu(n) = 0$ si $n$ es divisible por algún cuadrado (Möbius).
\end{frame}

\begin{frame}{Formalización algebraica}
    \begin{align*}
        & \sum_{d \mid n} \mu(d) = 0, \text{si } n > 1 && \text{(Mertens, 1874)} \\
        & \sum_{d \mid n} \varphi(d) = n && \text{(Gauss, 1801)} \\
        & \sum_{d \mid n} \mu(d) \frac{n}{d} = \varphi(d) && \text{(Laguerre, 1863)}.
    \end{align*}
\end{frame}

\begin{frame}{Formalización algebraica}
    Liouville, J. \emph{Sur quelque functions numeriques}.

    Bell, E. T. \emph{Outline of a theory of arithmetical functions in their algebraic aspects}.
\end{frame}

\begin{frame}{Convolución de Dirichlet}
    \begin{equation*}
        \sum_{d \mid n} f(d) g \left( \frac{n}{d} \right) : = (f * g)(n)
    \end{equation*}
\end{frame}

\begin{frame}{Convolución de Dirichlet}
    \begin{equation*}
        \sum_{d \mid n} f(d) g \left( \frac{n}{d} \right) : = (f * g)(n)
    \end{equation*}
    \begin{equation*}
         f(n) + g(n) : = (f + g)(n)
    \end{equation*}
\end{frame}

\begin{frame}{Anillo de funciones aritméticas}
    \begin{itemize}
    \item $(\mathcal{A},+,*)$ es un álgebra conmutativa con identidad sobre el campo $\mathbb{C}$ (o $\mathbb{C}$-álgebra).
    \item $(\mathcal{A},+)$ es un espacio vectorial sobre $\mathbb{C}$.
    \item $(\mathcal{A},+,*)$ es un dominio entero.
    \end{itemize}
\end{frame}

\begin{frame}{Elementos irreducibles y primos}
    \begin{itemize}
    \item $f \in \mathcal{A}$ es invertible si y sólo si $f(1) \ne 0$.
    \item $\mathcal{N}(f) = p$ primo implica que $f$ es irreducible en $\mathcal{A}$.
    \end{itemize}
\end{frame}

\begin{frame}{Anillo de funciones aritméticas}
    \begin{itemize}
    \item $\mathcal{A}$ es un dominio de factorización.
    \item $\mathcal{A}$ es un dominio de factorización única (DFU).
    \item $\mathcal{A}$ es un anillo local.
    \end{itemize}
\end{frame}

\begin{frame}{Elementos irreducibles y primos}
    Corolario:
    \begin{itemize}
    \item $f$ es primo en $\mathcal{A}$ si y sólo si $f$ es irreducible en $\mathcal{A}$.
    \end{itemize}
    Podríamos haber ahorrado trabajo probando que $\mathcal{A}$ es un dominio de ideales principales, pues todo DIP es un DFU. Este no es el caso.
\end{frame}

\begin{frame}{Condición de la cadena ascendente}
    $\mathcal{A}$ cumple una versión débil de la cadena ascendente: si $(f_1) \subset (f_2) \subset (f_3) \subset \cdots$ y $f_i \nsim f_{i+1}, \forall i \in \mathbb{N}$ (no son asociados), entonces existe $n \in \mathbb{N}$ tal que $(f_i) = (f_n), \forall i \ge n$.
\end{frame}

\begin{frame}{Condición de la cadena descendente}
    A propósito de cadenas descendentes, $\mathcal{A}$ no es Artiniano, como lo muestra la sucesión de ideales $I_n = \{ f \in \mathcal{A} \mid \mathcal{N}(f) \ge n \} \cup \{ \mathbf{0} \}$.
\end{frame}

\section{Isomorfismos}

\begin{frame}{Isomorfismos}
    \begin{equation*}
        (\mathcal{A}_{\mathbb{R}}, +) \cong (P, *)
    \end{equation*}
    \begin{equation*}
        P = \{ f \in \mathcal{A} \mid f(1) > 0 \}
    \end{equation*}
\end{frame}

\begin{frame}{Isomorfismos}
    \begin{equation*}
        (\mathcal{M}, *) \cong (\mathcal{A}', +)
    \end{equation*}
    \begin{equation*}
        \mathcal{A}' = \{ f \in \mathcal{A}_{\mathbb{R}} \mid f(n) = 0, \forall n \ne p^\alpha, p \text{ primo y } \alpha \in \mathbb{N} \}
    \end{equation*}
\end{frame}

\begin{frame}{Isomorfismos}
    \begin{equation*}
        (\mathcal{A}_{\mathbb{R}}, +) \cong (\mathcal{A}', +)
    \end{equation*}
\end{frame}

\begin{frame}{Isomorfismos}
    \begin{equation*}
        (\mathcal{A}_{\mathbb{R}}, +) \cong (\mathcal{A}_1, +)
    \end{equation*}
    \begin{equation*}
        \mathcal{A}_1 = \{ f \in \mathcal{A} \mid f(1) \in \mathbb{R} \}
    \end{equation*}
\end{frame}

\begin{frame}{Isomorfismos}
    \begin{equation*}
        (\mathcal{A}_{\mathbb{R}}, +) \cong (\mathcal{A}, +)
    \end{equation*}
\end{frame}

\begin{frame}{Isomorfismos}
    \begin{equation*}
        (\mathcal{A}_{\mathbb{R}}, +) \cong (P, *) \cong (\mathcal{M},*) \cong (\mathcal{A}', +) \cong (\mathcal{A}_1, +) \cong (\mathcal{A}, +)
    \end{equation*}
\end{frame}

\section{Funciones pares}

\begin{frame}{Funciones pares módulo $r$}
    \begin{equation*}
        f(n) = f((n,r)), \forall n \in \mathbb{N}
    \end{equation*}
\end{frame}

\begin{frame}{Sumas de Ramanujan}
    \begin{align*}
        \frac{\sigma(n)}{n} & = \frac{\pi^2}{6} \Bigg( 1 + \frac{(-1)^n}{2^2} + \frac{2 \cos \left( \frac{2}{3} \pi n \right)}{3^2} + \frac{2 \cos \left( \frac{1}{2} \pi n \right)}{4^2} \\
                            & + \frac{2 \left[ \cos \left( \frac{2}{5} \pi n \right) + \cos \left( \frac{4}{5} \pi n \right) \right]}{5^2} + \frac{2 \cos \left( \frac{1}{3} \pi n \right)}{6^2} + \cdots \Bigg)
    \end{align*}
\end{frame}

\begin{frame}{Sumas de Ramanujan}
    \begin{align*}
        & \frac{\sigma(n)}{n} = \frac{\pi^2}{6} \sum_{r=1}^{\infty} \frac{c_r(n)}{r^2}, \\
        & c_r(n) = \sum_{\substack{a=1 \\ (a,r) = 1}}^{r} \cos \left( \frac{2 \pi}{r} a n \right)
    \end{align*}
\end{frame}

\begin{frame}{Sumas de Ramanujan}
    \begin{equation*}
        c_r(n) = \sum_{d \mid (n,r)} \mu \left( \frac{r}{d} \right) d
    \end{equation*}
\end{frame}

\begin{frame}{Funciones pares}
    \begin{itemize}
    \item Cualquier función constante
    \item $(n,r)$
    \item El número de divisores de $(n,r)$
    \item $f((n,r))$, donde $f$ es cualquier función aritmética
    \item $\sum_{d \mid (n,r)} g(d)$
    \end{itemize}
\end{frame}

\begin{frame}{Expansión de ``Fourier''}
    $f$ par módulo $r$ implica
    \begin{equation*}
        f(n) = \sum_{d \mid r} \alpha(d) c_d(n), \forall n \in \mathbb{N}
    \end{equation*}
    para algunos coeficientes $\alpha$.
\end{frame}

\begin{frame}{Sumas de Ramanujan}
    \centering
    $\mathcal{B}_r = \{ c_d \}_{d \mid r}$ es una base de $\mathcal{A}_r$

    ¡$\mathcal{A}_r$ tiene dimensión $\tau(r)$!
\end{frame}

\begin{frame}{Sumas de Ramanujan}
    $\mathcal{A}_r$ también tiene producto interno
    \begin{equation*}
        \langle f, g \rangle = \sum_{d \mid r} \varphi(d) f \left( \frac{r}{d} \right) \overline{g \left( \frac{r}{d} \right)}
    \end{equation*}
    y
    \begin{equation*}
        \mathcal{B}'_r = \left\{ \frac{1}{\sqrt{r \varphi(d)}} c_d \right\}
    \end{equation*}
    es una base ortonormal de $\mathcal{A}_r$.
\end{frame}

\section{Aplicaciones}

\begin{frame}{Señal}
    \begin{equation*}
        x : \mathbb{Z} \longrightarrow \mathbb{C}
    \end{equation*} 
\end{frame}

\begin{frame}{Señal periódica con periodo $r$}
    \begin{equation*}
        x[n] = x[n + k r], \forall n, k \in \mathbb{Z}
    \end{equation*}
\end{frame}

\begin{frame}{Señal simética módulo $r$}
    \begin{equation*}
        x[n] = x[n - r], \forall n \in \mathbb{Z}
    \end{equation*}
\end{frame}

\begin{frame}{Señal par módulo $r$}
    \begin{equation*}
        x[n] = x[(n,r)], \forall n \in \mathbb{Z}
    \end{equation*}
\end{frame}

\begin{frame}{Implicaciones}
    Toda señal par módulo $r$ es simétrica módulo $r$ y también periódica módulo $r$.
\end{frame}

\begin{frame}{Transformada Discreta de Fourier}
    \begin{equation*}
        X[k] = \sum_{n=1}^{r} x[n] W_r^{n k}, k = 1,\ldots,r,
    \end{equation*}
    \begin{equation*}
        x[n] = \frac{1}{r} \sum_{k=1}^{r} X[k] W_r^{-n k}, n \in \mathbb{Z},
    \end{equation*}
    con $W_r = e^{-j (2 \pi / r)}$ y $x$ periódica con periodo $r$.
\end{frame}

\begin{frame}{Sistemas de residuos}
    Si $d$ es un divisor positivio de $r$, se define
    \begin{equation*}
        S_d = \left\{ \left( \frac{r}{d} \right) \alpha \mid (\alpha, d) = 1, 1 \le \alpha \le d \right\}.
    \end{equation*}
\end{frame}

\begin{frame}{Sistemas de residuos}
    \begin{equation*}
        \bigcup_{d \mid r} S_d = \{ 1,2,\ldots,r \} \hspace{2.5em} (\text{Gauss, 1801})
    \end{equation*}
\end{frame}

\begin{frame}{Funciones indicadoras}
    Se define
    \begin{equation*}
        h_{r,d}[n] = \begin{cases}
            \hfil 1 & \text{si } n \in S_d \\
            \hfil 0 & \text{en otro caso.}
        \end{cases}
    \end{equation*}
    Para $n \in \{ 1,\ldots,r \}$. Se puede extender a todo $\mathbb{Z}$ haciendo $n$ módulo $r$ y la señal $h_{r,d}$ se vuelve par.
\end{frame}

\begin{frame}{Funciones indicadoras}
    Los coeficientes de Fourier clásicos para las funciones $h_{r,d}$, para $d$ divisor de $r$ fijo, están dados por:
    \begin{equation*}
        H_{r,d}[k] = c_d[k], k=1,\ldots,r.
    \end{equation*}
    ¡Sus coeficientes son las sumas de Ramanujan módulo $d$! Conexión entre Gauss y Ramanujan.
\end{frame}

\begin{frame}{Corolarios}
    Si $x$ es par módulo $r$, entonces sus coeficientes de Fourier clásicos están dados por:
    \begin{equation*}
        X[k] = \sum_{d \mid r} x \left[ \frac{r}{d} \right] c_d[k], k=1,\ldots,r.
    \end{equation*}
    Más aún,
    \begin{equation*}
        x[n] = \frac{1}{r} \sum_{d \mid r} X \left[ \frac{r}{d} \right] c_d[n], \forall n \in \mathbb{Z}.
    \end{equation*}
\end{frame}

\begin{frame}{Ejemplo}
    Si por ejemplo, supóngase que se tiene una señal par módulo $10$ y se quiere calcular sus Transformada Discreta Fourier. Se tiene que los sistemas de residuos de 10 son:
    \begin{align*}
        S_1 & = \{ 10 \} \\
        S_2 & = \{ 5 \} \\
        S_5 & = \{ 2, 4, 6, 8 \} \\
        S_{10} & = \{ 1, 3, 5, 7 \} 
    \end{align*}
\end{frame}

\begin{frame}{Ejemplo}
    Luego
    {\scriptsize
    \begin{align*}
        X[1] = X[3] = X[5] = X[7] & = \sum_{d \mid 10} x \left[ \frac{10}{d} \right] c_d[1] = x[10] - x[5] - x[2] + x[1] \\
        X[2] = X[4] = X[6] = X[8] & = \sum_{d \mid 10} x \left[ \frac{10}{d} \right] c_d[2] = x[10] + x[5] - x[2] - x[1] \\
        X[5] & = \sum_{d \mid 10} x \left[ \frac{10}{d} \right] c_d[5] = x[10] - x[5] + 4 x[2] - 4 x[1] \\
        X[10] & = \sum_{d \mid 10} x \left[ \frac{10}{d} \right] c_d[10] = x[10] + x[5] + 4 x[2] + 4 x[1] \\
    \end{align*}}
    Entonces, para obtener el valor de su transformada en cualquier entero basta conocer los valores de $X[1], X[2], X[5], X[10]$.
\end{frame}

\begin{frame}{Ejemplo}
    \begin{equation*}
        \begin{bmatrix}
        1 & -1 & -1 & 1 \\
        1 & 1 & -1 & -1 \\
        1 & -1 & 4 & -4 \\
        1 & 1 & 4 & 4 \\
        \end{bmatrix}
    \end{equation*}
\end{frame}

\begin{frame}[standout]
¡Gracias!
\end{frame}

\end{document}
