\documentclass{beamer}
\usetheme{metropolis}
\title{Anillo de funciones aritméticas}
\author{José Luis Juanico}
\date{Julio 11, 2025}
\institute{ESFM-IPN}

\begin{document}

\maketitle
\pagestyle{empty}

\section{Convolución de Dirichlet}
\begin{frame}{Funciones aritméticas}
    Dedekind (1857): si $F(n) = \sum f(d)$, donde $d$ recorre todos los divisores positivos de $n$, entonces
    \begin{equation*}
        f(n) = F(n) - \sum F \left( \frac{n}{a} \right) + \sum F \left( \frac{n}{a b} \right) - \sum F \left( \frac{n}{a b c} \right) + \cdots,
    \end{equation*}
    donde las sumas recorre las combinaciones de primos $a, b, \ldots$ de $n$.
\end{frame}

\begin{frame}{Funciones aritméticas}
    Laguerre (1863) expresó esto en la notación
    \begin{equation*}
        f(n) = \sum_{d \mid n} \mu \left( \frac{n}{d} \right) F(d)
    \end{equation*}
    y fue el primero en usarla. Donde $\mu(1) = 1$, $\mu(p_1 p_2 \cdots p_k) = (-1)^k$ y $\mu(n) = 0$ si $n$ es divisible por algún cuadrado (Möbius).
\end{frame}

\begin{frame}{Formalización algebraica}
    \begin{align*}
        & \sum_{d \mid n} \mu(d) = 0, \text{si } n > 1 && \text{(Mertens, 1874)} \\
        & \sum_{d \mid n} \varphi(d) = n && \text{(Gauss, 1801)} \\
        & \sum_{d \mid n} \mu(d) \frac{n}{d} = \varphi(d) && \text{(Laguerre, 1863)}.
    \end{align*}
\end{frame}

\begin{frame}{Formalización algebraica}
    Liouville, J. \emph{Sur quelque functions numeriques}.

    Bell, E. T. \emph{Outline of a theory of arithmetical functions in their algebraic aspects}.
\end{frame}

\begin{frame}{Convolución de Dirichlet}
    \begin{equation*}
        \sum_{d \mid n} f(d) g \left( \frac{n}{d} \right) : = (f * g)(n)
    \end{equation*}
\end{frame}

\begin{frame}{Convolución de Dirichlet}
    \begin{equation*}
        \sum_{d \mid n} f(d) g \left( \frac{n}{d} \right) : = (f * g)(n)
    \end{equation*}
    \begin{equation*}
         f(n) + g(n) : = (f + g)(n)
    \end{equation*}
\end{frame}

\begin{frame}{Anillo de funciones aritméticas}
    \begin{itemize}
    \item $(\mathcal{A},+,*)$ es un álgebra conmutativa con identidad sobre el campo $\mathbb{C}$ (o $\mathbb{C}$-álgebra).
    \item $(\mathcal{A},+)$ es un espacio vectorial sobre $\mathbb{C}$.
    \item $(\mathcal{A},+,*)$ es un dominio entero.
    \end{itemize}
\end{frame}

\begin{frame}{Elementos irreducibles y primos}
    \begin{itemize}
    \item $f \in \mathcal{A}$ es invertible si y sólo si $f(1) \ne 0$.
    \item $\mathcal{N}(f) = p$ primo implica que $f$ es irreducible en $\mathcal{A}$.
    \end{itemize}
\end{frame}

\begin{frame}{Anillo de funciones aritméticas}
    \begin{itemize}
    \item $\mathcal{A}$ es un dominio de factorización.
    \item $\mathcal{A}$ es un dominio de factorización única (DFU).
    \item $\mathcal{A}$ es un anillo local.
    \end{itemize}
\end{frame}

\begin{frame}{Elementos irreducibles y primos}
    Corolario:
    \begin{itemize}
    \item $f$ es primo en $\mathcal{A}$ si y sólo si $f$ es irreducible en $\mathcal{A}$.
    \end{itemize}
    Podríamos haber ahorrado trabajo probando que $\mathcal{A}$ es un dominio de ideales principales, pues todo DIP es un DFU. Este no es el caso.
\end{frame}

\begin{frame}{Condición de la cadena ascendente}
    $\mathcal{A}$ cumple una versión débil de la cadena ascendente: si $(f_1) \subset (f_2) \subset (f_3) \subset \cdots$ y $f_i \nsim f_{i+1}, \forall i \in \mathbb{N}$ (no son asociados), entonces existe $n \in \mathbb{N}$ tal que $(f_i) = (f_n), \forall i \ge n$.
\end{frame}

\begin{frame}{Condición de la cadena descendente}
    A propósito de cadenas descendentes, $\mathcal{A}$ no es Artiniano, como lo muestra la sucesión de ideales $I_n = \{ f \in \mathcal{A} \mid \mathcal{N}(f) \ge n \} \cup \{ \mathbf{0} \}$.
\end{frame}

\section{Isomorfismos}

\end{document}
