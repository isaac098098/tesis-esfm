%%% Introducción
\phantomsection
\section*{Introducción}
\markboth{Introducción}{Introducción}
\addcontentsline{toc}{section}{Introducción}

Euclides (circa 325 a. C.) probó que $2^{p-1}(2^p-1)$ es un número perfecto si y sólo si $2^p-1$ es un número primo \cite{Di1}. En 1638, Descartes afirmó que todo número perfecto par es de la forma descrita por Euclides y que todo número perfecto impar debe ser de la forma $ps^2$, con $p$ un número primo. Dickson \cite{Di1} dió una prueba para lo primero, aunque Euler ya había probado ambos casos mucho antes y sus demostraciónes no salieron a la luz hasta después de su muerte.
\bigskip

Mersenne publicó en 1644 los primeros once números perfectos pares, dados por la fórmula de Euclides para los primos $p=2,3,5,7,13,17,19,31,67,127$ y $257$, pero se equivocó al agregar al $67$ y al no incluir a los primos $61,89$ y $107$. Más tarde en 1640 y probablemente influenciado por éstas observaciones sobre números perfectos, Fermat consideró a los números de la forma $a^{p-1}-1$, donde $p$ es cualquier primo y $a$ cualquier entero. Observó que si $a$ no es divisible por $p$ entonces $a^{p-1}-1$ es divisible por $p$, hecho que se conocería más tarde como el Teorema de Fermat. El caso $a=2$ era conocido en China al menos desde el año 500 a. C. La primera demostración publicada fue dada por Euler en 1736 y más tarde generalizada a cualquier entero: si $\varphi(n)$ denota el número de enteros positivos menores que $n$ que son primos relativos a $n$, entonces $a^{\varphi(n)}-1$ es divisible por $n$, para cualquier entero $a$ primo relativo a $n$.
\bigskip

Aunque la función $\varphi$ de Euler se definió inicialmente para enunciar la generalización anterior, ésta posee remarcables propiedades que hacen valer la pena estudiarla por sí misma. Por ejemplo, en 1801, Gauss probó que si $n\in\mathbb{N}$ y $d_1,d_2,\cdots,d_k$ son todos los divisores positivos de $n$, entonces $\varphi(d_1)+\varphi(d_2)+\cdots+\varphi(d_k)=n$.
