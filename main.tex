%documentclass
\documentclass[titlepage,letterpaper,twoside]{article}
%use twoside if using fancyhdr
%use titlepage to center title

%Packages

\usepackage[utf8]{inputenc} %character codification

%Layout
\usepackage[left=3.0cm,right=3.0cm,top=3.0cm,bottom=3.3cm,headheight=0.8cm,asymmetric,showframe]{geometry} %page size
%use headheight to control the distance between header and text
%use asymmetric to fix marginpar notes to one side
\usepackage{fancyhdr} %footers and headers

%Font size
\usepackage{anyfontsize}
\usepackage[fontsize=13pt]{scrextend}

%Language
\usepackage[spanish,activeacute,es-tabla]{babel} %languages
%use activeacute to directly type accents
%use es-lcroman to lowercase roman numerals
%use es-tabla to translate table to tabla instead of cuadro

%Math packages
\usepackage{amsthm} %theorem alike environmetns
\usepackage[tbtags]{amsmath} %math
%use tbtags to tag equations at the bottom
\usepackage{amssymb} %more math symbols
\usepackage{statmath} %math mode bold, use \bf[char]
\usepackage{esvect} %better vector arrows
\usepackage{leftindex} %left indexes
\usepackage{centernot} %centernot symbols

%Formatting
\usepackage[hyphens]{url} %hyperlinks in bibliography
%use hyphens to break hyperlinks
%load before hyperref
\usepackage[hidelinks]{hyperref} %clickable cross-references and url's
%use hidelinks to remove hyperlink's boxes
\usepackage[spanish]{cleveref} %reference element type
\usepackage{bookmark} %pdf bookmarks
\usepackage[nottoc]{tocbibind} %include bib, index and figure/table lists in toc automatically
%use nottoc to exclude toc in toc
\usepackage{tocloft} %toc, tof and tot formatting
%use titles option to change default toc and tof titles
%\setlength{\cftbeforesecskip}{6pt} %space before section in toc
%\setlength{\cftbeforesubsecskip}{2pt} %vertical space before subsection in toc
%\setlength{\cftbeforesubsubsecskip}{2pt} %vertical space before subsubsection in toc
\usepackage{titlesec} %custom sections title format
\usepackage{enumitem} %lists formatting
\usepackage{caption} %figure and table captions
\usepackage{float} %floats positioning, [H] option
%\usepackage[tocskip=2pt]{parskip} %paragraph separation
%use tocskip to add space in toc
\usepackage{libertine}
\usepackage[libertine]{newtxmath}

% Bibliography and cites
\usepackage{natbib}

% Diagrams and plots
\usepackage{tikz}
\usepackage{pgfplots} %advanced data plotting

%Tables
\usepackage{array} %math arrays
\usepackage{fancyhdr} %headers
%\usepackage{booktabs} %book tables
%\usepackage{cellspace} %vertical table spacing
%use
%\setlength\cellspacetoplimit{0pt}
%\setlength\cellspacebottomlimit{0pt}
%\usepackage{multirow} %multiple row cells

%Other
%\usepackage{lipsum}
%\usepackage{relsize} %relative text resizing
\usepackage{scalerel} %relative math text resizing
\usepackage{xcolor}

%Preamble

%Formatting
\decimalpoint %decimal point
\renewcommand\qedsymbol{$\blacksquare$} %black filled qed symbol
\numberwithin{equation}{section} %equation numbered by section
\pgfplotsset{compat=1.18} %plot style
%\pgfplotsset{compat=1.18,ytick style={draw=none},xtick style={draw=none},width=9.0cm,height=8.3cm} %plot style
\hypersetup{breaklinks=true} %hyperref, allow hyperlinks to break
%\setlength{\textfloatsep}{0.1cm} %vertical spacing between floats and text
\allowdisplaybreaks %allow equations to break across pages
\linespread{1} %vertical spacing between lines, relative

%Layout

\setlength\parindent{0pt} %paragraph indentation
%\setlength\columnsep{0.5cm} %horizontal column separation
%\setlength{\parskip}{0.35cm} %vertical paragraph separation
\renewcommand{\headruleskip}{1mm} %distance between header ruler and header text

%Section formatting, depends on titlesec

\titlespacing*{\section}{0cm}{0.7cm}{0.7cm}
\titlespacing*{\subsection}{0em}{0.7cm}{0.7cm}
%use * to apply to all instances
%\titlespacing*{sec}{left spacing}{above spacing}{below spacing}

%\titleformat{\section}{\centering\large\scshape}{\thesection.}{0.5em}{}
\titleformat{\section}{\Large}{\thesection.}{0.5em}{}
%\titleformat{\subsection}{\centering\normalsize\scshape}{\thesubsection.}{0.5em}{}
\titleformat{\subsection}{\large}{\thesubsection.}{0.5em}{}
%titleformat{sec}{format}{number format}{number to title spacing}{prenote (between number and title)}
\renewcommand{\thesection}{\arabic{section}} %section number format
%\renewcommand{\thesection}{\normalfont\Roman{section}} %subsection number format
%\renewcommand{\thesubsection}{\Roman{section}.\arabic{subsection}} %subsection number format
\renewcommand{\thesubsection}{\arabic{section}.\arabic{subsection}} %subsection number format

%toc
%does not affect section titles in the document body

\addto{\captionsspanish}{\renewcommand{\contentsname}{Índice general \vspace{0.5cm}}} %babel, toc title format
%use addto to add title to spanish title definitions
%use captionenglish otherwise
\renewcommand{\cfttoctitlefont}{\Large} %toc title format
\renewcommand{\cftsecfont}{\bfseries} %section entry format
\renewcommand{\cftsecpagefont}{} %section entry number format
\renewcommand{\cftsubsecpagefont}{} %subsection entry number format
\renewcommand{\cftsecleader}{\cftdotfill{\cftdotsep}} %dot leaders (dot lines)
\renewcommand{\cftsecaftersnum}{.} %dot after section number
\renewcommand{\cftsubsecaftersnum}{.} %dot after subsection number
%\renewcommand{\cftsecnumwidth}{0pt} %horizontal space after section number
%\renewcommand{\cftsubsecnumwidth}{0pt} %horizontal space after subsection number
%\renewcommand{\cftsecafterpnum}{\vskip0pt} %vertical spacing between sections
%\renewcommand{\cftsecafterpnum}{\vskip0pt} %vertical spacing between subsections
%\addtocontents{toc}{\vspace{-0.5cm}} %vertical spacing after title

%Bibliography, cites and index

\setcitestyle{square,comma,numbers,sort} %natbib, 
%use square to use square brackets in citations
%use comma to use commas in multiple citations
%use numbers to change to number citation format
%use sort to sort citations in alphabetical order

%\addto{\captionsspanish}{\renewcommand{\bibname}{Bibliografía}} %babel, biblatex, changes bib section title
\addto{\captionsspanish}{\renewcommand{\refname}{Bibliografía}} %babel, bibtex, changes bib section title

%\makeatletter %deprecated
%\def\adjustpenalty{\@beginparpenalty\@M \@itempenalty\@M}
%\makeatother 

%Pagestyles, depends on fancyhdr

%section only header, use \leftmark
%\renewcommand{\sectionmark}[1]{\markboth{\scshape{#1}}{}}

%section and number header, use \leftmark
%\renewcommand{\sectionmark}[1]{\markboth{\thesection.\enskip\scshape{#1}}{}}

%section and subsection header, use \rightmark
\renewcommand{\sectionmark}[1]{\markright{\thesection.\enskip\scshape{#1}}{}}
\renewcommand{\subsectionmark}[1]{\markright{\thesubsection.\enskip\scshape{#1}}{}}

\makeindex

%pagestyles

\fancypagestyle{title-subsection}{
	\fancyhf{} %clear footer and header
	\fancyheadoffset{0.5cm} %fancyhdr, left and right header offset
	\fancyhead[LO]{\hspace{0.8cm}\small\scshape\nouppercase\notestitle} %left odd
	\fancyhead[RO]{\small\thepage\hspace{0.8cm}} %right odd
	\fancyhead[LE]{\hspace{0.8cm}\small\thepage} %left even
	\fancyhead[RE]{\small\scshape\nouppercase\rightmark\hspace{0.8cm}} %right even
}

\fancypagestyle{plain}{
	\fancyhf{} %clear footer and header
	\renewcommand{\headrulewidth}{0pt} %hide header ruler
	\cfoot{\thepage} %footer
}

%Boxes
\setlength\fboxsep{8pt}
\setlength\fboxrule{0.7pt}

%Metadata, depends on hyperref
\title{\Huge\scshape\notestitle}
\date{\today}
\author{\normalsize\notesprof \\ \normalsize\notesauthor}
\date{}

%\hypersetup{
	%pdftitle={},
	%pdfsubject={},
	%pdfauthor={},
	%pdfkeywords={}
%}

%Symbol resizing, depends on scalerel

\let\oldcap\cap
\renewcommand{\cap}{\mathbin{\scaleobj{1.1}{\oldcap}}}
\let\oldcup\cup
\renewcommand{\cup}{\mathbin{\scaleobj{1.1}{\oldcup}}}

\let\oldpartial\partial
\renewcommand{\partial}{\mathbin{\scaleobj{1.15}{\oldpartial}}}

\let\oldsubset\subset
\renewcommand{\subset}{\mathbin{\scaleobj{1.1}{\oldsubset}}}
\let\oldsubseteq\subseteq
\renewcommand{\subseteq}{\mathbin{\scaleobj{1.1}{\oldsubseteq}}}

\let\oldsupset\supset
\renewcommand{\supset}{\mathbin{\scaleobj{1.1}{\oldsupset}}}
\let\oldsupseteq\supseteq
\renewcommand{\supseteq}{\mathbin{\scaleobj{1.1}{\oldsupseteq}}}

%Theorem-like environments

\newtheoremstyle{theorem}
{\topsep} %Separación superior
{\topsep} %Separación inferior
{\itshape} %Fuente del cuerpo
{0em} %Sangría
{\bfseries} %Fuente del encabezado
{.} %Puntuación después del encabezado
{0.5em} %Espacio después del encabezado
{\thmname{#1}\thmnumber{ #2}\textnormal{\thmnote{ (#3)}}} %Teorema [número] [nombre]
\theoremstyle{theorem}
\newtheorem{theorem}{Teorema}[section]

\newtheoremstyle{proposition}
{\topsep} %Separación superior
{\topsep} %Separación inferior
{\itshape} %Fuente del cuerpo
{0em} %Sangría
{\bfseries} %Fuente del encabezado
{.} %Puntuación después del encabezado
{0.5em} %Espacio después del encabezado
{\thmname{#1}\thmnumber{ #2}\textnormal{\thmnote{ (#3)}}} %Proposición [número] [nombre]
\theoremstyle{proposition}
\newtheorem{proposition}{Proposición}[section]

\newtheoremstyle{corollary}
{\topsep} %Separación superior
{\topsep} %Separación inferior
{\itshape} %Fuente del cuerpo
{0em} %Sangría
{\bfseries} %Fuente del encabezado
{.} %Puntuación después del encabezado
{0.5em} %Espacio después del encabezado
{\thmname{#1}\thmnumber{ #2}\textnormal{\thmnote{ (#3)}}} %Corolario[número] [nombre]
\theoremstyle{corollary}
\newtheorem{corollary}{Corolario}[section]

\newtheoremstyle{lemma}
{\topsep} %Separación superior
{\topsep} %Separación inferior
{\itshape} %Fuente del cuerpo
{0em} %Sangría
{\bfseries} %Fuente del encabezado
{.} %Puntuación después del encabezado
{0.5em} %Espacio después del encabezado
{\thmname{#1}\thmnumber{ #2}\textnormal{\thmnote{ (#3)}}} %Lema [número] [nombre]
\theoremstyle{lemma}
\newtheorem{lemma}{Lema}[section]

\newtheoremstyle{definition}
{\topsep} %Separación superior
{\topsep} %Separación inferior
{\itshape} %Fuente del cuerpo
{0em} %Sangría
{\bfseries} %Fuente del encabezado
{.} %Puntuación después del encabezado
{0.5em} %Espacio después del encabezado
{\thmname{#1}\thmnumber{ #2}\textnormal{\thmnote{. (#3)}}} %Definición [número] [nombre]
\theoremstyle{definition}
\newtheorem{definition}{Definición}[section]

\newtheoremstyle{remark}
{\topsep} %Separación superior
{\topsep} %Separación inferior
{\rmfamily} %Fuente del cuerpo
{0em} %Sangría
{\itshape} %Fuente del encabezado
{.} %Puntuación después del encabezado
{0.5em} %Espacio después del encabezado
{\thmname{#1}\thmnumber{ #2}\textnormal{\thmnote{ (#3)}}} %Observación [número] [nombre]
\theoremstyle{remark}
\newtheorem{remark}{Observación}[section]

\newtheoremstyle{exercise}
{\topsep} %Separación superior
{\topsep} %Separación inferior
{\rmfamily} %Fuente del cuerpo
{0em} %Sangría
{\scshape} %Fuente del encabezado
{.} %Puntuación después del encabezado
{0.5em} %Espacio después del encabezado
{\thmname{#1}\thmnumber{ #2}\textnormal{\thmnote{ (#3)}}} %Ejercicio [número] [nombre]
\theoremstyle{exercise}
\newtheorem{exercise}{Ejercicio}[section]

\newtheoremstyle{example}
{\topsep} %Separación superior
{\topsep} %Separación inferior
{\rmfamily} %Fuente del cuerpo
{0em} %Sangría
{\scshape} %Fuente del encabezado
{.} %Puntuación después del encabezado
{0.5em} %Espacio después del encabezado
{\thmname{#1}\thmnumber{ #2}\textnormal{\thmnote{ (#3)}}} %Ejemplo [número] [nombre]
\theoremstyle{example}
\newtheorem{example}{Ejemplo}[subsection]

%Lectures
\newcommand{\seclecture}[2]{
	\section{#1}
    \marginpar{\footnotesize\textsf{\mbox{#2}}}
}
\newcommand{\sublecture}[2]{
	\subsection{#1}
    \marginpar{\footnotesize\textsf{\mbox{#2}}}
}

%Macros

\newcommand{\Mod}[1]{\mathrm{m\acute{o}d\:}#1}

%Operators

\DeclarePairedDelimiter\bra{\langle}{\rvert}
\DeclarePairedDelimiter\ket{\lvert}{\rangle}
\DeclarePairedDelimiterX\braket[2]{\langle}{\rangle}{#1\,\delimsize\vert\,\mathopen{}#2}

%Trigonometric functions

\DeclareMathOperator{\arccot}{arccot}
\DeclareMathOperator{\arcsec}{arcsec}
\DeclareMathOperator{\arccsc}{arccsc}
\DeclareMathOperator{\sech}{sech}
\DeclareMathOperator{\csch}{csch}
\DeclareMathOperator{\arcsinh}{arcsinh}
\DeclareMathOperator{\arccosh}{arccosh}
\DeclareMathOperator{\arctanh}{arctanh}
\DeclareMathOperator{\arccoth}{arccoth}
\DeclareMathOperator{\arcsech}{arcsech}
\DeclareMathOperator{\arccsch}{arccsch}

%Logic

\let\oldforall\forall
\renewcommand{\forall}{\oldforall\,}
\let\oldexists\exists
\renewcommand{\exists}{\:\oldexists\:}
\let\oldnexists\nexists
\renewcommand{\nexists}{\:\oldnexists\:}

%Sets and inclusion

\newcommand{\std}{\, : \,}

%Derivatives

\newcommand{\dx}{\,\text{d}}
\newcommand{\dr}{\text{d}}
\newcommand{\der}[2]{\frac{\dr#1}{\dr#2}}
\newcommand{\Der}[2]{\frac{\dr}{\dr#2}#1}
\newcommand{\ndr}[3]{\frac{\dr^{#1}#2}{\dr#3^{#1}}}
\newcommand{\Ndr}[3]{\frac{\dr^{#1}}{\dr#3^{#1}}#2}
\newcommand{\pdr}[2]{\frac{\partial#1}{\partial#2}}
\newcommand{\Pdr}[2]{\frac{\partial}{\partial#2}#1}
\newcommand{\npd}[3]{\frac{\partial^{#1}#2}{\partial#3^{#1}}}
\newcommand{\Npd}[3]{\frac{\partial^{#1}}{\partial#3^{#1}}#2}
\newcommand{\evl}[1]{\mathrel{\bigg|_{#1}}}

\begin{document}
%%% Portada
\phantomsection
\markboth{Portada}{Portada}
\addcontentsline{toc}{section}{Portada}
\thispagestyle{empty}
\pagenumbering{roman}
\begin{center}
    \begin{minipage}{0.1\textwidth}
        \centering
        \includegraphics[width=0.95\textwidth]{img/ipn.jpg}
    \end{minipage}
    \hspace{10pt}
    \begin{minipage}{0.7\textwidth}
        \centering
        {\Large\textsc{Instituto Politécnico Nacional}} \\
        {\large\textsc{Escuela Superior de Física y Matemáticas}}
        \medskip
    \end{minipage}
    \hspace{10pt}
    \begin{minipage}{0.1\textwidth}
        \centering
        \includegraphics[width=\textwidth]{img/esfm.jpg}
    \end{minipage}
    \begin{center}
        \begin{tikzpicture}
            \draw[thick, black] (0,0)--(0.99\textwidth,0);
        \end{tikzpicture}
    \end{center}
\end{center}

\vfill

\begin{center}
    \makebox[0.85\linewidth][s]{\LARGE \scshape Estructura del anillo de funciones}
    \makebox[0.85\linewidth][s]{\LARGE \scshape aritméticas {} y {} una {} aplicación}
    \makebox[0.85\linewidth][s]{\LARGE \scshape al {} procesamiento {} de {} señales}
\end{center}

\vspace{20pt}

\begin{center}
    \begin{minipage}{0.7\textwidth}
        \centering
        {\Large\textsc{Propuesta de Tesis}} \\
        {\normalsize\textsc{Que para obtener el título de}} \\
        {\normalsize\textsc{Licenciado en Física y Matemáticas}}
        \medskip
    \end{minipage}
\end{center}

\vspace{5pt}

\begin{center}
    \begin{minipage}{0.7\textwidth}
        \centering
        {\normalsize\textsc{Presenta}} \\
        {\Large\textsc{José Luis Juanico López}}
        \medskip
    \end{minipage}
\end{center}

\hspace{10pt}

\begin{center}
    \begin{minipage}{0.7\textwidth}
        \centering
        {\normalsize\textsc{Asesor de Tesis}} \\
        {\normalsize\textsc{Dr. Pablo Lam Estrada}}
        \medskip
    \end{minipage}
\end{center}

\vfill

\begin{center}
    \begin{minipage}{0.8\textwidth}
        {\normalsize\textsc{Cuidad de México} \hfill {\small Marzo de 2025}}
    \end{minipage}
\end{center}

\newpage
\thispagestyle{empty}
\

\newpage
%%% Dedicatoria
\phantomsection
\markboth{Dedicatoria}{Dedicatoria}
\addcontentsline{toc}{section}{Dedicatoria}
\thispagestyle{empty}
\hfill\textit{A mis padres, Adriana y Antonio.}
\newpage
\thispagestyle{empty}
\

\newpage
%%% Agradecimientos
\phantomsection
\markboth{Agradecimientos}{Agradecimientos}
\addcontentsline{toc}{section}{Agradecimientos}
\thispagestyle{plain}
\section*{Agradecimientos}

Debo agradecer a mis padres, quienes en su infinita paciencia vieron como se gestó este trabajo de inicio a fin, con la incertidumbre de si se terminaría algún día. No es necesario mencionar que sin su apoyo económico y emocional este trabajo no existiría.
\bigskip

Debo darle las gracias también a mi asesor, el Dr. Pablo Lam, por su atención incodicional durante toda la realización de este trabajo y su firme convicción en que se podría terminar de forma satisfactoria, sin mencionar la gran ayuda y orientación sobre todos los detalles técnicos y burocráticos necesarios para iniciar y terminar esta tesis. No está de más agradecer a la Dra. Myriam Maldonado, al Dr. César Escobar, al M. en C. Andrés Sabino y al Dr. David Bretón por tomarse la molestia de leer este trabajo y ayudarme a mejorarlo. A final de cuentas, un trabajo siempre es incompleto si no incluye enfoques, observaciones y puntos de vista diferentes al de uno. Cualquier error que se encuentre en este trabajo recae únicamente sobre mi.
\bigskip

Finalmente, siento una inmensa gratitud a la ESFM, o lo que es lo mismo, a todos los profesores que me formaron como matemático en este centro de estudios. De principio a fin me sentí fascinado por los enigmáticos conocimientos que se nos presentaban, y no hubo profesor del que no aprendiera algo maravilloso. Los temas que elegí para discutir aquí están en gran parte influenciados por mis más tiernos inicios al entrar a esta escuela, particularmente por la unidad de álgebra I, a cuyos contenidos volví para desarrollar gran parte de la tesis. En mi opinión, esto prueba que la formación que esta escuela ofrece es de carácter universal y fundamental, abriéndole las puertas a todos sus alumnos para profundizar en cualquier rama de las matemáticas que deseen.
\newpage
\thispagestyle{empty}
\

\newpage
%%% Índice general
\phantomsection
\markboth{Índice general}{Índice general}
\addcontentsline{toc}{section}{Índice general}
\thispagestyle{plain}
\tableofcontents
\newpage
\thispagestyle{empty}
\

%%% Introducción
\phantomsection
\section*{Introducción}
\markboth{Introducción}{Introducción}
\addcontentsline{toc}{section}{Introducción}

En 1640 Fermat afirmó que poseía una demostración del hecho de que si $p$ es un número primo y $x$ es cualquier entero no divisible por $p$, entonces $x^{p-1}-1$ es divisible por $p$. Ahora llamado Teorema de Fermat, es uno de los teoremas fundamentales de la teoría de números \cite{Di1}. Este resultado fue generalizado más tarde por Euler en 1760: si $\varphi(n)$ denota el número de enteros positivos no mayores a $n$ que son primos relativos a $n$, entonces $x^{\varphi(n)-1}-1$ es divisible por $p$.
Aunque la función $\varphi$ de Euler se definió para enunciar la generalización anterior, ésta posee remarcables propiedades que hacen valer la pena estudiarla por sí misma. Por ejemplo, en 1801, Gauss probó que si $n\in\mathbb{N}$ y $d_1,d_2,\cdots,d_k$ son todos los divisores positivos de $n$, entonces $\varphi(d_1)+\varphi(d_2)+\cdots+\varphi(d_k)=n$.
\bigskip

A. F. Möbius definió la función $\mu(n)$ como cero si $n$ es divisible por un cuadrado distinto de 1, y como $(-1)^k$ si $n$ es producto de $k$ primos distintos, mientras que $\mu(1)=1$ y empleó dicha función en la inversión de series:
\begin{equation*}
    F(x) = \sum_{s=1}^{\infty} \frac{f(s x)}{s^n} \text{ implica } f(x) = \sum_{s=1}^{\infty} \mu(s) \frac{F(s x)}{s^n}.
\end{equation*}

Dedekind probó que si $F(m) = \sum f(d)$, donde $d$ recorre todos los divisores positivos de $m$, entonces
\begin{equation*}
    f(n) = F(n) - \sum F \left( \frac{n}{a} \right) + \sum F \left(  \frac{n}{a b} \right) - \sum F \left( \frac{n}{a b c} \right) + \cdots,
\end{equation*}
donde las sumas se extienden sobre todas las combinaciónes de los distintos factores primos $a, b, \ldots$ de $n$. Laguerre expresó la ecuación anterior como
\begin{equation*}
    f(n) = \sum \mu \left(  \frac{n}{d} \right) F(d).
\end{equation*}
En particular, como $\sum \varphi(d) = n$, se tiene
\begin{equation*}
    \varphi(n) = n - \sum \frac{n}{a} + \sum \frac{n}{a b} - \cdots = n \left( 1 - \frac{1}{a} \right) \left( 1 - \frac{1}{b} \right) \cdots
\end{equation*}

F. Mertens notó que $\sum \mu(d) = 0$ si $n>1$, donde $d$ recorre todos los divisores positivos de $n$.
\bigskip

N. V. Bugaiev consideró la función $\nu(x)$ con valor $\log p$ si $x$ es potencia de un primo $p$ y con valor 0 en otro caso. Si $d$ recorre todos los divisores positivos de $n$, $\sum \nu(d) = \log n$ implica que $\sum \mu(d) \log d = -\nu(n)$. Bugaiev llamó a $F(n) = \sum f(d)$, la integral numérica de $f(n)$, donde la suma es sobre todos los divisores positivos $d$ de $n$, y llamó a $f(n)$ la derivada numérica de la función $F(n)$.
\bigskip

En 1857 Liouville estableció sin prueba un gran número de identidades interesantes, en sus cuatro artículos \emph{Sur quelques functions numeriques}, sobre funciones ariméticas específicas, como la suma y número de divisores de un entero, la función $\varphi$ de Euler, la función de Möbius $\mu$, su propia función $\lambda$, etc. Afirmó que estaba en posesión de un método general de extrema simplicidad, con el que tales identidades se podrían escribir a voluntad. Tales identidades provaron ser un valioso punto de partida para la evaluación asintótica de funciones aritméticas, pero su interés peculiar era más bien algebráico.

%%% Estructura del anillo
\newpage
\thispagestyle{empty}
\
\newpage
\pagenumbering{arabic}
\section{Estructura del anillo de funciones aritméticas}

\begin{definition}
Se entenderá como \textbf{función aritmética} a cualquier función $f : \mathbb{N} \longrightarrow \mathbb{C}$. Se denota al conjunto de todas las funciones aritméticas como $\mathcal{A}$.
\end{definition}

\begin{definition}[Función constante]
La función aritmética constante de valor $c \in \mathbb{C}$ en $\mathbb{N}$ será escrita en negritas como $\mathbf{c}$. Por ejemplo, $\mathbf{1}(n)=1, \forall n \in \mathbb{N}$.
\end{definition}

La siguiente función aritmética, conocida como función de Möbius, es de importancia central en la teoría de números. Aunque a primera vista su definición parece más bien artificial, se verá que aparece naturalmente al derivar propiedades del producto de Dirichlet.

\begin{definition}[Función de Möbius]
La función $\mu$ de Möbius está definida por $\mu(1)=1$ y dada $n=p_1^{\alpha_1}p_2^{\alpha_2}\cdots p_k^{\alpha_k}$ la factorización de $n$ en primos, entonces
\begin{equation*}
	\mu(n) =
		\begin{cases}
			(-1)^k & \text{si} \: \alpha_1=\alpha_2=\cdots=\alpha_k=1 \\ \hfil
			0 & \text{en otro caso.}
		\end{cases}
\end{equation*}
\end{definition}

Una primera forma natural de operar funciones aritméticas es haciéndo su suma o multiplicación puntual, obteniéndo otra función aritmética.

\begin{definition}
Si $f,g \in \mathcal{A}$, definimos la \textbf{suma} de $f$ y $g$ como la función \begin{align*}
    f+g : \mathbb{N} & \longrightarrow \mathbb{C} \\
    n & \longmapsto f(n)+g(n)
\end{align*}
y el \textbf{producto} de $f$ y $g$ como la función
\begin{align*}
    fg : \mathbb{N} & \longrightarrow \mathbb{C} \\
    n & \longmapsto f(n)g(n).
\end{align*}
\end{definition}

Es fácil verificar que para cualesquiera funciones aritméticas $f$ y $g$,
\begin{enumerate}[label=\textnormal{(\roman*)}]
\item $f+\mathbf{0}=\mathbf{0}+f=f$ 
\item $f\mathbf{1}=\mathbf{1}f=f$ 
\item $f+g=g+f$
\item $fg=gf$.
\end{enumerate}

\subsection{Convolución de Dirichlet}

\begin{definition}
Sean $f$ una función aritmética, $n\in\mathbb{N}$ y $d_1,d_2,\cdots,d_k$ todos los divisores positivos de $n$. Se define 
\begin{equation*}
	\sum_{d \mid n} f(d)=f(d_1)+f(d_2)+\cdots+f(d_k).
\end{equation*}
\end{definition}

\begin{definition}[Convolución de Dirichlet]
Si $f$ y $g$ son funciones aritméticas, definimos la \textbf{convolución de Dirichlet} o \textbf{producto de Dirichlet} de $f$ y $g$, como la función aritmética $f*g$ dada por 
\begin{equation*}
	(f*g)(n)=\sum_{d \mid n} f(d)g\left(\frac{n}{d}\right),\:\forall \: n\in\mathbb{N}.
\end{equation*}
\end{definition}

Para ver que la operación $*$ es asociativa, conmutativa y distributiva respecto a la suma son necesarios algunos lemas.

\begin{lemma}
Si $k\in\mathbb{N}$, $D\subset \mathbb{N}$ y $f,g: \{1,\ldots,k\} \longrightarrow D$ son dos funciones biyectivas y estrictamente crecientes, entonces $f=g$.
\end{lemma}
\begin{proof}
Como $D\subset \mathbb{N}$ es finito, se pueden ordenar los elementos de $D$. Sea $D=\{d_1,\ldots,d_k\}$, donde $d_1<d_2<\cdots<d_k$. Se tiene que $d_1$ es el elemento mínimo de $D$. Sin embargo, $f(1)\leq f(i)$ y $g(1)\leq g(i),\:\forall \: i=1,\ldots,k$ y como $f$ y $g$ son suprayectivas, entonces $f(1)\leq d_1$ y $g(1)\leq d_1$, además $d_1\leq f(1)$ y $d_1\leq g(1)$ por ser $d_1$ el elemento mínimo de $D$. Luego $f(1)=d_1=g(1)$.
\bigskip

%Ahora, tenemos que $d_2$ es el elemento mínimo de $D \setminus \{d_1\}$. Notemos que $f(2),g(2)\in D \setminus \{d_1\}$. En efecto, si $f(2)=d_1$ o $g(2)=d_1$, entonces $f(2)=f(1)$ o $g(2)=g(1)$ y como ambas funciones son inyectivas, entonces $2=1$, lo cual es absurdo. En consecuencia $d_2\leq f(2)$ y $d_2\leq g(1)$. Por otro lado, como $d_2\in D$, deben existir $i_1,i_2\in \{1,\ldots,k\}$ tales que $f(i_1)=d_2$ y $g(i_2)=d_2$. Más aún, $2\leq i_1$ y $2\leq i_2$, pues en caso contrario se tendría $i_1=1$ o $i_2=1$ y por tanto $f(1)=d_2$ o $g(1)=d_2$, es decir, $d_1=d_2$, lo cual es falso por hipótesis. Luego $f(2)\leq f(i_1)=d_2$ y $g(2)\leq f(i_2)=d_2$ y por tanto $f(2)=d_2=g(2)$.
%\bigskip

Supóngase que $f(i)=d_i=g(i),\:\forall \: i=1,\ldots,n$ y $n+1\leq k$. Si $n+1=k$, como $f$ y $g$ son biyectivas, necesariamente $f(n+1)=d_{n+1}=g(n+1)$. Supóngase pues que $n+1<k$. Se tiene que $d_{n+1}$ es el elemento mínimo del conjunto $D \setminus \{1,\ldots,d_n\}$. Además, $f(n+1),g(n+1)\in D\setminus \{1,\ldots,d_n\}$. En efecto, pues si $f(n+1)=d_{i_1}$ o $g(n+1)=d_{i_2}$, para algunos $i_1,i_2\in \{1,\ldots,n\}$, entonces $f(n+1)=f(i_1)$ y $g(n+1)=g(i_2)$ por hipótesis de inducción y por inyectividad se tendría que $n+1=i_1\leq n$ o $n+1=i_2\leq n$, lo cual es absurdo. En consecuencia $f(n+1),g(n+1)\in D \setminus \{1,\ldots,d_n\}$ y por tanto $d_{n+1}\leq f(n+1)$ y $d_{n+1}\leq g(n+1)$. 
\bigskip

Por otra parte, se tiene por suprayectividad que existen $j_1,j_2\in \{1,\ldots,k\}$ tales que $f(j_1)=d_{n+1}$ y $g(j_2)=d_{n+1}$, más aún, $n+1\leq j_1$ y $n+1\leq j_2$, pues en caso contrario se tendría que $j_1<n$ o $j_2<n$, es decir, $f(j_1)<f(n)$ o $g(j_2)<g(n)$, es decir, $d_{n+1}<d_n$, lo que contradice la hipótesis. Luego $f(n+1)\leq f(j_1)=d_{n+1}$ y $g(n+1)\leq g(j_2)=d_{n+1}$. Se sigue finalmente que $f(n+1)=d_{n+1}=g(n+1)$.
\end{proof}

\begin{lemma}\label{lemma:div1}
Si $n\in\mathbb{N}$ y $d_1=1<d_2<\cdots<d_{k-1}<d_k=n$ son todos los divisores positivos de $n$, entonces $d_i d_{k+1-i}=n,\:\forall \: i=1,\ldots,k$.
\end{lemma}
\begin{proof}
Sea $D=\{d_1,\ldots,d_k\}$ y consideremos las funciones $f: \{1,\ldots,k\} \longrightarrow D$ definida como $f(i)=d_i,\:\forall \: i=1,\ldots,k$ y $g: \{1,\ldots,k\} \longrightarrow D$ definida como $g(i)=n/d_{k+1-i},\:\forall \: i=1,\ldots,k$. Es fácil ver que $f$ y $g$ cumplen las condiciones del lema anterior y por tanto $f(i)=g(i),\:\forall \: i=1,\ldots,k$, es decir, $d_i d_{k+1-i}=n,\:\forall \: i=1,\ldots,k$.
\end{proof}

\begin{proposition}\label{prop:dir1}
Si $f$ y $g$ son funciones aritméticas, $n\in\mathbb{N}$ y $d_1<\cdots<d_k$ son todos los divisores positivos de $n$, entonces 
\begin{equation*}
	(f*g)(n)=\sum_{i=1}^{k} f(d_i)g(d_{k+1-i})=f(d_1)g(d_k)+\cdots+f(d_k)g(d_1).
\end{equation*}
\begin{proof}
Se sigue de la definición de $(f*g)(n)$ y del \Cref{lemma:div1}.
\end{proof}
\end{proposition}

\begin{definition}[Función identidad]\label{def:str2}
Se define a la función identidad $I$ como
\begin{equation*}
	I(n) =
		\begin{cases}
			\hfil 1 & \text{si} \: n=1 \\ 
			\hfil 0 & \text{si} \: n>1,
		\end{cases}
\end{equation*}
para cada $n\in\mathbb{N}$.
\end{definition}

La siguiente proposición muestra que la función $I$ actúa como la identidad bajo la convolución de Dirichlet, entre otras propiedades algebraicas.

\begin{proposition}
Si $f,g$ y $h$ son funciones aritméticas, entonces se verifica lo siguiente:

\begin{enumerate}[label=\textnormal{(\roman*)}]
	\item $(f*g)*h=f*(g*h)$
	\item $f*I=I*f=f$
	\item $f*(g+h)=(f*g)+(f*h)$
	\item $f*g=g*f$
\end{enumerate}
\end{proposition}
\begin{proof}
Sea $n\in\mathbb{N}$, sean $d_1=1<d_2<\cdots<d_k=n$ todos los divisores positivos de $n$ y para cada $i=1,\ldots,k$ sean $c_{i,1}<c_{i,2}<\cdots<c_{i,m_i}$ los divisores positivos de $d_i$.
\bigskip

({\scshape \romannumeral 1}) Se tiene
\begin{equation}\label{eqn:sum1}
	((f*g)*h)(n)=\sum_{i=1}^{k} \sum_{j=1}^{m_i} f(c_{i,j})g(c_{i,m_i+1-j})h(d_{k+1-i})
\end{equation}
y 
\begin{equation}\label{eqn:sum2}
	(f*(g*h))(n)=\sum_{i=1}^{k} \sum_{j=1}^{m_{k+1-i}} f(d_i)g(c_{m_{k+1-i},j})h(c_{m_{k+1-i},m_{k+1-i}+1-j}).
\end{equation}

Defínanse los conjuntos

\begin{align*}
	& \mathcal{C} = \left\{f(c_{i,j})g(c_{i,m_i+1-j})h(d_{k+1-i}) \: \mid i=1,\ldots,k,j=1,\ldots,m_i \: \right\} \\
	& \\
	& \mathcal{D} = \left\{f(d_i)g(c_{m_{k+1-i},j})h(c_{m_{k+1-i},m_{k+1-i}+1-j}) \: \mid i=1,\ldots,k,j=1,\ldots,m_i \: \right\},
\end{align*}

y $\mathcal{E} = \left\{f(a)f(b)f(c) \mid a,b,c\in\mathbb{N} \textrm{ y } a b c=n\right\}$. Se tiene que $\mathcal{C}=\mathcal{E}$ y $\mathcal{D}=\mathcal{E}$.
\bigskip

En efecto, si $f(c_{i,j})g(c_{i,m_i+1-j})h(d_{k+1-i})$, entonces $c_{i,j}c_{i,m_i+1-j}d_{k+1-i}=d_i d_{k+1-i}=n$, aplicando dos veces el \Cref{lemma:div1}. Recíprocamente, si $a,b,c\in\mathbb{N}$ son tales que $a b c = n$, entonces $c \mid n$, por tanto $c=d_j$, para algún $j=1,\ldots,k$, es decir, $c=d_{k+1-i}$ para $i=k+1-j$ con $i=1,\ldots,k$. Notemos entonces que por el \Cref{lemma:div1}, necesariamente se debe tener $a b=d_i$, por lo que $a=c_{i,j}$, para algún $j=1,\ldots,m_i$ y aplicando el lema de nuevo se debe tener que $b=c_{i,m_i+1-j}$. En consecuencia $f(a)g(b)h(c)=f(c_{i,j})g(c_{i,m_i+1-j})h(d_{k+1-i})\in \mathcal{C}$. Se sigue pues que $\mathcal{C}=\mathcal{E}$. Similarmente se demuestra que $\mathcal{D}=\mathcal{E}$.
\bigskip

Se tiene pues que $\mathcal{C}=\mathcal{D}$ y como las sumas \eqref{eqn:sum1} y \eqref{eqn:sum2} se extienden sobre los conjuntos $\mathcal{C}$ y $\mathcal{D}$, entonces deben coincidir, es decir, $((f*g)*h)(n)=(f*(g*h))(n)$.
\bigskip

({\scshape \romannumeral 2}) Como $1<d_i,\:\forall \: i=2,\ldots,k$, entonces $I(d_i)=0,\:\forall \: i=2,\ldots,k$, luego por la \Cref{prop:dir1} se tiene que

\begin{align*}
	(f*I)(n) &= \sum_{i=1}^{k} f(d_i)I(d_{k+1-i})=f(d_1)I(d_k)+\cdots+f(d_k)I(d_1) \\
			 &= f(d_k)I(d_1)=f(n)I(1)=f(n)\cdot 1=f(n)
\end{align*}
Y
\begin{align*}
	(I*f)(n) &= \sum_{i=1}^{k} I(d_i)f(d_{k+1-i})=I(d_1)f(d_k)+\cdots+I(d_k)f(d_1) \\
			 &= I(d_1)f(d_k)=I(n)f(1)=1\cdot f(n)=f(n)
\end{align*}
\bigskip

({\scshape \romannumeral 3}) Se tiene
\begin{align*}
	(f*(g+ & h)) (n) = \sum_{i=1}^{k} f(d_i)(g+h)(d_{k+1-i}) = \sum_{i=1}^{k} f(d_i)[g(d_{k+1-i})+h(d_{k+1-i})] \\
		   & = \sum_{i=1}^{k} f(d_i)g(d_{k+1-i})+\sum_{i=1}^{k} f(d_i)h(d_{k+1-i}) = (f*g)(n)+(f*h)(n).
\end{align*}
\bigskip

({\scshape \romannumeral 4}) La conmutatividad de la convolución de Dirichlet es clara, pues 
\begin{equation*}
	(f*g)(n) = \sum_{i=1}^{k} f(d_i)g(d_{k+1-i}) = \sum_{i=1}^{k} g(d_i)f(d_{k+1-i})=(g*f)(n).
\end{equation*}
\end{proof}

Se puede definir también la bien conocida multiplicación por escalares en el conjunto de funciones aritméticas, análoga a la del espacio euclideo $\mathbb{R}^n$.

\begin{definition}
Dados $c \in \mathbb{C}$ y $f \in \mathcal{A}$, se define $c f \in \mathcal{A}$ como la función
\begin{align*}
    c f : \mathbb{N} & \longrightarrow \mathbb{C} \\
    n & \longmapsto c f(n).
\end{align*}
\end{definition}

\begin{proposition}
El grupo abeliano $(\mathcal{A}, +)$ junto con la multiplicación por escalares definida anteriormente constituyen un espacio vectorial sobre el campo $\mathbb{C}$, donde el elemento neutro aditivo es la función $\mathbf{0}$. De ahora en adelante, este espacio vectorial será llamado simplemente el espacio de las funciones aritméticas.
\end{proposition}

\begin{proposition}
Si $c \in \mathbb{C}$ y $f, g \in \mathcal{A}$, entonces $c(f*g)=(c f)*g=f*(c g)$.
\end{proposition}
\begin{proof}
Se tiene que para cualquier $n \in \mathbb{N}$,
\begin{equation*}
    (c(f*g))(n) = c(f*g)(n) = c \sum_{d \mid n} f(n) g \left( \frac{n}{d} \right) = \sum_{d \mid n} (c f(n)) g \left( \frac{n}{d} \right) = ((c f)*g)(n).
\end{equation*}
Luego $c(f*g)=(c f)*g$. Las demás igualdades se prueban similarmente.
\end{proof}

\begin{corollary}\label{cor:est1}
El anillo $(\mathcal{A},+,*)$ es un álgebra conmutativa con identidad sobre el campo $\mathbb{C}$.
\end{corollary}

%%% Una norma
\subsection{Una norma para funciones aritméticas}

\begin{definition}
Sea $\mathcal{A}$ el conjunto de todas las funciones aritméticas. Se define
\begin{align*}
    \mathcal{N} : \mathcal{A} & \longrightarrow \mathbb{N} \cup \{0\} \\
	   f & \longmapsto \mathcal{N}(f)=
	   \begin{cases}
           \hfil 0 & \textnormal{si } f=\mathbf{0} \\ \hfil
           \min \left\{n : f(n)\neq 0\right\} & \textnormal{si } f \neq \mathbf{0}.
       \end{cases}
\end{align*}
\end{definition}

\begin{proposition}
La función $\mathcal{N}$ definida anteriormente tiene las siguientes propiedades:
\begin{enumerate}[label=\textnormal{(\roman*)}]
    \item $\mathcal{N}(f)=0 \iff f=0, \forall f \in \mathcal{A}$.
    \item $\mathcal{N}(f*g)=\mathcal{N}(f)\mathcal{N}(g), \forall f,g \in \mathcal{A}$.
    \item $\min \{\mathcal{N}(f),\mathcal{N}(g)\}\leq \mathcal{N}(f+g), \forall f,g \in \mathcal{A}$. 
    \item Si $\mathcal{N}(f) \ne \mathcal{N}(g)$ entonces $\mathcal{N}(f+g)=\min \{ \mathcal{N}(f),\mathcal{N}(g) \}$.
\end{enumerate}
\end{proposition}
\begin{proof}
({\scshape \romannumeral 1}) Si $f=0$ por definición se tiene que $\mathcal{N}(f)=0$. Si $f \neq 0$, entonces $\min \left\{n : f(n) \neq 0\right\}\neq 0$, i.e. $\mathcal{N}(f)\geq 1\neq 0$. Por tanto $\mathcal{N}(f)=0$ implica que $f=0$.
\bigskip

({\scshape \romannumeral 2}) Si $f=0$ o $g=0$ entonces $\mathcal{N}(f)=0$ o $\mathcal{N}(g)=0$. Además $(f*g)(n)=\sum_{d \mid n} f(n)g(n/d)=0, \:\forall\: n \in\mathbb{N}$, es decir, $\mathcal{N}(f*g)=0=\mathcal{N}(f)\mathcal{N}(g)$. Supongamos pues que $f \neq 0$ y $g \neq 0$. Sean $a=\mathcal{N}(f)$ y $b=\mathcal{N}(g)$. Afirmamos que $a b=\min \left\{n : (f*g)(n) \ne 0\right\}=m$.
\bigskip

En efecto, se tiene
\begin{align*}
	(f*g)(a b) &= \sum_{d \mid a b} f(d)g(a b/d) \\
			   &= \sum_{\substack{d \mid a b \\ a \leq d}} f(d)g(a b/d), \textnormal{ pues }f(d)=0, \:\forall\: d<a \\
			   &= \sum_{\substack{d \mid a b \\ a \leq d \\ a b/d \leq b}}, \textnormal{ pues }a \leq d \implies a b/d \leq b \\
			   &= \sum_{a=d} f(d)g(a b/d), \textnormal{ pues }g(d)=0, \:\forall\: d<b \\
			   &= f(a)g(b)\neq 0.
\end{align*}
Luego $m \leq a b$ por elección de $m$. Si $m<a b$ entonces
\begin{equation*}
	(f * g)(m) = \sum_{d \mid m} f(d) g(m/d) = \sum_{\substack{d \mid m \\ b \le m/d}} f(d) g(m/d) = \sum_{\substack{d \mid m \\ d<a}} f(d) g(m/d) = 0,
\end{equation*}
pues $b \le m/d$ implica que $d<a$ y $f(d)=0$. Pero esto contradice la elección de $m$. Por tanto, $m=a b$.

\bigskip

({\scshape \romannumeral 3}) Sin pérdida de generalidad se puede suponer que $a \le b$, de tal manera que $\min \{ a,b \} = a$. Si $n<a$ entonces $(f+g)(n) = f(n) + g(n) = 0$, por lo que
\begin{equation*}
    \min \{ \mathcal{N}(f),\mathcal{N}(g) \}=a \le \min \{ n \std (f+g)(n) \ne 0 \}=\mathcal{N}(f+g).
\end{equation*}

({\scshape \romannumeral 4}) Supóngase de nuevo sin pérdida de generaliad que $a<b$. Entonces
\begin{equation*}
	(f+g)(a) = f(a) + f(a) = f(a) + 0 = f(a) \ne 0,
\end{equation*}
por tanto, $N(f+g) \le a = \min \{ a,b \}=\min \{ \mathcal{N}(f),\mathcal{N}(g) \}$. El resultado se sigue ahora del punto ({\scshape \romannumeral 3}).

\end{proof}
\begin{theorem}
$\mathcal{A}$ es un dominio entero. 
\end{theorem}
\begin{proof}
Si $f,g \in \mathcal{A}$ y $f*g=0$ entonces $\mathcal{N}(f*g)=0 \implies \mathcal{N}(f)\mathcal{N}(g)=0 \implies \mathcal{N}(f)=0$ o $\mathcal{N}(g)=0 \implies f=0$ o $g=0$.
\end{proof}

\begin{remark}
Como ocurre en cualquier anillo con identidad, el conjunto de elementos invertibles forma un grupo respecto a la operación de multiplicación, en este caso, respecto a la convolución de Dirichlet. Este grupo se denotará $(\mathcal{A}^*,*)$ o simplemente como $\mathcal{A}^*$ cuando no haya riesgo de confusión.
\end{remark}

\begin{proposition}\label{prop:str1}
$f \in \mathcal{A}^*$ si y sólo si $\mathcal{N}(f)=1$.
\end{proposition}
\begin{proof}
Si $f(1) \ne 0$, defínase
\begin{equation}\label{eq:inv1}
    \begin{split}
	f^{-1}(1) &= \frac{1}{f(1)} \\
	f^{-1}(n) &= -\frac{1}{f(1)} \sum_{\substack{d \mid n \\ d \ne n}} f \left( \frac{n}{d} \right) f^{-1}(d), \hspace{0.2cm} n>1.
\end{split}
\end{equation}

Entonces la ecuación recursiva \eqref{eq:inv1} define a $f^{-1}$ de tal forma que $f*f^{-1} = I$, pues $f(1)f^{-1}(1)=1$ y si $n>1$ entonces
\begin{align*}
	(f*f^{-1})(n) = \sum_{d \mid n} f(d)f^{-1}\left( \frac{n}{d} \right) &= f(1)f^{-1}(n) + \sum_{\substack{d \mid n \\ d \ne n }} f(d)f^{-1}\left( \frac{n}{d} \right) \\
	&= f(1)f^{-1}(n)-f(1)f^{-1}(n) = 0
\end{align*}
es decir, $f*f^{-1}=I$.
\bigskip

Si se supone ahora que $f$ es invertible, entonces, en particular, se tiene que $(f*f^{-1})(1)=1$,  y por tanto $f(1) \ne 0$, es decir, $\mathcal{N}(f)=1$.
\end{proof}

\begin{proposition}
Si $\mathcal{N}(f)=p$ para algún número primo $p$, entonces $f$ es irreducible en $\mathcal{A}$.
\end{proposition}
\begin{proof}
Como $p \ne 0$ y $p \ne 1$, entonces $f$ no es cero ni es una unidad. Además, si $f=g*h$ para algunas funciones $g,h \in \mathcal{A}$, entonces  $g,h \ne 0$,  pues en caso contrario $f=0$ y en consecuencia $\mathcal{N}(f)=0 \ne p$, así que $\mathcal{N}(f)$ y $\mathcal{N}(h)$ son enteros positivos. Luego $\mathcal{N}(f)=\mathcal{N}(g*h)=\mathcal{N}(g)\mathcal{N}(h)=p$ y como $p$ es primo, entonces $\mathcal{N}(g)=1$ o bien $\mathcal{N}(h)=1$, es decir, $g$ o $h$ es unidad. Como $g$ y $h$ fueron arbitrarios, entonces $f$ debe ser irreducible en $\mathcal{A}$.
\end{proof}

\begin{theorem}\label{thm:chain1}
No puede existir una sucesión $\{ f_i \}_{i \in \mathbb{N}}$ de funciones aritméticas con la propiedad de que $f_1 \ne 0$ y $f_i=f_{i+1}*g_{i+1}$ y $g_{i+1}$ no es unidad, para cada $i \in \mathbb{N}$.
\end{theorem}
\begin{proof}
Asuma que dicho conjunto existe. Nótese que $\mathcal{N}(f_1) = \mathcal{N}(f_2) \mathcal{N}(g_2) > \mathcal{N}(f_2)$, pues $g_2$ no es unidad. De hecho se tiene
\begin{equation*}
    \mathcal{N}(f_1) > \mathcal{N}(f_2) > \cdots > \mathcal{N}(f_i), \forall i \in \mathbb{N}.
\end{equation*}
Más aún, como cada uno de éstos números son enteros, entonces
\begin{equation*}
    \mathcal{N}(f_1) \ge \mathcal{N}(f_2) + 1 \ge \cdots \ge \mathcal{N}(f_{i+1}) + i, \forall i \in \mathbb{N}
\end{equation*}
En particular, como $\mathcal{N}(f_1) \ge 1$ por ser $f_1 \ne \mathbf{0}$, se tiene que $\mathcal{N}(f_1) \ge \mathcal{N}\left(f_{\mathcal{N}(f_1)+1}\right) + \mathcal{N}(f_1)$, luego $0 \ge \mathcal{N}\left(f_{\mathcal{N}(f_1)+1}\right)$ y por tanto $\mathcal{N}(f_{\mathcal{N}(f_1)+1}) = \mathbf{0}$, en consecuencia $f_{\mathcal{N}(f_1)+1} = \mathbf{0}$, lo cual no es posible, pues de ser así $f_1 = f_{\mathcal{N}(f_1)+1} \cdot g_{\mathcal{N}(f_1)+1} \cdots g_2 = \mathbf{0}$, contradiciendo la hipótesis.
\end{proof}

El teorema anterior permite probar que cualquier elemento no cero y no unidad de $\mathcal{A}$ se puede expresar como producto finito de elementos irreducibles de $A$.

\begin{proposition}
Si $f \in \mathcal{A}\setminus (\mathcal{A}^* \cup \{ \mathbf{0} \})$, entonces $f$ es producto finito de elementos irreducibles en $\mathcal{A}$.
\end{proposition}
\begin{proof}
Como $f \ne \mathbf{0}$, en lo que sigue de esta demostración se debe tener que todas las funciones involucradas son distintas de cero. Se probará primero que $f$ tiene un factor irreducible. En efecto, si $f$ es irreducible, entonces no hay nada que probar. Supóngase que este no es el caso y por tanto $f=f_1*g_1$, donde $f_1$ y $g_1$ no son unidades. Si $f_1$ es irreducible hemos concluido. En caso contrario, se tiene $f_1=f_2*g_2$, donde $f_2$ y $g_2$ no son unidades. De manera inductiva se tiene una sucesión de funciones $\{ f_i \}_{i \in \mathbb{N}}$ tal que $f_i=f_{i+1}*g_{i+1}$, donde $g_{i+1}$ no es unidad para cada $i \in \mathbb{N}$, lo cual es imposible por el teorema anterior. En consecuencia, este proceso debe terminar y debe existir $M \in \mathbb{N}$ tal que $f_M=f_{M+1}*g_{M+1}$, donde $f_{M+1}$ es irreducible y $f_{M+1} \mid f$.
\bigskip

Se probará ahora el resultado principal. Escribiendo $f_{M+1}=p_1$, se tiene que $f=p_1*q_1$, con $p_1$ irreducible. Si $q_1$ es una unidad, entonces $f$ es irreducible y ya terminamos. Si $q_1$ no es unidad, nuevamente, $q_1$ debe tener un factor irreducible, es decir, $q_1=p_2*q_2$, donde $p_2$ es irreducible, y por tanto no es unidad. Si $q_2$ es unidad, entonces $q_1$ es irreducible y $f=p_1*q_1$ es la factorización buscada. Si este proceso nunca terminara, de forma inductiva se tendría una sucesión $\{ q_i \}_{i \in \mathbb{N}}$ tal que $q_i=p_{i+1}*q_{i+1}$, con $p_i$ no unidad, para cada $i \in \mathbb{N}$, lo que contradice de nuevo el Teorema \ref{thm:chain1}. En consecuencia, el proceso eventualmente termina y por tanto existe $N \in \mathbb{N}$ tal que $q_N=p_{N+1}*q_{N+1}$, donde $q_{N+1}$ es unidad y $p_{N+1}$ es irreducible. Luego
\begin{equation*}
    f=p_1*p_2*\cdots*p_N*q_N,
\end{equation*}
donde $p_1,\ldots,p_{N}$ y $q_N$ son irreducibles.
\end{proof}

Habiendo llegado tan lejos, se puede conjeturar que el dominio $\mathcal{A}$ es un dominio de factorización única. Esta sospecha es, de manera sorprendente, acertada. Sin embargo, la demostración de este hecho no es tan sencilla como la de la proposición anterior.

\begin{theorem}
$\mathcal{A}$ es un dominio de factorización única.
\end{theorem}
\begin{proof}
El hecho de que toda función aritmética se puede escribir como producto de funciones aritméticas irreducibles ha quedado en evidencia en la proposición anterior. Una demostración de la unicidad de dicha factorización se puede encontrar en \cite[18, p. 985]{Ca1959}. Ahí se prueba que el anillo de series de potencias formales en un conjunto numerable de variables $\{ x_1,x_2,\ldots \}$ es un dominio de factorización única. El resultado se sigue entonces del hecho de que este anillo es isomorfo al anillo de funciones aritméticas mediante el isomorfismo
\begin{align*}
    P : \mathcal{A} & \longrightarrow \mathbb{C}[[x_1,x_2,\ldots]] \\
    P(f) & \longmapsto \sum_{n \in \mathbb{N}} f(n) x_1^{\alpha_1}\cdots x_{\nu}^{\alpha_{\nu}},
\end{align*}
donde $n=p_1^{\alpha_1}\cdots p_{\nu}^{\alpha_{\nu}}$ es la factorización en primos de $n$. Se tiene que $P(f+g)=P(f)+P(g)$ y $P(f*g)=P(f)P(g)$, donde la multiplicación de dos series de este tipo se realiza agrupando términos ``semejantes'', es decir, monomios iguales. Otra demostración de este hecho se puede encontrar en \cite{Nish1}. Ambas demostraciones utilizan el hecho de que los anillos de series de potencias formales en un número finito de variables $\mathbb{C}[[x_1,\ldots,x_n]]$ son dominios de factorización única, para cada $n \in \mathbb{N}$.
\end{proof}

\begin{corollary}\label{cor:fac1}
Todo elemento irreducible en $\mathcal{A}$ es primo en $\mathcal{A}$.
\end{corollary}

Siendo $\mathcal{A}$ un dominio de factorización única, cabe preguntarse si también es un dominio de ideales principales. La siguiente proposición muestra que este no es el caso.

\begin{proposition}
$\mathcal{A}$ no es un dominio de ideales principales.
\end{proposition}
\begin{proof}
Considere $f=(0,1,0,\ldots)$ y $g=(0,0,1,0,\ldots)$. Se tiene que $\mathcal{N}(f)=2$ y $\mathcal{N}(g)=3$, ambos números primos. Afirmamos que $I$ es un máximo común divisor de $f$ y $g$. Claro que $I \mid f$ y $I \mid g$. Si $h \in \mathcal{A}$ es tal que $h \mid f$ y $h \mid g$, entonces $f=h k_1$ y $g=h k_2$, con $h,k_1,k_1 \in \mathcal{A}\setminus \{ \mathbf{0} \}$. Luego $2=\mathcal{N}(h)\mathcal{N}(k_1)<3=\mathcal{N}(h)\mathcal{N}(k_2)$, en consecuencia, $1 \le \mathcal{N}(k_1)<\mathcal{N}(k_2)$, así que necesariamente $\mathcal{N}(k_2)=3$ y $\mathcal{N}(h)=1$. Luego $h$ es unidad, es decir $h \mid I$. Esto prueba que $I$ es máximo común divisor de $f$ y $g$.
\bigskip

Si $\mathcal{A}$ fuera un dominio de ideales principales por \cite[\S III.3, Thm. 3.11.(ii), p. 140]{Hun1}, existirían $s,t \in \mathcal{A}$ tales que $I=f*s+g*t$, en particular, $1=I(1)=f(1)s(1)+g(1)t(1)=0$, lo cual es imposible.
\end{proof}

\begin{theorem}
$\mathcal{A}$ es un anillo local.
\end{theorem}
\begin{proof}
Por \cite[\S III.4, Thm. 4.13.(iii), p. 147]{Hun1}, basta probar que los elementos no invertibles de $\mathcal{A}$ forman un ideal de $\mathcal{A}$. En efecto, se tiene que $\mathbf{0} \in \mathcal{A}\setminus \mathcal{A}^*$. Si $f \in \mathcal{A}\setminus \mathcal{A}^*$ y $g \in \mathcal{A}$, entonces $f(1)=0$, en consecuencia $(f*g)(1)=f(1)g(1)=0$, es decir, $f*g \in \mathcal{A}\setminus \mathcal{A}^*$. Además, si $h \in \mathcal{A}\setminus \mathcal{A}^*$, entonces $h(1)=0$ y por tanto $f(1)-h(1)=0$, es decir $f-h \in \mathcal{A}\setminus \mathcal{A}^*$. Esto prueba que $\mathcal{A}\setminus \mathcal{A}^*$ es un ideal de $\mathcal{A}$.
\end{proof}

\begin{proposition}
Si $f \in \mathcal{A}$ es tal que $f(1) = 0$ y $f(p) \ne 0$ para alǵun primo $p$, entonces $f$ es irreducible en $\mathcal{A}$.
\end{proposition}
\begin{proof}
Por hipótesis $f$ no es cero y no es unidad. Supongamos que $f=g*h$. Si $g$ y $h$ no fueran unidades se tendría que $g(1)=0$ y $h(1)=0$, por tanto, $f(p)=(g*h)(p)=g(1)h(p)+g(p)h(1)=0$, lo que contradice la hipótesis. En consecuencia alguna de las funciones $g$ o $h$ es unidad.
\end{proof}

Un colorario de la proposición anterior y el \Cref{cor:fac1} es el siguiente.

\begin{corollary}
Si $f \in \mathcal{A}$ es tal que $f(1) = 0$ y $f(p) \ne 0$ para algún primo $p$, entonces $f$ es un elemento primo de $\mathcal{A}$.
\end{corollary}

El \Cref{thm:chain1} afirma que el anillo $\mathcal{A}$ satisface una forma débil de condición de la cadena ascendente, ¿satisfará también la condición de la cadena descendente? Es decir, ¿es $\mathcal{A}$ artiniano? El siguiente contraejemplo desacredita esta observación.

\begin{proposition}
$\mathcal{A}$ no es un anillo artiniano.
\end{proposition}
\begin{proof}
Para cada $n \in \mathbb{N}$ defínase $I_n = \{ f \in \mathcal{A} \std \mathcal{N}(f) \ge n \} \cup \{ \mathbf{0} \}$. Se tiene lo siguiente:\begin{enumerate}[label=\textnormal{(\arabic*)}]
\item $I_n$ es un ideal de $\mathcal{A}$, para cada $n$. En efecto, por definición se tiene $I_n \ne \emptyset$. Si $f,g \in I_n$ entonces $\mathcal{N}(f) \ge n$ y $\mathcal{N}(g) \ge n$, luego $\mathcal{N}(f-g) \ge \min \{ \mathcal{N}(f), \mathcal{N}(-g) \} = \min \{ \mathcal{N}(f),\mathcal{N}(g) \} \ge n$, luego $f-g \in I_n$.
\bigskip

Además, si $h \in \mathcal{A}$, se tienen dos casos. Si $h=\mathbf{0}$, entonces $f*h=0 \in I_n$. Si $h \ne \mathbf{0}$, entonces $\mathcal{N}(f) \ge 1$, de tal manera que $\mathcal{N}(f*g)=\mathcal{N}(f)\mathcal{N}(h) \ge \mathcal{N}(f) \ge n$, es decir, $f*h \in I_n$. Esto prueba que $I_n$ es un ideal de $\mathcal{A}$.
\item $I_{n+1} \subset I_n$, para cada $n \in \mathbb{N}$, pues $\mathcal{N}(f) \ge n+1$ implica que $\mathcal{N}(f) \ge n$.
\item $I_n \not \subset I_{n+1}$, para cada $n \in \mathbb{N}$, pues considere $f \in \mathcal{A}$ definida como
\begin{equation*}
    f(k) = \begin{cases}
        \hfil 1 & \text{si } k=n \\
        \hfil 0 & \text{en otro caso.}
    \end{cases}
\end{equation*}
Entonces $\mathcal{N}(f) = n < n+1$, es decir, $f \in I_n$, pero $f \notin I_{n+1}$.
\end{enumerate}
Se tiene pues una cadena descendente de ideales distintos entre sí de $\mathcal{A}$, luego $\mathcal{A}$ no es artiniano.
\end{proof}

\subsection{Funciones multiplicativas}

\begin{definition}[Función multiplicativa]\label{def:mul1}
Se dice que una función aritmética $f$ es \textbf{multiplicativa} si no es idénticamente cero y para todo $m,n \in \mathbb{N}$, $(m,n)=1$ implica que $f(m n)=f(m)f(n)$.
\end{definition}

\begin{remark}
Se denota al conjunto de funciones multiplicativas como $\mathcal{M}$. En general si $f$ y $g$ son funciones multiplicativas entonces $f-g$ no es necesariamente una función multiplicativa, sin embargo, $f*g$ sí lo es.
\end{remark}

\begin{lemma}
Si $(a,b)=1$ y $d \in \mathbb{N}$, entonces $(a b,d)=(a,d)(b,d)$.
\end{lemma}
\begin{proof}
Escríbanse $(a,d)=a x+d y$ y $(b,d)=b s+d t$, para algunos $x,y,s,t \in \mathbb{Z}$. Entonces
\begin{equation*}
	(a,d)(b,d) = a b x s + a x d t + d y b s + d y d t = a b (xs) + d(a x t + y b s + y d t),
\end{equation*}
por tanto, $(a b,d) \mid (a,d)(b,d)$. 
\bigskip

Por otro lado, escríbase $1=a z + b w$, para algunos $z,w \in \mathbb{Z}$. Entonces $d=d a z + d b w$. Además, como $a=(a,d)m$, $b=(b,d)n$, $d=(b,d)p$ y $d=(a,d)q$ para algunos $m,n,p,q \in \mathbb{Z}$, entonces
\begin{equation*}
    d = (a,d)(b,d) (p m z + q n w),
\end{equation*}
es decir, $(a,d)(b,d) \mid d$. Dado que $a b=(a,d)(b,d)m n$, entonces $(a,d)(b,d) \mid a b$ y en consecuencia $(a,d)(b,d) \mid (a b,d)$. Se sigue que $(a b,d)=(a,d)(b,d)$.
\end{proof}

\begin{lemma}\label{lem:div1}
Si $(a,b)=1$, $a_1,\ldots,a_l$ son todos los divisores positivos de $a$ y $b_1,\ldots,b_m$ son todos los divisores positivos de $b$, entonces $\{ d>0 \std d \mid a b \}=\{ a_i b_j \std i=1,\ldots,l, \hspace{0.2cm} j=1,\ldots,m \}$.
\end{lemma}
\begin{proof}
Si $a_i$, $b_j$ son divisores de $a$ y $b$, respectivamente, entonces existen $s,t \in \mathbb{Z}$ tales que $a=a_i s$ y $b=b_j t$, luego $a b=a_i b_j s t$, es decir, $a_i b_j \mid a b$. Recíprocamente, si $d$ es un divisor de $a b$, entonces $(a b,d)=d$, pero por el lema anterior $(a b,d)=(a,d)(b,d)$, luego $d=(a,d)(b,d)$, donde $(a,d)$ es un divisor positivo de $a$ y $(b,d)$ es un divisor positivo de $b$.
\end{proof}

\begin{theorem}\label{thm:sb1}
$(\mathcal{M},*)$ es un subgrupo de $(\mathcal{A}^*,*)$.
\end{theorem}
\begin{proof}
Si $f \in \mathcal{M}$, entonces $f \ne \mathbf{0}$ y existe $N \in \mathbb{N}$ tal que $f(N) \ne 0$, luego $f(N)=f(1 \cdot N)=f(1)f(N)$ y en consecuencia $1=f(1)$, es decir, $f \in \mathcal{A}^*$. Esto prueba que $\mathcal{M} \subset \mathcal{A}^*$.
\bigskip

Claro que el conjunto $\mathcal{M}$ es no vacío, pues $I \in \mathcal{M}$. Veamos que la operación $*$ es cerrada en $\mathcal{M}$. Sean $f,g$ funciones multiplicativas, sean $a,b \in \mathbb{N}$ tales que $(a,b)=1$ y sean $a_1,\ldots,a_l$ y $b_1,\ldots,b_m$ todos los divisores positivos de $a$ y $b$, respectivamente. Entonces $(a_i,b_j)=1$, para cada $i=1,\ldots,l$ y para cada $j=1,\ldots,m$, luego
\begin{align*}
	(f*g)(a)(f*g)(b) &= \left[ \sum_{i=1}^{l} f(a_i)g\left( \frac{a}{a_i} \right) \right]\left[ \sum_{j=1}^{m} f(b_j)g \left( \frac{b}{b_i} \right) \right] \\
					 &= \sum_{i=1}^{l} \sum_{j=1}^{m} f(a_i)g \left( \frac{a}{a_i} \right)f(b_i)g \left( \frac{b}{b_i} \right) \\
					 &= \sum_{i=1}^{l} \sum_{j=1}^{m} f(a_i b_j)g \left( \frac{a b}{a_i b_j} \right) \\
					 &= \sum_{d \mid a b}f(d)g \left( \frac{a b}{d} \right) = (f*g)(a b)
\end{align*}
por el \Cref{lem:div1}.
\bigskip

\thispagestyle{easter4}

Como ya se probó al inicio de esta demostración, si $f$ es multiplicativa entonces $f(1)=1$, por lo que existe $f^{-1}$. Veamos que $f^{-1} \in \mathcal{M}$. Para esto construiremos, a partir de $f$, una función multiplicativa $g$ con la propiedad de que $f*g=I$, con lo que quedará demostrado que $f^{-1}$ es multiplicativa por la unicidad de la inversa. Se procede definiendo a $g$ de forma gradual:

\begin{enumerate}[label=\textnormal{(\arabic*)},ref=\textnormal{\arabic*}]
\item $g(1)=1$.
\item Para cada primo $p$ se define $g(p)=-f(p)$. De tal manera que
	\begin{equation*}
		(f*g)(p) = \sum_{d \mid p} f(p)g \left( \frac{p}{d} \right) = f(1)g(p) + f(p)g(1) = -f(p) + f(p) = 0.
	\end{equation*}
\item \label{it:mul1} Para cada $a \in \mathbb{N}$ y para cada primo $p$ se define, recursivamente,
	\begin{equation*}
	    g(p^a)=-f(p)g(p^{a-1})-\cdots-f(p^a)g(1)
	\end{equation*}
	de tal manera que 
	\begin{align*}
		(f&*g)(p^a) = \sum_{d \mid p^a} f(d)g \left( \frac{p^a}{d} \right) = f(1)g(p^a)+f(p)g(p^{a-1})+\cdots+f(p^a)g(1) \\
				   &= -f(p)g(p^{a-1})-\cdots-f(p^a)g(1)+f(p)g(p^{a-1})+\cdots+f(p^a)g(1) = 0.
	\end{align*}
\item Se define
	\begin{equation*}
	    g \left( \prod p_i^{a_i} \right) = \prod g(p_i^{a_i}).
	\end{equation*}
	para cualquier producto finito de potencias de primos, con $p_i \ne p_j$ si $i \ne j$. La función $g$ ha quedado entonces definida para cualquier entero positivo.
\item $g$ es multiplicativa, pues si $a=p_1^{\alpha_1}\cdots p_m^{\alpha_m}$ y $b=q_1^{\beta_1}\cdots q_l^{\beta_l}$ son tales que $(a,b)=1$, entonces $p_i \ne q_j$, luego
	\begin{align*}
		g(a b) = g(p_1^{\alpha_1}\cdots p_m^{\alpha_m} q_1^{\beta_1}\cdots q_l^{\beta_l}) &= g(p_1^{\alpha_1})\cdots g(p_m^{\alpha_m})g(q_1^{\beta_1})\cdots g(q_l^{\beta_l}) \\
																				 &= g(p_1^{\alpha_1}\cdots p_m^{\alpha_m})g(q_1^{\beta_1}\cdots q_l^{\beta_l}) = g(a)g(b)
	\end{align*}
\item Como la operación $*$ es cerrada en $\mathcal{M}$, entonces $f*g$ es multiplicativa.
\item Si $n>1$ y $n=p_1^{\alpha_1}\cdots p_l^{\alpha_l}$ es su factorización en primos, entonces
	\begin{equation*}
		(f*g)(n) = (f*g)(p_1^{\alpha_1})\cdots (f*g)(p_l^{\alpha_l}) = 0
	\end{equation*}
	donde la primera igualdad se cumple por ser $f*g$ multiplicativa y la segunda por el inciso \eqref{it:mul1}. Además, $(f*g)(1)=f(1)g(1)=1$. En consecuencia $f*g=I$.
\item Se sigue que $g=f^{-1}$ y como $g$ es multiplicativa, entonces $f^{-1}$ también lo es.
\end{enumerate}
\end{proof}

\begin{corollary}
Si $f*g$ es multiplicativa y $g$ es multiplicativa, entonces $f$ también lo es.
\end{corollary}
\begin{proof}
Como $g$ es multiplicativa, entonces existe $g^{-1}$ y también es multiplicativa, luego $f=(f*g)*g^{-1}$ es multiplicativa por ser producto de funciones multiplicativas.
\end{proof}

\subsection{Isomorfismos entre grupos de funciones aritméticas}

Se denotará como $\mathcal{A}_\mathbb{R}$ al conjunto de funciones aritméticas real valuadas, es decir, $\mathcal{A}_{\mathbb{R}}=\{ f \in \mathcal{A} \std f(n) \in \mathbb{R}, \forall n \in \mathbb{N} \}$. Asimismo, se define $P=\{ f \in \mathcal{A} \std f(1)>0 \}$. Es fácil verificar que $(\mathcal{A}_{\mathbb{R}},+)$ y $(P,*)$ son subgrupos de $(\mathcal{A},+)$ y de $(\mathcal{A}^*,*)$, respectivamente. Más aún, estos grupos son isomorfos.

\begin{lemma}\label{prop:iso1}
$(\mathcal{A}_{\mathbb{R}},+) \cong (P,*)$.
\end{lemma}
\begin{proof}
El isomorfismo buscado es
\begin{align*}
    L : (P,*) & \longrightarrow (\mathcal{A}_{\mathbb{R}},+) \\
    f & \longmapsto L f
\end{align*}
donde $L f(1)=\log(1)$ y $L f(n)=\sum_{d \mid n} \log(d)f(d)f^{-1}(n/d)$ para $n>1$. Se tiene que $L$ es en efecto un homomorfismo, pues para $n=1$ se tiene
\begin{equation*}
    L (f*g)(1) = \log (f*g) (1) = \log(f(1)g(1)) = \log f(1) + \log g(1) = L f(1) + L g(1).
\end{equation*}
Para el caso $n>1$, nótese primero que para cualquier $n \in \mathbb{N}$,
\begin{align*}
	\log(n)(f*g)(n) &= \log(n) \sum_{d \mid n} f(d)g \left( \frac{n}{d} \right) \\
					&= \sum_{d \mid n} f(d) g \left( \frac{n}{d} \right) \left[ \log \frac{n}{d} + \log d \right] \\
					&= \sum_{d \mid n} f(d) g \left( \frac{n}{d} \right) \log \left( \frac{n}{d} \right) + \sum_{d \mid n} f(d) g \left( \frac{n}{d} \right) \log(d) \\
					&= (f*(\log \cdot g))(n)+((\log \cdot f)*g)(n),
\end{align*}
es decir, $\log \cdot (f*g) = f*(\log \cdot g)+(\log \cdot f)*g$. Multiplicando por $(f*g)^{-1}=f^{-1}*g^{-1}$ a ambos lados de la ecuación, se tiene que
\begin{equation*}
	(\log \cdot (f*g))*(f*g)^{-1} = (\log \cdot g)*g^{-1} + (\log \cdot f)*f^{-1},
\end{equation*}
es decir, $L(f*g)=L f+L g$ y en particular para $n>1$. Esto prueba que $L$ es un homomorfismo.
\bigskip

$L$ también es suprayectivo, pues si $f \in \mathcal{A}_{\mathbb{R}}$, defínase $g(1)=\exp(f(1))$. Entonces $L g(1)=\log g(1)=\log \exp(f(1))=f(1)$, pues $f(1) \in \mathbb{R}$. Además, como $g(1)>0$ existe $g^{-1}$ y se define recursivamente, para $n>1$,
\begin{equation*}
    g(n) = \frac{1}{\log(n) g^{-1}(1)} \left[ f(n) - \sum_{\substack{d \mid n \\ d \ne 1,n}} \log(d) g(d) g^{-1} \left( \frac{n}{d} \right)\right].
\end{equation*}
Esta ecuación implica que
\begin{align*}
	f(n) = g(n)\log(n)g^{-1}(1)+g(1)\log(1)g^{-1}(n)&+\sum_{\substack{d \mid n \\ d \ne 1,n}} \log(d)g(d)g^{-1} \left( \frac{n}{d} \right) \\
		 &= \sum_{d \mid n} \log(d)g(d)g^{-1} \left( \frac{n}{d} \right) = L g(n).
\end{align*}
En consecuencia, $L g(n)=f(n), \forall n \in \mathbb{N}$, es decir, $L g=f$.
\bigskip

Finalmente, se tiene que $L$ es inyectivo. En efecto, si $L(f)=L(g)$, entonces $L(f)-L(g)=0$, pero $-L g=L g^{-1}$ por ser $L$ un homomorfismo, luego $L f+L g^{-1}=L(f*g^{-1})=0$. Para $n=1$ esto implica que $\log(f*g^{-1}(1))=0$ y por tanto $(f*g^{-1})(1)=1$. Si $n=2$, entonces
\begin{equation*}
    L(f*g^{-1})(2) = \log(1)(f*g^{-1})(1)(f*g^{-1})^{-1}(2) + \log(2)(f*g^{-1})(2)(f*g^{-1})(1) = 0,
\end{equation*}
pero $\log(1)=0$, por tanto $\log(2)(f*g^{-1})(2)(f*g^{-1})(1)=0$ y dado que $(f*g^{-1})(1) \ne 0$, entonces $(f*g^{-1})(2)=0$. Supóngase que $(f*g^{-1})(d)=0$, para cada $1<d<n$. Entonces $L(f*g)=0$ implica que
\begin{equation*}
	\log(n)(f*g^{-1})(n)(f*g^{-1})(1) + \sum_{\substack{d \mid n \\ d \ne 1,n}} \log(d)\underbrace{(f*g^{-1})(d)}_{0}(f*g^{-1})^{-1} \left( \frac{n}{d} \right) = 0,
\end{equation*}
pues $\log(1)=0$, por tanto, $\log(n)(f*g^{-1})(n)(f*g^{-1})(1)=0$ y por tanto $(f*g^{-1})(n)=0$. Esto prueba que para cada $n>1$, $(f*g^{-1})=0$. Así pues, se tiene que $f*g^{-1}=I$, por tanto, $f=g$.
\end{proof}

Se denota $\mathcal{A}^\prime=\{ f \in \mathcal{A}_{\mathbb{R}} \std f(n) = 0,\forall n \ne p^\alpha, p \text{ primo y } \alpha \in \mathbb{N} \}$. La siguiente proposición es una caracterización de las funciones multiplicativas respecto al conjunto $\mathcal{A}^\prime$ y al isomorfismo $L$.

\begin{proposition}
$f \in \mathcal{M}$ si y sólo si $L f \in \mathcal{A}^\prime$.
\end{proposition}
\begin{proof}
Supóngase primero que $f$ es multiplicativa. Entonces $f(1)=1$, por tanto, $L f(1)=\log f(1)=\log 1 = 0$. Si $N>1$ no es potencia de ningún primo, entonces $N=m n$, con $(m,n)=1$ y $n,m>1$. Luego
\begin{align*}
	L f(N) = L f(m n) &= \sum_{d \mid m n} \log(d)f(d)f^{-1} \left( \frac{m n}{d} \right) \\
					  &= \sum_{d \mid m}\sum_{e \mid n} f(d)f(e)f^{-1}\left( \frac{m}{d} \right)f^{-1}\left( \frac{n}{e} \right)(\log(d)+\log(e)) \\
					  &= \sum_{d \mid m} \log(d)f(d)f^{-1}\left( \frac{m}{d} \right)\sum_{e \mid n} f(e)f^{-1}\left( \frac{n}{e} \right) \\
					  &+ \sum_{e \mid n} \log(e)f(e)f^{-1}\left( \frac{n}{e} \right)\sum_{d \mid m}f(d)f^{-1}\left( \frac{m}{e} \right) \\
					  &= L f(m)\sum_{e \mid n} f(e)f^{-1}\left( \frac{n}{e} \right)+L f(n) \sum_{d \mid m}f(d)f^{-1}\left( \frac{m}{d} \right) \\
					  &= L f(m)I(n) + L f(n)I(m) = 0,
\end{align*}
pues $m,n>1$. Luego $f \in \mathcal{A}^\prime$. 
\bigskip

Recíprocamente, supóngase que $Lf \in \mathcal{A}^\prime$. En particular se tilene que $Lf(1)=0$ y por tanto $f(1)=1$. Se definirá una función multiplicativa $g$ y se probará que coincide con $f$.
\begin{enumerate}[label=\textnormal{(\arabic*)},ref=\textnormal{\arabic*}]
    \item Se define $g(1)=1=f(1)$.
    \item Para cada primo, se define
        \begin{equation*}
            g(n) = \prod_{p \mid n} f(p^\nu),
        \end{equation*}
        donde $\nu := \max \{ \alpha \std p^\alpha \mid n \}$.
    \item $g$ es multiplicativa, pues $(m,n)=1$ implica que
        \begin{equation*}
            g(m n) = \prod_{p \mid m n} f(p^\nu) = \prod_{p \mid n} f(p^\nu) \prod_{p \mid n} f(p^\nu) = g(m) g(n).
        \end{equation*}
    \item \label{it:mul2} $g$ coincide con $f$ en todas las potencias de primos, pues si $q$ es un primo y $\alpha \in \mathbb{N}$,
        \begin{equation*}
            g(q^{\alpha}) = \prod_{p \mid q^{\alpha}} f(p^\nu) = f(q^{\alpha}).
        \end{equation*}
    \item $g^{-1}$ coincide con $f^{-1}$ en todas las potencias de primos, pues si $q$ es primo,
        \begin{equation*}
            g^{-1}(q) = - \sum_{\substack{d \mid q \\ d \ne q}} g \left( \frac{q}{d} \right) g^{-1}(d) = -g(q)g^{-1}(1) = -g(q) = -f(q) = f^{-1}(q),
        \end{equation*}
        por el \cref{it:mul2}. Además, de forma recursiva se tiene que
        \begin{align*}
            g^{-1}(q^\alpha) &= -[g(q^{\alpha-1})g^{-1}(1)+\cdots+g(q)g^{-1}(q^{\alpha-1})] \\
                             &= -[f(q^{\alpha-1})f^{-1}(1)+\cdots+f(q)f^{-1}(q^{\alpha-1})] = f^{-1}(q^\alpha),
        \end{align*}
        donde $g^{-1}$ coincide con $f^{-1}$ en $1,q,q^2,\ldots,q^{\alpha-1}$.
    \item El punto anterior implica que $Lf(q^\alpha)=Lg(q^\alpha)$ para todo primo $q$ y para todo $\alpha \in \mathbb{N}$, pues
        \begin{equation*}
            Lf(q^\alpha) = \sum_{d \mid p^\alpha} \log(d) f(d)f^{-1}\left( \frac{n}{d} \right) = \sum_{d \mid p^\alpha} \log(d) g(d)g^{-1}\left( \frac{n}{d} \right) = Lg(q^\alpha).
        \end{equation*}
        Además, como $g$ es multiplicativa, entonces $Lg(n)=0$ para todo $n$ no potencia de algún primo, por la primera parte de esta demostración. Luego, por hipótesis se tiene que $Lf(n)=0=Lg(n)$ para todo $n$ no potencia de algún primo, así que de hecho $Lf(n)=Lg(n)$ para todo $n$, es decir, $Lf=Lg$ y como la aplicación $L$ es inyectiva, entonces $f=g$. Luego $f$ es multiplicativa, pues $g$ lo es.
\end{enumerate}
\end{proof}

\begin{lemma}\label{prop:iso2}
$(\mathcal{M},*) \cong (\mathcal{A}^\prime,+)$.
\end{lemma}
\begin{proof}
Es fácil ver que $(\mathcal{A}^\prime,+)$ es un subgrupo de $(\mathcal{A}_{\mathbb{R}},+)$ y que $(\mathcal{M},*)$ es un subgrupo de $(P,*)$. Luego la restricción del homomorfismo a $L$ a $(\mathcal{M},*)$ sigue siendo un isomorfismo y su imagen es $(\mathcal{A}^\prime,+)$ por la proposición anterior.
\end{proof}

\begin{lemma}\label{prop:iso3}
$(\mathcal{A}_{\mathbb{R}},+) \cong (\mathcal{A}^\prime,+)$.
\end{lemma}
\begin{proof}
Sea
\begin{align*}
    \phi : (\mathcal{A}_{\mathbb{R}},+) & \longrightarrow (\mathcal{A}^\prime,+) \\
    f & \longmapsto F,
\end{align*}
donde $F$ es la función definida como $F(n)=f(p_n), \forall n \in \mathbb{N}$ y $p_n$ es el $n-$ésimo término en la sucesión de potencias de primos en orden ascendente.
\bigskip

Se tiene que $\phi$ es un homomorfismo, pues $\phi(f+g)(n)=(f+g)(p_n)=f(p_n)+g(p_n)=\phi(f)(n)+\phi(g)(n), \forall n \in \mathbb{N}$, luego $\phi(f+g)=F+G$. Se también tiene que $\phi$ es inyectivo, pues si $f,g \in (\mathcal{A}^\prime,+)$ son tales que $\phi(f)=\phi(g)$, entonces $f(p_n)=g(p_n), \forall n \in \mathbb{N}$, además $f(n)=g(n)=0$ si $n$ no es potencia de algún primo, de manera que $f(n)=g(n), \forall n \in \mathbb{N}$, es decir, $f=g$.
\bigskip

Finalmente se tiene que $\phi$ es suprayectivo, pues si $F \in (\mathcal{A}^\prime,+)$, defínase $f(p_n)=F(n), \forall n \in \mathbb{N}$ y $f(n)=0$ para toda $n$ no potencia de algún primo. Entonces $f \in (\mathcal{A}_{\mathbb{R}},+)$ y $\phi(f)(n)=f(p_n)=F(n), \forall n \in \mathbb{N}$, es decir, $\phi(f)=F$.
\end{proof}

\begin{lemma}\label{prop:iso4}
$(\mathcal{A}_{\mathbb{R}},+) \cong (\mathcal{A}_1,+)$, donde $\mathcal{A}_1 = \{ f \in \mathcal{A} \std f(1) \in \mathbb{R} \}$.
\end{lemma}
\begin{proof}
Es claro que $(\mathcal{A}_1,+)$ es un grupo aditivo. Defínase
\begin{align*}
    \psi : (\mathcal{A}_{\mathbb{R}},+) & \longrightarrow (\mathcal{A}_1,+) \\
    f & \longmapsto F
\end{align*}
donde $F$ es la función definida como $F(n)=f(2n-2)+i f(2n-1), \forall n>1$ y $F(1)=f(1)$. Se tiene que $\psi$ es un homomorfismo, pues $\psi(f+g)(1)=(f+g)(1)=f(1)+g(1)=\psi(f)(1)+\psi(g)(1)$, además,
\begin{align*}
    \psi(f+g)(n) &= (f+g)(2n-2)+i (f+g)(2n-1) \\
                 &= f(2n-2)+g(2n-2)+i f(2n-1)+i g(2n-1) \\
                 &= [f(2n-2)+i f(2n-1)] + [g(2n-2)+i g(2n-1)] \\
                 &= \psi(f)(n)+\psi(g)(n),
\end{align*}
luego $\psi(f+g)=\psi(f)+\psi(g)$.
\bigskip

El homomorfismo $\psi$ es también inyectivo, pues si $f,g \in (\mathcal{A}_{\mathbb{R}},+)$ son tales que $\psi(f)=\psi(g)$, entonces $f(n),g(n) \in \mathbb{R}, \forall n \in \mathbb{N}$ y además $f(2n-2)+i f(2n-1)=g(2n-2)+i g(2n-1)$, por tanto $f(2n-2)=g(2n-2)$ y $f(2n-1)=g(2n-1)$ y $f(1)=g(1)$, así que $f(n)=g(n), \forall n \in \mathbb{N}$, es decir $f=g$.
\bigskip

Finalmente, $\psi$ también es suprayectivo, pues dada $F \in (\mathcal{A}_1,+)$, se puede escribir $F=F_1+i F_2$, donde $F_1,F_2 \in \mathcal{A}_{\mathbb{R}}$. Defínase $g(1)=F(1)$ y 
{\everymath{\displaystyle}
    \begin{equation*}
        g(n) = \begin{dcases}
            \hfil F_1 \left( \frac{n}{2}+1 \right) & \text{si } n \text{ es par } \\
            \hfil F_2 \left( \frac{n+1}{2} \right) & \text{si } n \text{ es impar y } n>1.
        \end{dcases}
    \end{equation*}
}

Entonces $g \in \mathcal{A}_{\mathbb{R}}$, $\psi(g)(1)=g(1)=F(1)$ y $\psi(g)(n)=g(2n-2)+i g(2n-1)=F_1(n)+i F_2(n)=F(n)$ para cada $n>1$, es decir, $\psi(g)=F$.
\end{proof}

\begin{lemma}\label{prop:iso5}
$(\mathcal{A}_{\mathbb{R}},+) \cong (\mathcal{A},+)$.
\end{lemma}
\begin{proof}
Defínase
\begin{align*}
    \gamma : (\mathcal{A}_{\mathbb{R}},+) & \longrightarrow (\mathcal{A},+) \\
    f & \longmapsto F,
\end{align*}
donde $F$ es la función definida como $F(n)=f(2n-1)+if(2n), \forall n \in \mathbb{N}$. Se tiene que $\gamma$ es un homomorfismo, pues
\begin{align*}
    \gamma(f+g)(n) &= (f+g)(2n-1)+i (f+g)(2n) \\
                   &= f(2n-1)+g(2n-1)+i f(2n)+i g(2n) \\
                   &= [f(2n-1)+i f(2n)]+[g(2n-1)+i g(2n)] \\
                   &= \gamma(f)(n)+\gamma(g)(n),
\end{align*}
por tanto, $\gamma(f+g)=\gamma(f)+\gamma(g)$.
\bigskip

El homomorfismo $\gamma$ es también inyectivo, pues si $f,g \in \mathcal{A}_{\mathbb{R}}$ son tales que $\gamma(f)=\gamma(g)$, entonces $\gamma(f)(n)=\gamma(g)(n)$ para cada $n$, luego $f(2n-1)+i f(2n)=g(2n-1)+ i g(2n)$, por tanto $f(2n-1)=g(2n-1)$ y $f(2n)=g(2n)$ para cada $n$, en consecuencia $f(n)=g(n), \forall n \in \mathbb{N}$, es decir, $f=g$.
\bigskip

Finalmente, se tiene que $\gamma$ es suprayectivo, pues si $F \in \mathcal{A}$, se puede escribir $F=F_1+i F_2$, con $F_1, F_2 \in \mathcal{A}_{\mathbb{R}}$. Defínase
{\everymath{\displaystyle}
    \begin{equation*}
        f(n) = \begin{dcases}
            \hfil F_1 \left( \frac{n+1}{2} \right) & \text{si }n \text{ es impar } \\
            \hfil F_2 \left( \frac{n}{2} \right) & \text{si }n \text{ es par}.
        \end{dcases}
    \end{equation*}
}

Entonces, $\gamma(f)(n)=f(2n-1)+i f(2n)=F_1(n)+i F_2(n)=F(n)$, para cada $n$, es decir, $f \in \mathcal{A}_{\mathbb{R}}$ es tal que $\gamma(f)=F$.
\end{proof}

El resultado principal de esta sección es el siguiente, corolario de los lemas \ref{prop:iso1}, \ref{prop:iso2}, \ref{prop:iso3}, \ref{prop:iso4} y \ref{prop:iso5}.

\begin{theorem}
Los grupos $(\mathcal{A}_{\mathbb{R}},+)$, $(P,*)$, $(\mathcal{M},*)$, $(\mathcal{A}^\prime,+)$, $(\mathcal{A}_1,+)$ y $(\mathcal{A},+)$ son todos isomorfos. 
\end{theorem}

%%% Algunas funciones aritmeticas
\newpage
\subsection{Algunas funciones aritméticas conocidas}

A continuación se presentan algunas funciones aritméticas que aparecen frecuentemente en teoría de números.

\begin{definition}[Función idéntica]
La función idéntica $N$ es tal que $N(n)=n$, para cada $n\in \mathbb{N}$.
\end{definition}

\begin{definition}[Función $\varphi$ de Euler]
Para cada $n\geq 1$, se define la función $\varphi$ de Euler $\varphi(n)$ como el número de enteros positivos no mayores a que $n$ que son primos relativos a $n$.
\end{definition}

\begin{definition}[Función de Mangoldt]
Para todo $n\in\mathbb{N}$, definimos la función de Mangoldt como 
\begin{equation*}
	\Lambda(n) =
		\begin{cases}
			\log p & \text{si} \: n=p^m \text{ para algún primo } p \text{ y } m\geq 1\\ \hfil
			0 & \text{en otro caso}.
		\end{cases}
\end{equation*}
\end{definition}

\begin{definition}[Función de Liouville]
Se define a la función $\lambda$ de Liouville como $\lambda(1)=1$ y dada $n=p_1^{\alpha_1}p_2^{\alpha_2}\cdots p_k^{\alpha_k}$ la factorización de $n$ en primos, entonces $\lambda(n)=(-1)^{\alpha_1+\alpha_2+\cdots+\alpha_k}$.
\end{definition}

\begin{definition}[Función divisor]
Para cada $k\in\mathbb{C}$ se define la función divisor de orden $k$ como 
\begin{equation*}
	\sigma_k(n)=\sum_{d \mid n} d^k.
\end{equation*}
A la función divisor de orden $1$ la llamaremos simplemente función divisor y se denotará como $\sigma$ en vez de $\sigma_1$. La función divisor de orden $0$ se denomina función número de divisores.
\end{definition}

Las funciones aritméticas por sí mismas pueden tener comportamientos aleatorios y difíciles de predecir, pero se pueden observar algunas regularidades cuando sumamos todos los valores que toma la función en los divisores positivos de un número natural dado. Para esto definimos la siguiente notación:

Se tienen las siguientes propiedades básicas de algunas funciones aritméticas.

\begin{proposition}
Para todo $n\in\mathbb{N}$, se tiene que 
\begin{equation*}
	\sum_{d \mid n} \mu(d) = I(n) =
		\begin{cases}
			1 & \text{si} \: n=1 \\ %\hfil
			0 & \text{si} \: n>1
		\end{cases}
\end{equation*}
\end{proposition}
\begin{proof}
Si $n=1$, por definición $\mu(n)=1$. Supongamos que $n>2$ y sea $n=q_1^{\alpha_1}\cdots q_k^{\alpha_k}$ la factorización de $n$ en primos. Todos los divisores de $n$ son de la forma $n=q_1^{\beta_1}\cdots q_k^{\beta_k}$, con $0\leq \beta_i\leq \alpha_i,\:\forall \: i=1,\ldots,k$. Sin embargo, hace falta considerar sólo los factores donde $0\leq \beta_i\leq 1$, pues la función de Möbius se anula para cualesquiera otros. Para un $1\leq i\leq k$ dado, existen $\binom{k}{i}$ $i$-combinaciones (sin repetición y desordenadas) de elementos del conjunto $P=\left\{q_1,\cdots,q_k\right\}$, véase \cite{Br1-1999}. Luego la suma buscada es igual a
\begin{align*}
    \mu(1)+&\sum_{p_1\in\left\{q_1,\ldots,q_k\right\}} \mu(p_1) + \sum_{\substack{p_1,p_2\in \left\{q_1,\ldots,q_k\right\} \\ p_1\neq p_2}} \mu(p_1 p_2)+\cdots+\sum_{\substack{p_1,\ldots,p_k\in \left\{q_1,\ldots,q_k\right\} \\ p_1\neq p_2 \neq \cdots \neq p_k}} \mu(p_1\cdots p_k) \\
						   &= \binom{k}{0}(-1)^0+\binom{k}{1}(-1)^1+\binom{k}{2}(-1)^2+\cdots+\binom{k}{k}(-1)^k =  (1-1)^k=0
\end{align*}
Es decir, $\sum_{d \mid n} \mu(d)=0$.
\end{proof}

\begin{corollary}[Inversión de Möbius]\label{cor:mob1}
Si $f,g \in \mathcal{A}$, entonces para cada $n \in \mathbb{N}$,
\begin{equation*}
    \sum_{d \mid n} f(n) = g(n) \iff \sum_{d \mid n} g(n)\mu \left( \frac{n}{d} \right) = f(n)
\end{equation*}
\end{corollary}
\begin{proof}
De acuerdo con la proposición anterior, se tiene $\mu*\mathbf{1}=\mathbf{1}*\mu=I$, de tal manera que
\begin{equation*}
    f*\mathbf{1} =  g \iff f*\mathbf{1}*\mu = g * \mu \iff f*I = g * \mu \iff f = g * \mu,
\end{equation*}
lo cual es equivalente al enunciado.
\end{proof}

\begin{proposition}[Gauss]\label{eq:gauss1}
Para todo $n\in\mathbb{N}$ se verifica que 
\begin{equation*}
	\sum_{d \mid n} \varphi(d)=n.
\end{equation*}
\end{proposition}
\begin{proof}
La siguiente demostración es debida a Gauss en \cite{Gauss1}. Sea $n\in\mathbb{N}$ y sean $d_1,\cdots,d_k$ los distintos divisores positivos de $n$. Para cada $d_i$, sean $c_{i,1},\cdots,c_{i,m_i}$ todos los enteros positivos primos relativos y no mayores a $d_i$. Notemos que $\varphi(d_i)=m_i$. Afirmamos que el conjunto formado por los números 
\begin{equation*}
	\renewcommand\arraystretch{2}
	\begin{matrix}
	(n/d_1)c_{1,1} & (n/d_1)c_{1,2} & \cdots & (n/d_1)c_{1,m_1} \\
	(n/d_2)c_{2,1} & (n/d_2)c_{2,2} & \cdots & (n/d_2)c_{2,m_2} \\
	\vdots & \vdots & \ddots & \vdots \\
	(n/d_k)c_{k,1} & (n/d_k)c_{k,2} & \cdots & (n/d_k)c_{k,m_k} \\
	\end{matrix}
\end{equation*}
es igual a $\left\{1,2,\ldots,n\right\}$. En efecto, sea $r$ un entero positivo tal que $1\leq r\leq n$  y sea $d=(n,r)$. Notemos que $n/d$ es un divisor de $n$, $r/d\leq n/d$ y $(n/d,r/d)=1$. Además $(n/(n/d))(r/d)=r$, luego $r$ está entre los elementos de la tabla anterior. Recíprocamente, se tiene que $1\leq (n/d_i)c_{i,j}\leq (n/d_i)d_i=n,\:\forall \: i=1,\ldots,k,\:\forall \: j=1,\ldots,m_i$.
\medskip

Finalmente veamos que todos los elementos de la tabla son distintos. Es claro que todos los elementos de cada fila son distintos, pues los divisores de cada $d_i$ son distintos por hipótesis. Si dos números fueran iguales, para algunos divisores $M$ y $N$ de $n$ distintos, podemos suponer que $M>N$. Se tendría pues que $(n/M)\mu=(n/N)\nu$, donde $\mu$ es primo relativo a $M$ y $\nu$ es primo relativo a $N$, luego $\mu N=\nu M$, de manera que $M \mid \mu N$, por tanto $M \mid N$, lo cual no puede ser pues $M>N$. Finalmente: 

\begin{equation*}
	\sum_{d \mid n} \varphi(d) = \varphi(d_1)+\cdots+\varphi(d_k) = m_1+\cdots+m_k = |\left\{1,\ldots,n\right\}| = n
\end{equation*}
\end{proof}

Existe una relación entre las funciones $\mu$ y $\varphi$ al sumar sobre los divisores de un entero positivo. El siguiente lema será útil para probar dicha relación.

\begin{lemma}\label{lemma:car1}
Si $n\in\mathbb{N}$, $d$ es un divisor positivo de $n$, $S=\left\{x\in\mathbb{N} \: : \: 1\leq x\leq n\right\}$ y $A=\left\{x\in S \: : \: d \mid x\right\}$ entonces $|A|=n/d$.
\end{lemma}
\begin{proof}
En efecto, tenemos que la función
\begin{align*}
	F:\left\{1,\ldots,n/d\right\} & \longrightarrow A \\
	x & \longmapsto dx
\end{align*}
es biyectiva, pues si $x,y\in \left\{1,\ldots,n/d\right\}$ son tales que $F(x)=F(y)$, entonces $dx=dy$ y por tanto $x=y$, pues $d\neq 0$. Además, si $r\in A$ entonces $d \mid r$ y $1\leq r\leq n$, por lo que existe $q\in\mathbb{N}$ tal que $r=dq$, luego $q$ es tal que $1\leq q\leq n/d$ y $F(q)=dq=r$. En consecuencia $|A|=|\left\{1,\ldots,n/d\right\}|=n/d$
\end{proof}

\begin{proposition}\label{prop:mob1}
Para todo $n\in\mathbb{N}$ se verifica que 
\begin{equation*}
	\sum_{d \mid n} \mu(d)\frac{n}{d}=\varphi(n).
\end{equation*}
\end{proposition}
\begin{proof}
Si $n=1$ claro que se tiene $\mu(1)=\varphi(1)=1$. Supongamos que $n>1$ y sea $n=p_1^{\alpha_1}\cdots p_r^{\alpha_r}$ su factorización en primos. Sea $S=\left\{1,\ldots,n\right\}$ y para cada $i=1,\ldots,r$ definamos $A_i=\left\{x\in S\: : \: p_i \mid x\right\}$.
\bigskip

Si $1\leq m\leq r$, como todos los $p_i$ son primos distintos, se debe tener que 
\begin{equation*}
	\bigcap_{s=1}^{m} A_i = \left\{x\in S \: : \: p_1 \mid x,p_2 \mid x,\ldots,p_m \mid x\right\}=\left\{x\in S \: : \: p_1 p_2 \ldots p_m \mid x\right\}.
\end{equation*}
Por otro lado, notemos que si $P=\left\{x\in S \: : \: (n,x)=1\right\}$ entonces 
\begin{equation*}
	\bigcap_{i=1}^{r} S \setminus A_i=P.
\end{equation*}
En efecto, si $x\in \bigcup_{i=1}^{r} A_i$ entonces $x\in S$ y $p_i \mid x$, para algún $p_i$, de manera que $p_i \mid n$ y $p_i \mid x$, y por tanto $(n,x)\geq p_i>1$, luego $x\not\in P$. Recíprocamente, si $x\in S$ y $x\not\in P$, entonces $(n,x)>1$ y por tanto debe existir un primo $q$ que divide a $(n,x)$, pero $(n,x)\mid n$ y $(n,x)\mid x$, por lo que $q \mid n$ y $q \mid x$, luego $q=p_i$, para algún $i=1,\ldots,m$. En consecuencia, $p_i \mid x$ y por tanto $x\in \bigcup_{i=1}^{r} A_i$. Se sigue que $\bigcup_{i=1}^{r} A_i=S \setminus P$, o bien $\bigcap_{i=1}^{r} S/A_i=P$.
\bigskip

Como $p_1 \cdots p_m \mid n,\:\forall \: m=1,\ldots,r$, por el lema \eqref{lemma:car1} se debe tener que  $|\bigcap_{s=1}^{m} A_i|=n/p_1 \cdots p_m,\:\forall \: m=1,\ldots,r$. Finalmente, por el principio de inclusión-exclusión, se tiene que 
\begin{align*}
	&\varphi(n) = |P| = \left|\bigcap_{i=1}^{r} S \setminus A_i\right| = |S|+\sum_{i_1\in \{1,\ldots,r\}} (-1)|A_{i_1}|+\sum_{\substack{i_1,i_2\in \{1,\ldots,r\} \\ i_1\neq j_2}} |A_{i_1} \cap A_{i_2}|+\cdots \\
	&+\sum_{\substack{i_1,\ldots,i_r\in \{1,\ldots,r\} \\ i_1\neq \cdots \neq i_r}} (-1)^{r}|A_{i_1}\cap \cdots\cap A_{i_r}| = n+\sum_{i_1\in \{1,\ldots,r\}} (-1)\frac{n}{p_{i_1}}+\sum_{\substack{i_1,i_2\in \{1,\ldots,r\} \\ i_1\neq j_2}} \frac{n}{p_{i_1}p_{i_2}} +\cdots \\
	&+\sum_{\substack{i_1,\ldots,i_r\in \{1,\ldots,m\} \\ i_1\neq \cdots \neq i_r}} (-1)^{r}\frac{n}{p_{i_1}\cdots p_{i_r}} = n+\sum_{i_1\in \{1,\ldots,r\}} \mu(p_{i_1})\frac{n}{p_{i_1}}+\sum_{\substack{i_1,i_2\in \{1,\ldots,r\} \\ i_1\neq j_2}} \mu(p_{i_1}p_{i_2})\frac{n}{p_{i_1}p_{i_2}} + \\
	&\cdots+\sum_{\substack{i_1,\ldots,i_r\in \{1,\ldots,m\} \\ i_1\neq \cdots \neq i_r}} \mu(p_{i_1}\cdots p_{i_r})\frac{n}{p_{i_1}\cdots p_{i_r}} = \sum_{d \mid n} \mu(d)\frac{n}{d}.
\end{align*}
\end{proof}

\begin{corollary}
La función $\varphi$ es multiplicativa, pues $\varphi = \mu * N$, donde $\mu$ y $N$ son funciones multiplicativas.
\end{corollary}

%%% Funciones pares
\newpage
\section{Funciones pares}

Al estudiar el espacio de funciones aritméticas se puede hacer una analogía con la teoría de Fourier del análisis para funciones definidas en todo el plano real o complejo, para la cuál se necesitará la noción de periodicidad. En este capítulo se considerarán dos clases de funciones aritméticas que capturan esta noción y se probará que son equivalentes. También se expondran resultados análogos a los de análisis respecto a funciones periódicas. Estos  resultados se puede encontrar en \cite{Coh1}.

\begin{remark}
Durante todo el capítulo se supondrá que $r$ es un entero positivo arbitrario pero fijo.
\end{remark}

\begin{definition}[Función par]
Una función aritmética se dice \textbf{par} $\Mod{r}$ si $f(n)=f((n,r))$, donde $(m,r)$ es el máximo común divisor de $n$ y $r$, para cada $n \in \mathbb{N}$.
\end{definition}

\begin{definition}[Función periódica]
Una función aritmética se dice \textbf{periódica} con periodo $r$ (o periódica $\Mod{r}$) si $m, n \in \mathbb{N}$ y $m \equiv n \pmod{r}$ implica que $f(m)=f(n)$.
\end{definition}

La siguiente proposición es una consecuencia inmediata de las definiciones anteriores.

\begin{proposition}
Toda función par $\Mod{r}$ es periódica con periodo $r$.
\end{proposition}
\begin{proof}
Si $m \equiv n \pmod{r}$ entonces $r \mid m-n$, por tanto existe $q \in \mathbb{Z}$ tal que $m-n=q r$. Por demostrar que $(n,r)=(m,r)$. En efecto, como $(n,r) \mid n$ y $(n,r) \mid r$, entonces $(n,r) \mid n+qr=m$, luego $(n,r) \mid (m,r)$. Análogamente, se tiene que $(m,r) \mid (n,r)$. Se sigue que $(n,r)=(m,r)$ y por tanto $f(n)=f((n,r))=f((m,r))=f(m)$.
\end{proof}

\subsection{Sumas de Ramanujan}

\begin{definition}[Sumas de Ramanujan]
Se define la función aritmética $c_r$ como
\begin{equation}\label{eq:ram0}
    c_r(n) = \sum_{d \mid (n,r)} \mu \left( \frac{r}{d} \right) d.
\end{equation}
Esta función será referida como la suma de Ramanujan módulo $r$ o simplemente suma de Ramanujan cuando no haya riesgo de confusión.
\end{definition}

\begin{proposition}
Algunas propiedades de la sumas de Ramanujan son las siguientes:
\begin{enumerate}[label=\textnormal{(\arabic*)},ref=\textnormal{\arabic*}]
\item $c_1 = \mathbf{1}$
\item $c_r(1) = \mu(r)$
\item $c_r(n) \le \max \{ \sigma(r), \sigma(n) \}$
\item \label{it:ram1} $c_r(n)$ es una función multiplicativa de $r$
\item Si $p$ es primo y $m$ es un entero positivo, entonces
    \begin{equation*}
        c_{p^m}(n) = \begin{cases}
            \hfil p^m - p^{m-1} & \text{si } p^m \mid n \\
            \hfil -p^{m-1} & \text{si } p^{m-1} \text{ pero } p^m \centernot\mid n \\
            \hfil 0 & \text{si } p^{m-1} \centernot\mid n.
        \end{cases}
    \end{equation*}
\end{enumerate}
\end{proposition}

\begin{proof}
\begin{enumerate}[label=\textnormal{(\arabic*)}]
\item Para cada $n \in \mathbb{N}$ se tiene que $(n,1)=1$ y por tanto
    \begin{equation*}
        c_1(n) = \sum_{d \mid (n,1)} \mu \left( \frac{1}{d} \right) d = \mu(1) 1 = 1.
    \end{equation*}
\item De manera similar,
    \begin{equation*}
        c_r(1) = \sum_{d \mid (1,r)} \mu \left( \frac{r}{d} \right) d = \mu(r) 1 = \mu(r).
    \end{equation*}
\item Por definición se tiene que $\sigma(k) = \sum_{d \mid k} d$. Además $\mu(k) \le 1$ para todo $k \in \mathbb{N}$, luego
    \begin{equation*}
        c_r(n) = \sum_{d \mid (n,r)} \mu \left( \frac{r}{d} \right) d \le \sum_{d \mid (n,r)} d = \sum_{\substack{d \mid n \\ d \mid r}} d \le \sum_{d \mid n} d, \sum_{d \mid r} d \le \max \{ \sigma(n),\sigma(r) \}.
    \end{equation*}
\item Defínase
    \begin{equation*}
        \eta_r(n) = \begin{cases}
            \hfil r & \text{si } r \mid n \\
            \hfil 0 & \text{en otro caso}.
        \end{cases}
    \end{equation*}
Se tiene que la función $\eta_\Box(n)$ es multiplicativa para $n$ fijo. En efecto, si $r,s \in \mathbb{N}$ son tales que $(r,s)=1$, entonces
\begin{equation*}
    \eta_{r s}(n) = \begin{cases}
        \hfil r s & \text{si } r s \mid n \\
        \hfil 0 & \text{en otro caso,}
    \end{cases}
\end{equation*}
pero $r s \mid n$ si y sólo si $r \mid n$ y $s \mid n$. En efecto, si $r s \mid n$ es claro que $r \mid n$ y $s \mid n$. Supóngase que $r \mid n$ y $s \mid n$, de tal manera que existen $q_1, q_2 \in \mathbb{Z}$ tales que $n=r q_1=s q_2$. Como $(r,s)=1$, también existen $x, y \in \mathbb{Z}$ tales que $1=r x + s y$, luego $n=n r x + n s y$, por lo que $n= r s (q_2 x + q_1 y)$, es decir, $r s \mid n$. Luego, si $r s \mid n$, entonces
\begin{equation*}
    \eta_{r s}(n) = r s = \eta_r(n) \eta_s(n),
\end{equation*}
y si $r s \centernot\mid$ entonces $r \centernot\mid n$ y $s \centernot\mid n$, por lo que
\begin{equation*}
    \eta_{r s}(0) = 0 = \eta_r(n) \eta_s(n).
\end{equation*}

Por otro lado, se tiene que
\begin{equation*}
    \sum_{d \mid r} \mu \left( \frac{r}{d} \right) \eta_d(n) = \sum_{\substack{d \mid r \\ d \mid n}} \mu \left( \frac{r}{d} \right) d = \sum_{d \mid (n,r)} \mu \left( \frac{r}{d} \right) d = c_r(n),
\end{equation*}
es decir, $c_\Box(n)=\mu*\eta_\Box(n)$. Luego $c_\Box(n)$ debe ser multiplicativa para $n$ fijo, por ser producto de funciones multiplicativas.
\item Tenemos los siguientes casos:
\begin{itemize}
\item Si $p^m \mid n$, entonces $(n,p^m)=p^m$, luego
    \begin{equation*}
        c_{p^m}(n) = \sum_{d \mid p^m} \mu \left( \frac{p^m}{d} \right) d = \mu(1) p^m + \mu(p) p^{m-1} = p^m - p^{m-1},
    \end{equation*}
    pues $\mu(p^i)=0$ para toda $i > 1$.
\item Si $p^{m-1} \mid n$ pero $p^m \centernot\mid n$, entonces $(n,p^m)=p^{m-1}$. En efecto, se tiene que $p^{m-1} \mid p^m$ y además $p^{m-1} \mid n$ por hipótesis. Si $e \in \mathbb{Z}$ es tal que $e \mid p^{m}$ y $e \mid n$, entonces $e=p^i$, para algún $0 \le i \le m-1$, pues $p^m \centernot\mid n$, por tanto $e \mid p^{m-1}$. Esto prueba que $(p^m,n)=p^{m-1}$, así
    \begin{equation*}
        c_{p^m}(n) = \sum_{d \mid p^{m-1}} \mu \left( \frac{p^m}{d} \right) d = \mu(p) p^{m-1} = -p^{m-1}
    \end{equation*}
    
\item Finalmente, si $p^{m-1} \centernot\mid n$, entonces $p^m \centernot\mid n$. Además, $(n,p^m) \mid p^m$, por tanto $(n,p^m)=p^i$ para algún $0 \le i \le m$. Más aún, por la hipótesis se debe tener que $0 \le i \le m-2$. Luego
    \begin{equation*}
        c_{p^m}(n) = \sum_{d \mid p^i} \mu \left( \frac{p^m}{d} \right) d = \mu(p^m)1+ \mu(p^{m-1}) p + \cdots + \mu(p^{m-i}) p^i = 0,
    \end{equation*}
    pues $i \le m-2$ implica que $2 \le m-i$ y por tanto $\mu(p^m)=\ldots=\mu(p^{m-i})=0$.
\end{itemize}
\end{enumerate}
\end{proof}

Del la demostración del \cref{it:ram1} se puede rescatar el siguiente corolario, usando la inversión de Möbius (\Cref{cor:mob1}).

\begin{corollary}\label{cor:ram5}
Para cada $n \in \mathbb{N}$ fijo se tiene
\begin{equation*}
    \sum_{d \mid r} c_d(n) = \eta_r(n) = \begin{cases}
        \hfil r & \text{si } r \mid n \\
        \hfil 0 & \text{en otro caso}.
    \end{cases}
\end{equation*}
\end{corollary}

Las sumas de Ramanujan gozan de la siguiente propiedad de ``ortogonalidad''.

\begin{lemma}\label{lem:ram4}
Si $r$ y $s$ dividen a $k$, entonces
\begin{equation*}
    \sum_{d \mid k} c_r(k/d) c_d(k/s) = \begin{cases}
        \hfil k & \text{si } r = s \\
        \hfil 0 & \text{en otro caso}.
    \end{cases}
\end{equation*}
\end{lemma}
\begin{proof}
Si $r$ y $s$ dividen a $k$, entonces
\begin{equation} \label{eq:ram2}
\begin{split}
\sum_{d \mid k} c_r (k/d) c_d (k/s) &= \sum_{d \mid k} c_d (k/s) \sum_{d' \mid (k/d,r)} \mu (r/d') d' \\
                                                                              &= \sum_{d \mid k} c_d (k/s) \sum_{\substack{d' \mid r \\ d' \mid k/d}} \mu(r/d') d' = \sum_{\substack{d \mid k \\ d' \mid r \\ d' \mid k/d}} c_d (k/s) \mu (r/d') d' \\
                                                                              &= \sum_{\substack{d \mid k/d' \\ d' \mid r \\ d' \mid r}} c_d(k/s) \mu(r/d')  d' = \sum_{\substack{d' \mid r \\ d' \mid k}} \mu(r/d') d' \sum_{d \mid k/d'} c_d(k/s) \\
                                                                              &= \sum_{d' \mid (k,r)} \mu(r/d) d' \eta_{k/d'} (k/s) = \sum_{d' \mid r} \mu(r/d) d' \eta_{k/d'} (k/s),
\end{split}
\end{equation}
dado que $(k,r)=r$ por ser $r$ divisor de $k$ y dado que los conjuntos $\{ d,d' \in \mathbb{N} \std d \mid k, d' \mid r, d' \mid k/d \}$ y $\{ d,d' \in \mathbb{N} \std d \mid k/d', d' \mid r, d' \mid k \}$ son iguales. En efecto, si $d \mid k$ entonces $k/d$ es un entero, lueg $d' \mid k/d$ implica que $k/d=d' q'$, luego $k=d' q' d$, por tanto $d \mid k/d'$ y $d' \mid k$.
\bigskip

Recíprocamente, si $d' \mid k$ entonces $k/d'$ es un entero, luego $d \mid k/d'$ implica que $k/d'=dq$, por tanto $k=d q d'$, por tanto $d \mid k$ y $d' \mid k/d$.
\bigskip

Si $s \centernot\mid r$ entonces $s \centernot\mid d'$ y por tanto $k/d' \centernot\mid k/s$. En efecto, pues si $s \mid d'$, como $d' \mid r$ entonces se tendría que $s \mid r$ por transitividad. Además, si $k/d' \mid k/s$ se tendría que $s \mid d'k$. Luego la suma \eqref{eq:ram2} se anula si $s \centernot\mid r$ y en particular si $r \ne s$, pues en este caso se tiene que $\eta_{k/d'}(k/s)=0$ para cada $d' \mid r$. 
\bigskip

Si $s \mid r$ entonces la suma \eqref{eq:ram2} es igual a
\begin{equation*}
\begin{split}
    \sum_{\substack{d' \mid r \\ k/d' \mid k/s}} \mu(r/d') d' \frac{k}{d'} &= \sum_{\substack{d' \mid r \\ k/d' \mid k/s}} \mu(r/d') k = \sum_{\substack{d' \mid r \\ s \mid d'}} \mu(r/d') k \\
    &= k \sum_{\substack{d' \mid r \\ d'=se}} \mu(r/se) = k \sum_{e \mid r/s} \mu(r/se) \\
    &= k \sum_{se \mid r} \mu(r/se) = \begin{cases}
        \hfil k & \text{si } r=s \\
        \hfil 0 & \text{en otro caso,}
    \end{cases}
\end{split}
\end{equation*}
pues $k/d' \mid k/s$ si y sólo si $s \mid d'$.
\end{proof}

\begin{corollary}\label{cor:ind}
Las sumas de Ramanujan son linealmente independientes respecto a la suma sobre los divisores de $r$. Más específicamente, si $\alpha, \beta$ son funciones aritméticas tales que
\begin{equation*}
    \sum_{d \mid r} \alpha(d) c_d(n) = \sum_{d \mid r} \beta(d) c_d(n),
\end{equation*}
para todo $n \in \mathbb{N}$, entonces $\alpha(d)=\beta(d)$ para todo $d \mid r$.
\end{corollary}
\begin{proof}
Basta probar que $\sum_{d \mid r} \alpha(d) c_d(n)=0$ implica que $\alpha(d)=0$ para todo $d \mid r$. Supóngase ésta hipótesis y sea $\delta \mid r$ arbitrario pero fijo. Si $e$ es un divisor de $r$, se tiene que $\sum_{d \mid r} \alpha(d) c_d(r/e)=0$, luego
\begin{align*}
    0 = \sum_{e \mid r} \left( \sum_{d \mid r} \alpha(d) c_d \left( \frac{r}{e} \right) \right) c_e \left( \frac{r}{\delta} \right)= \sum_{d \mid r} \alpha(d) \sum_{e \mid r} c_d \left( \frac{r}{e} \right) c_e \left( \frac{r}{\delta} \right) = \alpha(\delta) r,
\end{align*}
por el la proposición anterior, y dado que $r \ne 0$ entonces $\alpha(\delta)=0$. Como $\delta$ fue un divisor arbitrario de $r$, se tiene el resultado.
\end{proof}

\begin{lemma}
Si $d \mid r$ entonces $c_d(n)=c_d((n,r))$.
\end{lemma}
\begin{proof}
Si $d \mid r$ entonces $(n,d)=((n,r),d)$. En efecto, dado que $(n,d) \mid n$ y $(n,d) \mid d$, entonces $(n,d) \mid n$, $(n,d) \mid d$ y $(n,d) \mid r$, por lo que $(n,d) \mid (n,r)$ y $(n,d) \mid d$, es decir, $(n,d) \mid ((n,r),d)$. Recíprocamente se tiene que $((n,r),d) \mid n$ y $((n,r),d) \mid d$, así que $((n,r),d) \mid (n,d)$. Se sigue que $(n,d)=((n,r),d)$. Luego
\begin{equation*}
    c_d(n) = \sum_{e \mid (n,d)} \mu(d/e) e = \sum_{e \mid ((n,r),d)} \mu(d/e) e = c_d((n,r)).
\end{equation*}
\end{proof}

\begin{corollary}
La suma de Ramanujan módulo $r$ es par $\Mod{r}$.
\end{corollary}

\begin{definition}[Radical]
Sea $n \in \mathbb{N}$. Se define el \emph{radical} de $n$, denotado por $n_*$ como
\begin{equation*}
    n_* = \begin{cases}
        \hfil 1 & \text{si } n=1 \\
        \hfil p_1 \cdots p_r & \text{si } n = p_1^{\alpha_1} \cdots p_r^{\alpha_r}
    \end{cases}
\end{equation*}
donde $n=p_1^{\alpha_1} \cdots p_r^{\alpha_r}$ es la factorización de $n>1$ en primos.
\end{definition}

\begin{definition}
Una función aritmética $f$ se dirá \emph{separable} si $f(n)=f(n_*)$, para cada $n \in \mathbb{N}$.
\end{definition}

\begin{lemma}
Una función multiplicativa es separable si y sólo si $(\mu * f)(n)=0$ para todo $n$ no libre de cuadrado.
\end{lemma}
\begin{proof}
Sea $F=\mu * f$. Entonces $F*\mathbf{1}=f$, es decir,
\begin{equation*}
    \sum_{d \mid n} F(d) = f(n), \forall n \in \mathbb{N}.
\end{equation*}
Si $F(n)=0$ para cada $n$ no libre de cuadrado, entonces
\begin{equation*}
    f(n) = \sum_{d \mid n} F(d) = \sum_{d \mid n_*} F(d) = f(n_*),
\end{equation*}
es decir, $f$ es separable.
\bigskip

Supóngase ahora que $f$ es separable. Se tiene que para cada primo $p$ y para cada $m>1$,
\begin{align*}
    F(p^m) = \sum_{d \mid p^m} \mu(d) f \left( \frac{p^m}{d} \right) &= \mu(1)f(p^m) + \mu(p)f(p^{m-1}) \\
                                                                     &= f(p^m) - f(p^{m-1}) = f(p) - f(p) = 0.
\end{align*}
Además como $f$ es multiplicativa, entonces $F$ también lo es. Si $n$ es un entero positivo no libre de cuadrado, entonces existen un primo $p$ y enteros positivos $q$ y $m>1$ tales que $n=p^m q$ y $(p^m,q)=1$. Luego $F(n)=F(p^m)F(q)=0 \cdot F(q)=0$.
\end{proof}

\begin{lemma}\label{lem:par0}
Si $f$ es multiplicativa y separable, entonces para cualesquiera $a,b \in \mathbb{N}$ se tiene:
\begin{enumerate}[label=\textnormal{(\roman*)}]
\item $f(a)f(b)=f(a b)f((a,b))$.
\item $\displaystyle f(a) = f((a,b)) \sum_{\substack{d \mid a \\ (d,b) = 1}} (\mu * f)(d)$
\end{enumerate}
\end{lemma}

\begin{proof}
(\textsc{\romannumeral 1}) Nótese que si $p$ es un primo y $m,n > 1$ entonces
\begin{equation*}
    f(p^m)f(p^n) = f(p) f(p) = f(p^{m+n})f((p^m,p^n)),
\end{equation*}
pues $(p^m,p^n)=p^i$, con $i=\min \{ m,n \}$. Sean $a,b \in \mathbb{N}$ y escríbase sin pérdida de generalidad $a=p_1^{\alpha_1} \cdots p_r^{\alpha_r}$ y $b=p_1^{\beta_1} \cdots p_r^{\beta_r}$, $0 \le \alpha_i, \beta_i$. Entonces, como $f$ es multiplicativa,
\begin{align*}
    f(a b)f((a,b)) &= f(p_1^{\alpha_1+\beta_1} \cdots p_r^{\alpha_r+\beta_r}) f(p_1^{\min \{ \alpha_1,\beta_1 \}} \cdots p_r^{\min \{ \alpha_r,\beta_r \}}) \\
                   &= f(p_1^{\alpha_1+\beta_1}) \cdots f(p_r^{\alpha_r+\beta_r}) f(p_1^{\min \{ \alpha_1,\beta_1 \}}) \cdots f(p_r^{\min \{ \alpha_r,\beta_r \}}) \\
                   &= f(p_1^{\alpha_1+\beta_1}) \cdots f(p_r^{\alpha_r+\beta_r}) f((p_1^{\alpha_1},p_1^{\beta_1})) \cdots f((p_r^{\alpha_r},p_r^{\beta_r})) \\
                   &= f(p_1^{\alpha_1})f(p_1^{\beta_1}) \cdots f(p_r^{\alpha_r})f(p_r^{\beta_r}) \\
                   &= f(p_1^{\alpha_1} \cdots p_r^{\alpha_r})f(p_1^{\beta_1} \cdots p_r^{\beta_r}) \\
                   &= f(a)f(b)
\end{align*}
(\textsc{\romannumeral 2}) Al igual que en la demostración anterior, si $F=\mu * f$, entonces
\begin{equation*}
    \sum_{d \mid n} F(d) = f(n), \forall n \in \mathbb{N}.
\end{equation*}

Se verá primero que los conjuntos $\{ d \in \mathbb{N} \std d \mid a_* \text{ y } (d,b)=1\}$ y $\{ d \in \mathbb{N} \std d \mid a_* / (a,b)_* \}$ son iguales.
\bigskip

Para empezar, se tiene que $a_*/(a,b)_*$ es un entero. Si $(a,b)_*=1$ esto es claro. Si $(a,b)_*>1$ se puede escribir $(a,b)_*=q_1 \cdots q_s$, donde todos los primos son distintos. Luego $q_i \mid (a,b)_*$, pero $(a,b)_* \mid (a,b)$ y $(a,b) \mid a$, por tanto $q_i \mid a$ y por tanto $q_i \mid a_*$. Como $i \in \{ 1,\ldots,s \}$ fue arbitrario y todos los primos $q_i$ son distintos, entonces $q_1 \cdots q_s = (a,b)_* \mid a_*$, que es lo que se quería probar.
\bigskip

Procedamos a probar la igualdad de los conjuntos. Supóngase primero que $d \mid a_*$ y $(d,b)=1$. Entonces existe $c \in \mathbb{N}$ tal que $a_*=d c$. Por otro lado, se tiene que $(a,b) \mid b$ y por tanto $((a,b),d)=1$, más aún, como $(a,b)_* \mid (a,b)$ entonces también $((a,b)_*,d)=1$ y como $(a,b)_* \mid a_* = d c$, por el lema de Euclides se debe tener que $(a,b)_* \mid c$ es decir, $a_* = (a,b)_* d q$, para algún $q \in \mathbb{N}$, luego $d \mid a_* / (a,b)_*$.
\bigskip

Recíprocamente, supóngase que $d \mid a_*/(a,b)_*$. Se debe tener que
\begin{equation}\label{eq:mcd2}
    \left( \frac{a_*}{(a,b)_*},b \right) = 1.
\end{equation}
Pues en caso contrario, es decir, si este máximo común divisor fuera mayor que uno, existiría un primo $p$ tal que $p \mid b$ y $p \mid a_*/(a,b)_*$, pero $a_*/(a,b)_* \mid a_*$, luego $p \mid a_*$ y por tanto $p \mid a$. En consecuencia, $p \mid (a,b)$ y por tanto $p \mid (a,b)_*$. Se puede escribir entonces $a_*=p p_1 \cdots p_r$, $(a,b)_*= p q_1 \cdots q_s$, donde todos los primos son distintos. Además, como $a_*=(a,b)_* n$ para algún $n \in \mathbb{N}$, se tiene que
$p p_1 \cdots p_r = p q_1 \cdots q_s r_1 \cdots r_t$, con $n=r_1 \cdots r_t$, y $r_i$ números primos, no necesariamente distintos. Luego $p_1 \cdots p_r = q_1 \cdots q_s r_1 \cdots r_t$ y dado que ningúno de los primos $p_i$ son iguales a $p$, entonces ninguno de los primos $r_j$ puede ser igual a $p$, es decir $p$ no divide a $n=a_*/(a,b)_*$, lo cual es absurdo.
\bigskip

Esto prueba la igualdad de dichos conjuntos. Ahora es fácil calcular la siguiente suma,
\begin{equation*}
    \sum_{\substack{d \mid a \\ (d,b)=1}} F(d) = \sum_{\substack{ d \mid a_* \\ (d,b)=1}} F(d) = \sum_{d \mid a_* / (a,b)_*} F(d) = \sum_{d \mid (a_*/(a,b)_*)} (\mu * f)(d) = f(a_*/(a,b)_*).
\end{equation*}
Además, por una demostración similar a la de la \cref{eq:mcd2}, se tiene que
\begin{equation*}
    \left( (a,b)_*,\frac{a_*}{(a,b)_*} \right) = 1.
\end{equation*}
Finalmente, como $f$ es multiplicativa,
\begin{equation*}
    f(a) = f(a_*) = f((a,b)_*)f(a_*/(a,b)_*) = f((a,b)) \sum_{\substack{d \mid a \\ (d,b)=1}} (\mu * f)(d).
\end{equation*}
\end{proof}

\begin{example}
La función $\overline{\varphi}=\varphi(n)/n$ es separable. Nótese que para cualquier primo $p$ y $m>0$ se tiene $\varphi(p^m)=p^m-p^{m-1}$, luego $\varphi(p^m)/p^m=1-p^{-1}$ y también $\varphi(p)/p=1-p^{-1}$. Ahora, si $n=p_1^{\alpha_1} \cdots p_r^{\alpha_r}$ entonces, como $\varphi$ es multiplicativa,
\begin{equation*}
    \frac{\varphi(n)}{n} = \frac{\varphi(p_1^{\alpha_1})}{p_1^{\alpha_1}} \cdots \frac{\varphi(p_r^{\alpha_r})}{p_r^{\alpha^r}} = \left( 1-\frac{1}{p_1} \right) \cdots \left( 1-\frac{1}{p_r} \right) = \frac{\varphi(p_1)}{p_1} \cdots \frac{\varphi(p_r)}{p_r} = \frac{\varphi(n_*)}{n_*}
\end{equation*}
\end{example}

\begin{lemma}[Fórmula de Hölder]\label{lem:holder}
Para cada $n \in \mathbb{N}$ se tiene
\begin{equation*}
    c_r(n) = \frac{\varphi(r) \mu \left( \displaystyle \frac{r}{(n,r)} \right)}{\varphi \left( \displaystyle \frac{r}{(n,r)} \right)}
\end{equation*}
\end{lemma}

\begin{proof}
Se tiene
\begin{equation}\label{eq:par1}
    c_r (n) = \sum_{d \mid (n,r)} \mu \left( \frac{r}{d} \right) d = \sum_{\substack{d \mid (n,r) \\ \left( \frac{r}{(n,r)}, \frac{(n,r)}{d} \right) > 1}} \mu \left( \frac{r}{d} \right) d + \sum_{\substack{d \mid (n,r) \\ \left( \frac{r}{(n,r)}, \frac{(n,r)}{d} \right) = 1}} \mu \left( \frac{r}{d} \right) d,
\end{equation}
pero si $(r/(n,r),(n,r)/d)>1$ entonces $r/d$ debe tener un factor cuadrado, pues en este caso existe un primo $p$ tal que $p \mid r/(n,r)$ y $p \mid (n,r)/d$, luego $r=p (n,r) q_1$ y $(n,r)=p d q_2$ para algunos enteros $q_1$ y $q_2$, luego $r=p^2 d q_1 q_2$ y por tanto $p^2 \mid r/d$, así que $\mu(r/d)=0$. Luego la ecuación \eqref{eq:par1} es igual a
\begin{equation}\label{eq:par2}
\begin{split}
    \sum_{\substack{d \mid (n,r) \\ \left( \frac{r}{(n,r)}, \frac{(n,r)}{d} \right) = 1}} \mu \left( \frac{r}{d} \right) d & = \sum_{\substack{d \mid (n,r) \\ \left( \frac{r}{(n,r)}, \frac{(n,r)}{d} \right) = 1}} \mu \left( \frac{r}{(n,r)} \right) \mu \left( \frac{(n,r)}{d} \right) d \\
    & = \mu \left( \frac{r}{(n,r)} \right) \sum_{\substack{d \mid (n,r) \\ \left( \frac{r}{(n,r)}, \frac{(n,r)}{d} \right) = 1}} \mu \left( \frac{(n,r)}{d} \right) d \\
    & = \mu \left( \frac{r}{(n,r)} \right) \sum_{\substack{d \mid (n,r) \\ \left( \frac{r}{(n,r)},d \right)=1}} \mu(d) \frac{(n,r)}{d} \\
    & = (n,r) \mu \left( \frac{r}{(n,r)} \right) \sum_{\substack{d \mid (n,r) \\ \left( \frac{r}{(n,r),d} \right)=1}} \frac{\mu(d)}{d}
\end{split}
\end{equation}
pues $\mu$ es multiplicativa. Sea ahora $\Phi=\mu * \overline{\varphi}$, donde $\overline{\varphi}(s)=\varphi(s)/s$ para cada $s \in \mathbb{N}$. Se tiene entonces que
\begin{equation}\label{eq:par3}
\begin{split}
    \Phi(s) = \sum_{d \mid s} \mu(d) \overline{\varphi} \left( \frac{s}{d} \right) & = \sum_{d \mid s} \mu(d) \varphi \left( \frac{s}{d} \right) \frac{1}{s/d} = \sum_{d \mid s} \mu(d) \sum_{e \mid s/d} \mu(e) \frac{s/d}{e} \frac{1}{s/d} \\
            & = \sum_{d \mid s} \mu(d) \sum_{e \mid s/d} \frac{\mu(e)}{e}  = \sum_{e \mid s} \frac{\mu(e)}{e} \sum_{d \mid s/e} \mu(d) = \frac{\mu(s)}{s}
\end{split}
\end{equation}
pues si $d \mid s$ y $c \mid s/d$, entonces $d/s$ es un entero y $s/d=e q$ para algún entero $q$, luego $s= d e q$ y por tanto $e \mid s$ y $d \mid s/e$. El recíproco es similar. Además, todos los términos en la penúltima suma son cero excepto aquel para el cual $s/e = 1$, es decir, $s=e$. Luego la suma \eqref{eq:par2} es igual a
\begin{align*}
    (n,r) \mu \left( \frac{r}{(n,r)} \right) \sum_{\substack{d \mid (n,r) \\ \left( \frac{r}{(n,r)},d \right)=1}} \Phi(d) & = (n,r) \mu \left( \frac{r}{(n,r)} \right) \sum_{\substack{d \mid (n,r) \\ \left( \frac{r}{(n,r)},d \right)=1}} (\mu * \overline{\varphi}(d)) \\
    & = (n,r) \mu \left( \frac{r}{(n,r)} \right) \frac{\overline{\varphi}((n,r))}{\overline{\varphi} \left( (n,r), \frac{r}{(n,r)} \right)} \\
    & = (n,r) \mu \left( \frac{r}{(n,r)} \right) \frac{\overline{\varphi}(r)\overline{\varphi}((n,r))}{\overline{\varphi}((n,r))\overline{\varphi} \left( \frac{r}{(n,r)} \right)} \\
    & = (n,r) \mu \left( \frac{r}{(n,r)} \right) \frac{\overline{\varphi}(r)}{\overline{\varphi}\left( \frac{r}{(n,r)} \right)} \\
    & = (n,r)\mu \left( \frac{r}{(n,r)} \right) \frac{r \varphi(r)}{(n,r) r \varphi \left( \frac{r}{(n,r)} \right)} \\
    & = \frac{\mu \left( \frac{r}{(n,r)} \right) \varphi(r)}{\varphi \left( \frac{r}{(n,r)} \right)}
\end{align*}
donde la primera igualdad se cumple por definición de $\Phi$ y la ecuación \eqref{eq:par3}, la segunda por ser $\overline{\varphi}$ multiplicativa, separable y por el \Cref{lem:par0} (\textsc{\romannumeral 2}), la tercera por el \Cref{lem:par0} (\textsc{\romannumeral 1}) y la quinta por definición de $\overline{\varphi}$.
\end{proof}

\begin{theorem}\label{thm:fou1}
Toda función $f$ par $\Mod{r}$ tiene una expansión de la forma
\begin{equation}\label{eq:ram3}
    f(n) = \sum_{d \mid r} \alpha(d) c_d(n),
\end{equation}
y recíprocamente, toda función aritmética de esta forma es par $\Mod{r}$. Los coeficientes $\alpha(d)$ están dados por
\begin{equation}\label{eq:ram6}
    \alpha(d) = \frac{1}{r} \sum_{e \mid r} f \left( \frac{r}{e} \right) c_e \left( \frac{r}{d} \right),
\end{equation}
o por la fórmula equivalente,
\begin{equation*}
    \alpha(d) = \frac{1}{r \varphi(d)} \sum_{m=1}^{r} f(m) c_d(m).
\end{equation*}
A los coeficientes $\alpha$ se les llamará \textbf{coeficientes de Fourier} de la función par $f$.
\end{theorem}
\begin{proof}
Es claro que toda función de la forma \eqref{eq:ram3} es par $\Mod{r}$, pues por el lema anterior si $d \mid r$ entonces $c_d(n)=c_d((n,r))$. Nótese que
\begin{equation*}
\begin{split}
    \sum_{d \mid r} \alpha(d) c_d(n) &= \sum_{d \mid r} \left( \frac{1}{r} \sum_{e \mid r} f \left( \frac{r}{e} \right) c_e \left( \frac{r}{d} \right) \right) c_d(n) \\
                                     &= \frac{1}{r} \sum_{e \mid r} f \left( \frac{r}{e} \right) \sum_{d \mid r} c_e \left( \frac{r}{d} \right) c_d(n) \\
                                     &= \frac{1}{r} \sum_{e \mid r} f \left( \frac{r}{e} \right) \sum_{d \mid r} c_e \left( \frac{r}{d} \right) c_d((n,r)) \\
                                     &= \frac{1}{r} f \left( \frac{r}{q} \right) r = f((n,r)) = f(n),
\end{split}
\end{equation*}
por el \Cref{lem:ram4}, donde $r=(n,r) q$, para algún $q \in \mathbb{N}$ y donde la última igualdad se cumple por ser $f$ par $\Mod{r}$.
\bigskip

Por otro lado, de la demostración de la \cref{eq:gauss1} se puede rescatar el hecho de que el conjunto $\{ 1,2,\ldots,r \}$ es igual a $\bigcup_{e \mid r} \{ rx/e \std (x,e)=1, 1 \le x \le e \}$ y todos los conjuntos son disjuntos a pares, por tanto
\begin{equation*}
\begin{split}
    \frac{1}{r \varphi(d)}\sum_{m=1}^{r} f(m) c_d(m) &= \frac{1}{r \varphi(d)}\sum_{e \mid r} \sum_{\substack{(x,e)=1 \\ 1 \le x \le e}} f \left( \frac{rx}{e} \right) c_d \left( \frac{rx}{e} \right) \\
                               &= \frac{1}{r \varphi(d)}\sum_{e \mid r} \sum_{\substack{(x,e)=1 \\ 1 \le x \le e}} f \left( \left( \frac{rx}{e},r \right) \right) c_d \left( \left( \frac{rx}{e},r \right) \right) \\
                               &= \frac{1}{r \varphi(d)}\sum_{e \mid r} \sum_{\substack{(x,e)=1 \\ 1 \le x \le e}} f \left( \frac{r}{e} \right) c_d \left( \frac{r}{e} \right) \\
                               &= \frac{1}{r \varphi(d)}\sum_{e \mid r} f \left( \frac{r}{e} \right) c_d \left( \frac{r}{e} \right) \varphi(e) \\
                               &= \frac{1}{r \varphi(d)}\sum_{e \mid r} f \left( \frac{r}{e} \right) c_e \left( \frac{r}{d} \right) \varphi(d) \\
                               &= \frac{1}{r}\sum_{e \mid r} \left( \frac{r}{e} \right) c_e \left( \frac{r}{d} \right),
\end{split}
\end{equation*}
donde la segunda igualdad se cumple por ser $f$ par $\Mod{r}$, la tercera por ser $(rx/e,r)=r/e$, pues $(x,e)=1$ implica que $(r/e)(x,e)=r/e$, y como $r/e$ es un entero positivo, entonces $(rx/e,r)=r/e$. La cuarta por definición de $\varphi$, y la penúltima igualdad se cumple por la fórmula de Hölder (\Cref{lem:holder}) y el \Cref{cor:mcd1}, pues $e$ y $d$ dividen a $d$, así que
\begin{equation}\label{eq:holder}
    c_d \left( \frac{r}{e} \right) \varphi(e) = \frac{\varphi(d)\mu \left( \displaystyle \frac{d}{(r/e,d)} \right)}{\varphi \left( \displaystyle \frac{d}{(r/e,d)} \right)} = \frac{\varphi(d)\mu \left( \displaystyle \frac{e}{(r/d,e)} \right)}{\varphi \left( \displaystyle \frac{e}{(r/d,e)} \right)} \varphi(e) = \varphi(d) c_e \left( \frac{r}{d} \right)
\end{equation}
\end{proof}

\begin{corollary}
Si $f$ y $f'$ son funciones pares $\Mod{r}$, entonces se verifican las siguientes implicaciones
\begin{equation*}
    f(n) = \sum_{d \mid r} f'(d) c_d(n), \forall n \in \mathbb{N} \implies f'(\delta) = \frac{1}{r \varphi(\delta)} \sum_{m=1}^{r} f(m) c_d(m), \forall \delta \mid r,
\end{equation*}
\begin{equation*}
    f'(\delta) = \sum_{m=1}^{r} f(m) c_{\delta} (m), \forall d \mid r \implies f(n) = \frac{1}{r} \sum_{d \mid r} \frac{f'(d)}{\varphi(d)} c_d(n), \forall n \in \mathbb{N}.
\end{equation*}
\end{corollary}
\begin{proof}
En efecto, como $f$ es par, entonces $f$ tiene una única representación de la forma \eqref{eq:ram3}, por tanto se verifica la primera implicación. Para la segunda implicación se tiene que, de nuevo por la \cref{eq:ram3},
\begin{equation*}
    f(n) = \sum_{d \mid r} \left( \frac{1}{r \varphi(d)} \sum_{m=1}^{r} f(m) c_d(m) \right) c_d(n) = \frac{1}{r} \sum_{d \mid r} \frac{f'(d)}{\varphi(d)} c_d(n).
\end{equation*}
\end{proof}

Considerando los casos $\delta=1$ y $n=r$ se tiene el siguiente corolario.

\begin{corollary}
Si $f$ y $f'$ son funciones pares $\Mod{r}$, entonces
\begin{equation*}
    f(n) = \sum_{d \mid r} f'(d) c_d(n), \forall n \in \mathbb{N} \implies f'(1) = \frac{1}{r} \sum_{m=1}^{r} f(m),
\end{equation*}
\begin{equation*}
    f'(\delta) = \sum_{m=1}^{r} f(m) c_{\delta}(m), \forall d \mid r \implies f(r) = \frac{1}{r} \sum_{d \mid r} f'(d).
\end{equation*}
\end{corollary}

Volviendo a la definición de la suma de Ramanujan módulo $r$, \cref{eq:ram0}, se puede considerar una clase más general de funciones, aquellas que se pueden escribir como
\begin{equation*}
    f(n) = \sum_{\delta \mid (n,r)} g(\delta), \forall n \in \mathbb{N},
\end{equation*}
donde $g$ es una función aritmética.
\bigskip

La siguiente proposición muestra que esta generalización preserva la modularidad respecto a $r$. Más aún, el \Cref{thm:fou1} permite caracterizar a las funciones pares módulo $r$ de esta forma.

\begin{theorem}
Toda función $f$ par módulo $r$ se puede expresar como
\begin{equation}\label{eq:fou2}
    f(n) = \sum_{\delta \mid (n,r)} g(\delta), \forall n \in \mathbb{N}.
\end{equation}
Y recíprocamente, toda función aritmética que tenga esta forma es par módulo $r$.
\end{theorem}
\begin{proof}
Dado que $(n,r)=((n,r),r)$, es claro que toda función arimética que tenga dicha forma es par $\Mod{r}$. Supóngase que $f$ es par $\Mod{r}$. Por el \Cref{thm:fou1}, se puede escribir
\begin{align*}
    f(n) & = \sum_{d \mid r} \alpha(d) c_d(n) = \sum_{d \mid r} \alpha(d) \sum_{\delta \mid (n,d)} \mu \left( \frac{d}{\delta} \right) \delta \\
         & = \sum_{\delta \mid (n,d)} \sum_{d \mid r} \alpha(d) \mu \left( \frac{d}{\delta} \right) \delta = \sum_{\delta \mid (n,r)} \delta \sum_{d \mid r} \mu \left( \frac{d}{\delta} \right) \delta = \sum_{\delta \mid (n,r)} g(\delta) 
\end{align*}
\end{proof}

En vista de que toda función par tiene al menos dos representaciones, la del teorema anterior y la del \Cref{thm:fou1}, cabe preguntarse si existe alguna relación entre ellas. El siguiente teorema da respuesta a esta inquietud.

\begin{theorem}
Si $f$ es par módulo $r$, entonces sus respectivas expansiones \eqref{eq:ram3} y \eqref{eq:fou2} están relacionadas mediante la fórmula
\begin{equation*}
    \alpha(d) = \frac{1}{r} \sum_{e \mid r/d} g \left( \frac{r}{e} \right) e.
\end{equation*}
\end{theorem}
\begin{proof}
Se tiene que
\begin{align*}
    \alpha(d) = \frac{1}{r} \sum_{\delta \mid r} f \left( \frac{r}{\delta} \right) c_{\delta} \left( \frac{r}{d} \right) & = \frac{1}{r} \sum_{\delta \mid r} c_{\delta} \left( \frac{r}{d} \right) \sum_{e \mid (r/\delta,\delta)} g(e) = \frac{1}{r} \sum_{\delta \mid r} c_{\delta} \left( \frac{r}{d} \right) \sum_{e \mid r/\delta} g(e) \\
              & = \frac{1}{r} \sum_{\delta \mid r} c_{\delta} \left( \frac{r}{d} \right) \sum_{e \mid r/\delta} g(e) = \frac{1}{r} \sum_{e \mid r} g \left( \frac{r}{e} \right) \sum_{\delta \mid e} c_{\delta} \left( \frac{r}{d} \right) \\
              & = \frac{1}{r} \sum_{e \mid r} g \left( \frac{r}{e} \right) \cdot \begin{cases}
                  \hfil e & \text{si } e \mid r/d \\
                  \hfil 0 & \text{en otro caso} \\
              \end{cases} = \frac{1}{r} \sum_{e \mid r/d} g \left( \frac{r}{e} \right) e.
\end{align*}
por el \cref{cor:ram5}.
\end{proof}

\subsection{El subespacio de funciones pares}

Recuérdese del \cref{cor:est1} que el anillo $(\mathcal{A}, +, *)$ es un álgebra conmutativa con identidad. Se tiene que el conjunto de funciones pares es un subespacio de $\mathcal{A}$, pero no es un subanillo, pues en general $(f*g)(n) \ne (f*g)((n,r))$ aún cuando $f$ y $g$ sean pares $\Mod{r}$, tome por ejemplo dos funciones constantes. Se denotará como $\mathcal{A}_r$ al conjunto de funciones pares módulo $r$.

\begin{proposition}
El conjunto $\mathcal{A}_r$ es una subespacio de $\mathcal{A}$.
\end{proposition}
\begin{proof}
Basta verificar las siguientes condiciones:
\begin{enumerate}[label=\textnormal{(\roman*)}]
\item $\mathcal{A}_r \ne \emptyset$
\item $f, g \in \mathcal{A}_r$ implica $f+g \in \mathcal{A}_r$
\item $c \in \mathbb{C}$ y $f \in \mathcal{A}_r$ implica $c f \in \mathcal{A}$.
\end{enumerate}
Claro que $\mathbf{0} \in \mathcal{A}_r$. Sea $n \in \mathbb{N}$ y supóngase que $f, g \in \mathcal{A}_r$. Se tiene que $(c f)(n) = c f(n) = c f((n,r)) = (c f)((n,r))$, luego $c f \in \mathcal{A}_r$.
\bigskip

Además, $(f+g)(n)=f(n)+g(n)=f((n,r))+g((n,r))=(f+g)((n,r))$, así que $f+g \in \mathcal{A}_r$.
\bigskip
\end{proof}

El \cref{cor:ind} afirma que las sumas de Ramanujan $\mathcal{B}_r = \{ c_d \}_{d \mid r}$ son linealmente independientes respecto a la suma sobre los divisores de $r$, además el \Cref{thm:fou1} nos permite expresar cualquier función par como una suma de este tipo. En otras palabras, el conjunto $\mathcal{B}_r$ es una base del espacio vectorial $\mathcal{A}_r$.

\begin{corollary}
El espacio vectorial $\mathcal{A}_r$ tiene dimensión $d(r)$.
\end{corollary}

En lo que sigue de esta sección se verá que los coeficientes de la expansión \eqref{eq:ram3} pueden ser derivados de un producto interno en este espacio de funciones.

\begin{proposition}
La operación $\langle \phantom{f},\phantom{g} \rangle : \mathcal{A}_r\times \mathcal{A}_r \longrightarrow \mathbb{R}$ definida como
\begin{equation*}
    \langle f,g \rangle = \sum_{d \mid r} \varphi(d) f \left( \frac{r}{d} \right) \overline{g \left( \frac{r}{d} \right)}
\end{equation*}
es un producto interno en $\mathcal{A}_r$.
\end{proposition}
\begin{proof}
Sean $c \in \mathbb{C}$ y $f,g,h \in \mathcal{A}_r$. Escríbase $f(n)=f_1(n)+ i f_2(n)$ para todo $n \in \mathbb{N}$, donde $f_1,f_2 \in \mathcal{A}_{\mathbb{R}}$. Entonces
\begin{enumerate}[label=\textnormal{(\roman*)}]
\item \begin{align*}
        \langle f,f \rangle & = \sum_{d \mid r} \varphi(d) \left[ f_1 \left( \frac{r}{d} \right) + i f_2 \left( \frac{r}{d} \right) \right] \overline{\left[ f_1 \left( \frac{r}{d} \right) + i f_2 \left( \frac{r}{d} \right) \right]} \\
                            & = \sum_{d \mid r} \varphi(d) \left[ f_1^2 \left( \frac{r}{d} \right) + f_2^2 \left( \frac{r}{d} \right) \right] \ge 0
\end{align*}
Además, si $\langle f,f \rangle=0$, como todos los términos de la suma anterior son positivos, se debe tener que $f_1(d)=f_2(d)=0$ para todo $d$ divisor de $r$. Si $n \in \mathbb{N}$ entonces $f_1(n)=f((n,r))=0$, pues $f$ es par $\Mod{r}$ y $(n,r) \mid r$. Análogamente se tiene $f_2(n)=0$. Luego $f(n)=f_1(n)+i f_2(n)=0$ para cada $n \in \mathbb{N}$, es decir $f=\mathbf{0}$.
\item \begin{align*}
        \langle f+g,h \rangle & = \sum_{d \mid r} \varphi(d) (f+g) \left( \frac{r}{d} \right) \overline{g \left( \frac{r}{d} \right)} = \sum_{d \mid r} \varphi(d) \left[ f \left( \frac{r}{d} \right) \overline{h \left( \frac{r}{d} \right)} + g \left( \frac{r}{d} \right) \overline{h \left( \frac{r}{d} \right)} \right] \\
                              & = \sum_{d \mid r} \varphi(d) f \left( \frac{r}{d} \right) \overline{h \left( \frac{r}{d} \right)} + \sum_{d \mid r} g \left( \frac{r}{d} \right) \overline{h \left( \frac{r}{d} \right)} = \langle f,h \rangle + \langle g,h \rangle
\end{align*}
\item \begin{equation*}
    \langle c f,g \rangle = \sum_{d \mid r} \varphi(d) c f \left( \frac{r}{d} \right) \overline{g \left( \frac{r}{d} \right)} = c \sum_{d \mid r} \varphi(d) f \left( \frac{r}{d} \right) \overline{g\left( \frac{r}{d} \right)} = c \langle f,g \rangle
\end{equation*}
\item \begin{equation*}
    \overline{\langle g,f \rangle} = \overline{\sum_{d \mid r} \varphi(d) g \left( \frac{r}{d} \right) \overline{f \left( \frac{r}{d} \right)}} = \sum_{d \mid r} \varphi(d) f \left( \frac{r}{d} \right) \overline{g \left( \frac{r}{d} \right)} = \langle f,g \rangle
\end{equation*}
\end{enumerate}
\end{proof}

El producto interno así definido podría haberse escrito sin el factor $\varphi(d)$ dentro de la suma. Su uso se justifica al usar la ecuación \eqref{eq:holder} para probar que este producto interno hace de $\mathcal{B}_r$ una base ortogonal de $\mathcal{A}_r$.

\begin{proposition}
Si $i,j$ son divisores positivos de $r$, entonces
\begin{equation*}
    \langle c_i,c_j \rangle = \begin{cases}
        \hfil r \varphi(j) & \text{si } i=j \\
        \hfil 0 & \text{en otro caso}.
    \end{cases}
\end{equation*}
\end{proposition}
\begin{proof}
En efecto,
\begin{equation*}
    \langle c_i,c_j \rangle = \sum_{d \mid r} \varphi(d) c_i \left( \frac{r}{d} \right) c_j \left( \frac{r}{d} \right) = \sum_{d \mid r} c_i \left( \frac{r}{d} \right) \varphi(j) c_d \left( \frac{r}{j} \right) = 
    \begin{cases}
        \hfil r \varphi(j) & \text{si } i=j \\
        \hfil 0 & \text{en otro caso},
    \end{cases}
\end{equation*}
donde la segunda igualdad se cumple por la \cref{eq:holder} y la tercera por el \Cref{lem:ram4}.
\end{proof}

\begin{corollary}
El conjunto
\begin{equation*}
    \mathcal{B}'_r = \left\{ c'_d = \frac{1}{\sqrt{r \varphi(d)}} c_d \right\}_{d \mid r}
\end{equation*}
es una base ortonormal de $\mathcal{A}_r$.
\end{corollary}

Como consecuencia del corolario anterior, toda función $f$ par $\Mod{r}$ debe tener una expansión de la forma $\sum_{e \mid r} \beta(e) c'_e$. Además, si $d$ es un divisor arbitrario de $r$, entonces
\begin{equation*}
    \left\langle f, c'_d \right\rangle = \left\langle \sum_{e \mid r} \beta(e) c'_e,c'_d \right\rangle = \sum_{e \mid r} \beta(e) \left\langle c'_e,c'_d \right\rangle = \beta(d),
\end{equation*}
pero $f$ también tiene una única representación de la forma $\sum_{e \mid r} \alpha(d) c_d$, en consecuencia, $\alpha(d) = \beta(d)/\sqrt{r \varphi(d)}$ para cada $d$ divisor de $r$. Por tanto,
\begin{align*}
    \alpha(d) & = \frac{\beta(d)}{\sqrt{r \varphi(d)}} = \frac{1}{\sqrt{r \varphi(d)}} \left\langle f, c'_d \right\rangle \\
                                                     & = \frac{1}{\sqrt{r \varphi(d)}} \sum_{e \mid r} \varphi(e) f \left( \frac{r}{e} \right) c'_d \left( \frac{r}{e} \right) \\
                                                     & = \frac{1}{\sqrt{r \varphi(d)}} \sum_{e \mid r} \varphi(e) f \left( \frac{r}{e} \right) \frac{1}{\sqrt{r \varphi(d)}} c_d \left( \frac{r}{e} \right) \\
                                                     & = \frac{1}{r \varphi(d)} \sum_{e \mid r} f \left( \frac{r}{e} \right) \varphi(d) c_e \left( \frac{r}{d} \right) \\
                                                     & = \frac{1}{r} \sum_{e \mid r} f \left( \frac{r}{e} \right) c_e \left( \frac{r}{d} \right),
\end{align*}
donde la cuarta igualdad se cumple nuevamente por \eqref{eq:holder}. Esta es la \cref{eq:ram6}. Como es usual en el análisis funcional, ahora se pueden llamar propiamente \emph{coeficientes de Fourier} a los coeficientes $\alpha$ del \Cref{thm:fou1}.

%%% Signal processing
\newpage
\section{Procesamiento de señales}

Una \emph{señal} es una descripción de un fenómeno que evoluciona en el tiempo o espacio; el \emph{procesamiento de señales} se refiere a cualquier operación manual o mecánica que modifique, analice o manipule de otra forma la información contenida en una señal. Considérese, por ejemplo, la temperatura ambiente; se puede medir su evolución coon el tiempo de muchas formas y los datos resultantes representan una ``señal'' de temperatura. Algunas operaciones sobre esta señal se pueden hacer incluso a mano, por ejemplo calcular la temperatura promedio en un mes o graficar la señal en una hoja de papel.

\bigskip

El adjetivo ``digital'' proviene de \emph{digitus}, palabra latina para dedo. En este contexto, se refiere a un paradigma en el que el mundo físico se puede describir usando únicamente números enteros.
El \emph{procesamiento digital de señales} es por tanto una rama del procesamiento de señales en la cual todo, incluido el tiempo, es descrito en términos de números enteros. En el procesamiento de digital de señales, la representación abstracta subyacente siempre es el conjunto de números naturales, independientemente de la naturaleza de la señal \cite{Prand1}.

\begin{definition}
Más específicamente, una \emph{señal} es cualquier función $x : \mathbb{Z} \longrightarrow \mathbb{C}$.
\end{definition}

\begin{remark}
Es necesario establecer las dos convenciones siguientes a lo largo del capítulo, siguiendo la notación estándar en la literatura de procesamiento de señales:
\begin{itemize}
\item El valor de una señal $x$ en un índice entero $n$ se denotará como $x[n]$, con corchetes en vez de paréntesis.
\item A partir de ahora se le denotará $j$ a la unidad imaginaria, es decir, $j = \sqrt{-1}$.
\end{itemize}
\end{remark}
% \subsection{Señales simétricas}

En este capítulo se estudiará una clase particular de señales periódicas, llamadas señales \emph{simétricas} o \emph{pares}, con las herramientas del capítulo anterior. Estas señales no son más que funciones aritméticas pares definidas en los enteros.

\subsection{Transformada Discreta de Fourier}

\begin{definition}[Señal periódica]
Desde luego, una señal periódica con periodo $r \in \mathbb{N}$ es una señal para la cual $x[n] = x[n + k r]$ para cada $n,k \in \mathbb{Z}$.
\end{definition}

Una señal periódica con periodo $r$ contiene toda su información en un periodo, en el cuál toma $r$ valores complejos. Considérese pues un arreglo con estos valores, es decir, un elemento de $\mathbb{C}^r$.
\bigskip

Sea $W_r = e^{-j(2 \pi / r)}$ y considérese el producto interno $\langle x,y \rangle = \sum_{n=1}^{r} \overline{x[n]} y[n]$ en $\mathbb{C}^r$. Se sabe, véase por ejemplo \cite{Prand1}, que el conjunto $\{ w_k \}_{k=1}^r,$ donde \begin{equation*}
    w_k = (W_r^{-k}, W_r^{-2 k}, \ldots, W_r^{-(r-1) k}, 1) \in \mathbb{C}^r, w=1,\ldots,r,
\end{equation*}
es una base ortogonal de $\mathbb{C}^r$. En efecto,
\begin{equation*}
    \langle w_m, w_n \rangle = \sum_{i=1}^{r} \overline{W_r^{-m i}} W_r^{-n i} = \sum_{i=1}^{r} W_r^{(m-n)i} = \begin{cases}
        \hfil r & \text{si } m=n \\[7pt]
        \hfil \displaystyle\frac{1-W_r^{(m-n)r}}{1-W_r^{m-n}} = 0 & \text{en otro caso}.
    \end{cases}
\end{equation*}

Por otro lado, si $x \in \mathbb{C}^r$, existen por tanto $X(k) \in \mathbb{C}, k=1,\ldots,r$, tales que $r x = \sum_{k=1}^{r} X[k] w_k$. Pero
\begin{equation*}
    \langle w_k, x \rangle = \left\langle w_k, \frac{1}{r} \sum_{n=1}^{r} X[n] w_n \right\rangle = \frac{1}{r} \sum_{n=1}^{r} \left\langle w_k, X[n] w_n \right\rangle = \frac{X[k]}{r} r = X[k].
\end{equation*}

En consecuencia,
\begin{equation}\label{eq:analysis}
    X[k] = \sum_{n=1}^{r} \overline{W_r^{-n k}} x[n] = \sum_{n=1}^{r} x[n] W_r^{n k}, \forall k=1,\ldots,r
\end{equation}
y
\begin{equation}\label{eq:synthesis}
    x[n] = \frac{1}{r} \sum_{k=1}^{r} X[k] w_k[n] = \frac{1}{r} \sum_{k=1}^{r} X[k] W_r^{- n k}, \forall n=1,\ldots,r.
\end{equation}

Las ecuaciones \eqref{eq:analysis} y \eqref{eq:synthesis} se conocen como las fórmulas de \emph{análisis} y \emph{síntesis} de la \emph{Transformada Discreta de Fourier} del elemento $x \in \mathbb{C}^r$, respectivamente.
\bigskip

Considere la ecuación de síntesis \eqref{eq:synthesis}. Dado que $W_r^{(n + i r) k} = W_r^{n k}$ para cada $i \in \mathbb{Z}$, entonces $x[n + i r] = x[n]$ para todos $i \in \mathbb{Z}$, $n=1,\ldots,r$, de manera que se puede extender $x \in \mathbb{C}^r$ a una señal periódica a todo $\mathbb{Z}$ de forma natural.
\bigskip

Si ahora $\tilde{x}$ es una señal periódica con periodo $r$ se pueden escribir, gracias a la observación anterior y a las fórmulas de análisis y síntesis,
\begin{equation}\label{eq:analysis1}
    \tilde{X}[k] = \sum_{n=1}^{r} \overline{W_r^{-n k}} \tilde{x}[n] = \sum_{n=1}^{r} \tilde{x}[n] W_r^{n k}, \forall k \in \mathbb{Z}
\end{equation}
y
\begin{equation}\label{eq:synthesis2}
    \tilde{x}[n] = \frac{1}{r} \sum_{k=1}^{r} \tilde{X}[k] w_k[n] = \frac{1}{r} \sum_{k=1}^{r} \tilde{X}[k] W_r^{- n k}, \forall n \in \mathbb{Z}.
\end{equation}

La representación \eqref{eq:synthesis2} de la señal $\tilde{x}$ se llama \emph{Serie Discreta de Fourier} y sus fórmulas de análisis y síntesis son idénticas a las anteriores para elementos de $\mathbb{C}^r$, con la diferencia de que los índices son válidos en todo el dominio $\mathbb{Z}$.

\subsection{Señales simétricas}

Las señales simétricas son un análogo a las funciones aritméticas pares estudiadas en el capítulo anterior.

\begin{definition}[Señal simétrica]
Una señal $x$ se dice \emph{simétrica módulo} $r$ si $x[n] = x[(n,r)]$ para cada $n \in \mathbb{Z}$.
\end{definition}

Toda señal simétrica módulo $r$ es periódica módulo $r$ y la demostración es idéntica a la de la \Cref{prop:mod->per}.

%%% Divisibilidad
\newpage
\appendix
\section{Divisibilidad}

\begin{proposition}
Si $r \in \mathbb{N}$, $r=e q_1$, $r= d q_2$, $d=(q_1,d)k_1$ y $e=(q_2,e) k_2$ entonces $k_1=k_2$.
\end{proposition}
\begin{proof}
Nótese que $q_1,q_2$ son enteros positivos y claro que $(q_1 q_2, r) = (q_1 q_2, r)$, luego $(q_1 q_2,e q_1)=(q_1 q_2,d q_2)$ y por tanto $q_1 (q_2,e)=q_2 (q_1,d)$. En consecuencia, $(q_2,e)k_2 q_1 = e q_1 = r = d q_2 = (q_1,d)k_1 q_2$ y por la ley de cancelación se debe tener que $k_1 = k_2$.
\end{proof}

\begin{corollary}\label{cor:mcd1}
Si $r \in \mathbb{N}$, $e \mid r$ y $d \mid r$ con $e,r \in \mathbb{N}$, entonces
\begin{equation*}
    d/\left( r/e,d \right) = e/\left( r/d,e \right).
\end{equation*}
\end{corollary}

\begin{theorem}[Lema de Euclides]
Si $a \mid bc$ y $(a,b)=1$, entonces $a \mid c$.
\end{theorem}
\begin{proof}
Si $(a,b)=1$, podemos escribir $1=a s+b t$, donde $s,t\in \mathbb{Z}$. Luego $c=a(s c)+b c(t)$ y como $a \mid a$ y $a \mid bc$ por hipótesis, entonces $a \mid c$.
\end{proof}

%%% Códigos
\newpage
\section{Códigos}
{
\renewcommand\ttdefault{cmtt}
\begin{adjustwidth}{12mm+2mm}{2mm}
\begin{lstlisting}
import Numpy as numpy
print("Hello, world!")
\end{lstlisting}
\end{adjustwidth}
}


\newpage
\phantomsection
\nocite{*}
\bibliography{bibliography}
\bibliographystyle{acm}
\end{document}

\end{document}
