%%% Funciones pares
\newpage
\thispagestyle{empty}
\
\newpage
\section{Funciones pares}

Al estudiar el espacio de funciones aritméticas se puede hacer una analogía con la teoría de Fourier del análisis para funciones periódicas, al considerar la representación de una función periódica como suma de funciones sinusoidales. En este caso, estas funciones osciladoras serán las sumas de Ramanujan $c_r$ en vez de las funciones $\sin$ y $\cos$ del análisis real. Se considerarán las clases de funciones aritméticas pares y periódicas, siendo la primera una subclase propia de la segunda y se probará la existencia de dicha representación solo para las funciones pares.

\begin{remark}
Durante todo el capítulo se supondrá que $r$ es un entero positivo arbitrario pero fijo.
\end{remark}

\begin{definition}[Función par]\label{def:even}
Una función aritmética se dice \textbf{par} $\Mod{r}$ si $f(n)=f((n,r))$, donde $(m,r)$ es el máximo común divisor de $n$ y $r$, para cada $n \in \mathbb{N}$.
\end{definition}

\begin{definition}[Función periódica]
Una función aritmética se dice \textbf{periódica} con periodo $r$ (o periódica $\Mod{r}$) si $m, n \in \mathbb{N}$ y $m \equiv n \pmod{r}$ implica que $f(m)=f(n)$.
\end{definition}

La siguiente proposición es una consecuencia inmediata de las definiciones anteriores.

\begin{proposition}\label{prop:mod->per}
Toda función par $\Mod{r}$ es periódica con periodo $r$.
\end{proposition}
\begin{proof}
Si $m \equiv n \pmod{r}$ entonces $r \mid m-n$, por tanto existe $q \in \mathbb{Z}$ tal que $m-n=q r$. Por demostrar que $(n,r)=(m,r)$. En efecto, como $(n,r) \mid n$ y $(n,r) \mid r$, entonces $(n,r) \mid n+qr=m$, luego $(n,r) \mid (m,r)$. Análogamente, se tiene que $(m,r) \mid (n,r)$. Se sigue que $(n,r)=(m,r)$ y por tanto $f(n)=f((n,r))=f((m,r))=f(m)$.
\end{proof}

\subsection{Sumas de Ramanujan}

En 1918, Ramanujan publicó el artículo \cite{Ram1}, que contiene varias fórmulas notables expresando algunas funciones sobre $\mathbb{N}$ como el límite puntual de ciertas series trigonométricas. Por ejemplo, probó que
\begin{align*}
    \frac{\sigma(n)}{n} = \frac{\pi^2}{6} \bigg( 1 & + \frac{(-1)^n}{2^2} + \frac{2 \cos \left( \frac{2}{3} \pi n \right)}{3^2} + \frac{2 \cos \left( \frac{1}{2} \pi n \right)}{4^2} \\
                                                   & + \frac{2 \left[ \cos \left( \frac{2}{5}\pi n \right) + \cos \left( \frac{4}{5} \pi n \right) \right]}{5^2} + \frac{2 \cos \left(  \frac{1}{3} \pi n \right)}{6^2} + \cdots \bigg),
\end{align*}
o escrito de otra manera,
\begin{equation*}
    \frac{\sigma(n)}{n} = \frac{\pi^2}{6} \sum_{r=1}^{\infty} \frac{c_r(n)}{r^2},
\end{equation*}
donde
\begin{equation*}
    c_r(n) = \sum_{\substack{a=1 \\ (a,r)=1}}^{r} \cos \left( \frac{2 \pi}{r} a n \right).
\end{equation*}
Estas sumas son conocidas como \textrm{sumas de Ramanujan}. El valor $c_r(n)$ es de hecho la suma de las $n$-ésimas potencias de las raíces primitivas de la unidad.

\thispagestyle{easter1}
\begin{proposition}
Para toda $n \in \mathbb{N}$ se tiene que
\begin{equation*}
    c_r(n) = \sum_{\substack{a=1 \\ (a,r)=1}}^{r} e^{i(2 \pi /r) a n},
\end{equation*}
donde $i$ es la unidad imaginaria.
\end{proposition}

\begin{proof}
En efecto,
\begin{align*}
    \sum_{\substack{a=1 \\ (a,r)=1}}^{r} e^{i(2 \pi / r) a n} = \sum_{\substack{a=1 \\ (a,r)=1}}^{r} \cos \left( \frac{2 \pi}{r} a n \right) + i \sum_{\substack{a=1 \\ (a,r)=1}}^{r} \sin \left( \frac{2 \pi}{r} a n \right).
\end{align*}
Pero $(a,r)=1$ si y sólo si $(a-r,r)=1$. Además, si $1 \le a \le r-1$ entonces $1 \le r-a \le r-1$, por tanto
\begin{align*}
    \sum_{\substack{a=1 \\ (a,r)=1}}^{r} \sin \left( \frac{2 \pi}{r} a n \right) = \sum_{\substack{a=1 \\ (r-a,r)=1}}^{r} \sin \left( \frac{2 \pi}{r} (r-a) n \right) & = \sum_{\substack{a=1 \\ (r-a)=1}}^{r} - \sin \left( \frac{2 \pi}{r} a n \right) \\
                                                                                 &= - \sum_{\substack{a=1 \\ (a,r)=1}}^{r} \sin \left( \frac{2 \pi}{r} a n \right).
\end{align*}
Se sigue que la parte imaginaria de la primera suma se anula y se tiene la igualdad deseada.
\end{proof}

En el siguiente capítulo se verá que esta expresión aparece de forma natural al calcular los coeficientes de Fourier clásicos de cierta función aritmética.

\begin{proposition}\label{prop:ram0}
Para toda $n \in \mathbb{N}$ se tiene que
\begin{equation*}
    c_r(n) = \sum_{d \mid (n,r)} \mu \left( \frac{r}{d} \right) d.
\end{equation*}
\end{proposition}

\begin{proof}
Por la \Cref{prop:mob2} se tiene que la suma $\sum_{d \mid k} \mu(d)$ es igual a $1$ si $k=1$ e igual a 0 en otro caso, luego por la proposición anterior,
\begin{align*}
    c_r(n) = \sum_{\substack{a=1 \\ (a,r)=1}}^{r} e^{i(2 \pi/r) a n} & = \sum_{a=1}^{r} e^{i(2 \pi / r) a n} \left( \sum_{d \mid (a,r)} \mu(d) \right) = \sum_{\substack{d \mid a \\ d \mid r}} \sum_{a=1}^{r}  \mu(d) e^{i(2 \pi / r) a n} \\
           & = \sum_{d \mid r} \mu(d) \sum_{\substack{d \mid a \\ 1 \le a \le r}} e^{i(2 \pi / r) a n} = \sum_{d \mid r} \mu(d) \sum_{b=1}^{r/d} e^{i (2 \pi / (r/d)) b n} \\
\end{align*}
El valor de la suma interior depende de si $r/d$ divide a $n$ o no. Si lo hace, cada uno de sus términos es igual a 1 y por tanto la suma vale $r/d$. Si este no es el caso se puede sumar,
\begin{align*}
    \sum_{b=1}^{r/d} e^{i (2 \pi / (r/d)) b n} = \frac{e^{i(2 \pi / r) / (r/d)} (e^{i(2 \pi n)} - 1)}{e^{i(2 \pi / r) / (r/d)} - 1} = 0,
\end{align*}
pues el denominador no se anula. Luego
\begin{align*}
    c_r (n) = \sum_{\substack{a=1 \\ (a,r)=1}}^{r} e^{i(2 \pi/r) a n} = \sum_{d \mid r} \mu(d) \begin{cases}
               \hfil r/d & \text{si } r/d \mid n \\
               \hfil 0 & \text{en otro caso}
           \end{cases} = \sum_{\substack{d \mid q \\ (r/d) \mid n}} \mu(d) (r/d) \\
           = \sum_{\substack{d \mid r \\ d \mid n}} \mu \left( \frac{r}{d} \right) d = \sum_{d \mid (n,r)} \mu \left( \frac{r}{d} \right) d.
\end{align*}
\end{proof}

En lo que sigue de este capítulo se trabajará exclusivamente con ésta última representación de $c_r$ y se le referirá como \emph{suma de Ramanujan módulo} $r$ o simplemete \emph{suma de Ramanujan} cuando no haya riesgo de confusión.

\begin{proposition}
Algunas propiedades de la sumas de Ramanujan son las siguientes:
\begin{enumerate}[label=\textnormal{(\arabic*)},ref=\textnormal{\arabic*}]
\item $c_1 = \mathbf{1}$
\item $c_r(1) = \mu(r)$
\item $c_r(n) \le \max \{ \sigma(r), \sigma(n) \}$
\item \label{it:ram1} $c_r(n)$ es una función multiplicativa de $r$
\item Si $p$ es primo y $m$ es un entero positivo, entonces
    \begin{equation*}
        c_{p^m}(n) = \begin{cases}
            \hfil p^m - p^{m-1} & \text{si } p^m \mid n \\
            \hfil -p^{m-1} & \text{si } p^{m-1} \text{ pero } p^m \centernot\mid n \\
            \hfil 0 & \text{si } p^{m-1} \centernot\mid n.
        \end{cases}
    \end{equation*}
\end{enumerate}
\end{proposition}

\begin{proof}
\begin{enumerate}[label=\textnormal{(\arabic*)}]
\item Para cada $n \in \mathbb{N}$ se tiene que $(n,1)=1$ y por tanto
    \begin{equation*}
        c_1(n) = \sum_{d \mid (n,1)} \mu \left( \frac{1}{d} \right) d = \mu(1) 1 = 1.
    \end{equation*}
\item De manera similar,
    \begin{equation*}
        c_r(1) = \sum_{d \mid (1,r)} \mu \left( \frac{r}{d} \right) d = \mu(r) 1 = \mu(r).
    \end{equation*}
\item Por definición se tiene que $\sigma(k) = \sum_{d \mid k} d$. Además $\mu(k) \le 1$ para todo $k \in \mathbb{N}$, luego
    \begin{equation*}
        c_r(n) = \sum_{d \mid (n,r)} \mu \left( \frac{r}{d} \right) d \le \sum_{d \mid (n,r)} d = \sum_{\substack{d \mid n \\ d \mid r}} d \le \sum_{d \mid n} d, \sum_{d \mid r} d \le \max \{ \sigma(n),\sigma(r) \}.
    \end{equation*}
\item Defínase
    \begin{equation*}
        \eta_r(n) = \begin{cases}
            \hfil r & \text{si } r \mid n \\
            \hfil 0 & \text{en otro caso}.
        \end{cases}
    \end{equation*}
Se tiene que la función $\eta_\Box(n)$ es multiplicativa para $n$ fijo. En efecto, si $r,s \in \mathbb{N}$ son tales que $(r,s)=1$, entonces
\begin{equation*}
    \eta_{r s}(n) = \begin{cases}
        \hfil r s & \text{si } r s \mid n \\
        \hfil 0 & \text{en otro caso,}
    \end{cases}
\end{equation*}
pero $r s \mid n$ si y sólo si $r \mid n$ y $s \mid n$. En efecto, si $r s \mid n$ es claro que $r \mid n$ y $s \mid n$. Supóngase que $r \mid n$ y $s \mid n$, de tal manera que existen $q_1, q_2 \in \mathbb{Z}$ tales que $n=r q_1=s q_2$. Como $(r,s)=1$, también existen $x, y \in \mathbb{Z}$ tales que $1=r x + s y$, luego $n=n r x + n s y$, por lo que $n= r s (q_2 x + q_1 y)$, es decir, $r s \mid n$. Luego, si $r s \mid n$, entonces
\begin{equation*}
    \eta_{r s}(n) = r s = \eta_r(n) \eta_s(n),
\end{equation*}
y si $r s \centernot\mid$ entonces $r \centernot\mid n$ y $s \centernot\mid n$, por lo que
\begin{equation*}
    \eta_{r s}(0) = 0 = \eta_r(n) \eta_s(n).
\end{equation*}

Por otro lado, se tiene que
\begin{equation*}
    \sum_{d \mid r} \mu \left( \frac{r}{d} \right) \eta_d(n) = \sum_{\substack{d \mid r \\ d \mid n}} \mu \left( \frac{r}{d} \right) d = \sum_{d \mid (n,r)} \mu \left( \frac{r}{d} \right) d = c_r(n),
\end{equation*}
es decir, $c_\Box(n)=\mu*\eta_\Box(n)$. Luego $c_\Box(n)$ debe ser multiplicativa para $n$ fijo, por ser producto de funciones multiplicativas.
\item Se tienen los siguientes casos
\begin{itemize}
\item Si $p^m \mid n$, entonces $(n,p^m)=p^m$, luego
    \begin{equation*}
        c_{p^m}(n) = \sum_{d \mid p^m} \mu \left( \frac{p^m}{d} \right) d = \mu(1) p^m + \mu(p) p^{m-1} = p^m - p^{m-1},
    \end{equation*}
    pues $\mu(p^i)=0$ para toda $i > 1$.
\item Si $p^{m-1} \mid n$ pero $p^m \centernot\mid n$, entonces $(n,p^m)=p^{m-1}$. En efecto, se tiene que $p^{m-1} \mid p^m$ y además $p^{m-1} \mid n$ por hipótesis. Si $e \in \mathbb{Z}$ es tal que $e \mid p^{m}$ y $e \mid n$, entonces $e=p^i$, para algún $0 \le i \le m-1$, pues $p^m \centernot\mid n$, por tanto $e \mid p^{m-1}$. Esto prueba que $(p^m,n)=p^{m-1}$, así
    \begin{equation*}
        c_{p^m}(n) = \sum_{d \mid p^{m-1}} \mu \left( \frac{p^m}{d} \right) d = \mu(p) p^{m-1} = -p^{m-1}
    \end{equation*}
    
\item Finalmente, si $p^{m-1} \centernot\mid n$, entonces $p^m \centernot\mid n$. Además, $(n,p^m) \mid p^m$, por tanto $(n,p^m)=p^i$ para algún $0 \le i \le m$. Más aún, por la hipótesis se debe tener que $0 \le i \le m-2$. Luego
    \begin{equation*}
        c_{p^m}(n) = \sum_{d \mid p^i} \mu \left( \frac{p^m}{d} \right) d = \mu(p^m)1+ \mu(p^{m-1}) p + \cdots + \mu(p^{m-i}) p^i = 0,
    \end{equation*}
    pues $i \le m-2$ implica que $2 \le m-i$ y por tanto $\mu(p^m)=\ldots=\mu(p^{m-i})=0$.
\end{itemize}
\end{enumerate}
\end{proof}

Del la demostración del \cref{it:ram1} se puede rescatar el siguiente corolario, usando la inversión de Möbius (\Cref{cor:mob1}).

\begin{corollary}\label{cor:ram5}
Para cada $n \in \mathbb{N}$ fijo se tiene
\begin{equation*}
    \sum_{d \mid r} c_d(n) = \eta_r(n) = \begin{cases}
        \hfil r & \text{si } r \mid n \\
        \hfil 0 & \text{en otro caso}.
    \end{cases}
\end{equation*}
\end{corollary}

Otro corolario notable de la proposición anterior es el siguiente.

\begin{corollary}
La suma de las raíces $r$-ésimas primitivas de la unidad es igual a $\mu(r)$.
\end{corollary}

\begin{proof}
Por el punto (2) de la proposición anterior, se tiene que
\begin{equation*}
    \sum_{\substack{a=1 \\ (a,r)=1}}^{r} e^{i(2 \pi / r) a} = c_r(1)  = \mu(r).
\end{equation*}
\end{proof}

Las sumas de Ramanujan gozan de la siguiente propiedad de ``ortogonalidad''.

\begin{lemma}
Si $r$ y $s$ dividen a $k$, entonces
\begin{equation*}
    \sum_{d \mid k} c_r(k/d) c_d(k/s) = \begin{cases}
        \hfil k & \text{si } r = s \\
        \hfil 0 & \text{en otro caso}.
    \end{cases}
\end{equation*}
\end{lemma}
\begin{proof}
Si $r$ y $s$ dividen a $k$, entonces
\begin{equation} \label{eq:ram2}
\begin{split}
\sum_{d \mid k} c_r (k/d) c_d (k/s) &= \sum_{d \mid k} c_d (k/s) \sum_{d' \mid (k/d,r)} \mu (r/d') d' \\
                                                                              &= \sum_{d \mid k} c_d (k/s) \sum_{\substack{d' \mid r \\ d' \mid k/d}} \mu(r/d') d' = \sum_{\substack{d \mid k \\ d' \mid r \\ d' \mid k/d}} c_d (k/s) \mu (r/d') d' \\
                                                                              &= \sum_{\substack{d \mid k/d' \\ d' \mid r \\ d' \mid r}} c_d(k/s) \mu(r/d')  d' = \sum_{\substack{d' \mid r \\ d' \mid k}} \mu(r/d') d' \sum_{d \mid k/d'} c_d(k/s) \\
                                                                              &= \sum_{d' \mid (k,r)} \mu(r/d) d' \eta_{k/d'} (k/s) = \sum_{d' \mid r} \mu(r/d) d' \eta_{k/d'} (k/s),
\end{split}
\end{equation}
dado que $(k,r)=r$ por ser $r$ divisor de $k$ y dado que los conjuntos $\{ d,d' \in \mathbb{N} \std d \mid k, d' \mid r, d' \mid k/d \}$ y $\{ d,d' \in \mathbb{N} \std d \mid k/d', d' \mid r, d' \mid k \}$ son iguales. En efecto, si $d \mid k$ entonces $k/d$ es un entero, lueg $d' \mid k/d$ implica que $k/d=d' q'$, luego $k=d' q' d$, por tanto $d \mid k/d'$ y $d' \mid k$.
\bigskip

Recíprocamente, si $d' \mid k$ entonces $k/d'$ es un entero, luego $d \mid k/d'$ implica que $k/d'=dq$, por tanto $k=d q d'$, por tanto $d \mid k$ y $d' \mid k/d$.
\bigskip

Si $s \centernot\mid r$ entonces $s \centernot\mid d'$ y por tanto $k/d' \centernot\mid k/s$. En efecto, pues si $s \mid d'$, como $d' \mid r$ entonces se tendría que $s \mid r$ por transitividad. Además, si $k/d' \mid k/s$ se tendría que $s \mid d'k$. Luego la suma \eqref{eq:ram2} se anula si $s \centernot\mid r$ y en particular si $r \ne s$, pues en este caso se tiene que $\eta_{k/d'}(k/s)=0$ para cada $d' \mid r$. 
\bigskip

Si $s \mid r$ entonces la suma \eqref{eq:ram2} es igual a
\begin{equation*}
\begin{split}
    \sum_{\substack{d' \mid r \\ k/d' \mid k/s}} \mu(r/d') d' \frac{k}{d'} &= \sum_{\substack{d' \mid r \\ k/d' \mid k/s}} \mu(r/d') k = \sum_{\substack{d' \mid r \\ s \mid d'}} \mu(r/d') k \\
    &= k \sum_{\substack{d' \mid r \\ d'=se}} \mu(r/se) = k \sum_{e \mid r/s} \mu(r/se) \\
    &= k \sum_{se \mid r} \mu(r/se) = \begin{cases}
        \hfil k & \text{si } r=s \\
        \hfil 0 & \text{en otro caso,}
    \end{cases}
\end{split}
\end{equation*}
pues $k/d' \mid k/s$ si y sólo si $s \mid d'$.
\end{proof}

\begin{corollary}\label{cor:ind}
Las sumas de Ramanujan son linealmente independientes respecto a la suma sobre los divisores de $r$. Más específicamente, si $\alpha, \beta$ son funciones aritméticas tales que
\begin{equation*}
    \sum_{d \mid r} \alpha(d) c_d(n) = \sum_{d \mid r} \beta(d) c_d(n),
\end{equation*}
para todo $n \in \mathbb{N}$, entonces $\alpha(d)=\beta(d)$ para todo $d \mid r$.
\end{corollary}
\begin{proof}
Basta probar que $\sum_{d \mid r} \alpha(d) c_d(n)=0$ implica que $\alpha(d)=0$ para todo $d \mid r$. Supóngase la hipótesis y sea $\delta \mid r$ arbitrario pero fijo. Si $e$ es un divisor de $r$, se tiene que $\sum_{d \mid r} \alpha(d) c_d(r/e)=0$, luego
\begin{align*}
    0 = \sum_{e \mid r} \left( \sum_{d \mid r} \alpha(d) c_d \left( \frac{r}{e} \right) \right) c_e \left( \frac{r}{\delta} \right)= \sum_{d \mid r} \alpha(d) \sum_{e \mid r} c_d \left( \frac{r}{e} \right) c_e \left( \frac{r}{\delta} \right) = \alpha(\delta) r,
\end{align*}
por el la proposición anterior, y dado que $r \ne 0$ entonces $\alpha(\delta)=0$. Como $\delta$ fue un divisor arbitrario de $r$, se tiene el resultado.
\end{proof}

\begin{lemma}
Si $d \mid r$ entonces $c_d(n)=c_d((n,r))$.
\end{lemma}
\begin{proof}
Si $d \mid r$ entonces $(n,d)=((n,r),d)$. En efecto, dado que $(n,d) \mid n$ y $(n,d) \mid d$, entonces $(n,d) \mid n$, $(n,d) \mid d$ y $(n,d) \mid r$, por lo que $(n,d) \mid (n,r)$ y $(n,d) \mid d$, es decir, $(n,d) \mid ((n,r),d)$. Recíprocamente se tiene que $((n,r),d) \mid n$ y $((n,r),d) \mid d$, así que $((n,r),d) \mid (n,d)$. Se sigue que $(n,d)=((n,r),d)$. Luego
\begin{equation*}
    c_d(n) = \sum_{e \mid (n,d)} \mu(d/e) e = \sum_{e \mid ((n,r),d)} \mu(d/e) e = c_d((n,r)).
\end{equation*}
\end{proof}

\begin{corollary}
La suma de Ramanujan módulo $r$ es par $\Mod{r}$.
\end{corollary}

\begin{definition}[Radical]
Sea $n \in \mathbb{N}$. Se define el \emph{radical} de $n$, denotado por $n_*$ como
\begin{equation*}
    n_* = \begin{cases}
        \hfil 1 & \text{si } n=1 \\
        \hfil p_1 \cdots p_r & \text{si } n = p_1^{\alpha_1} \cdots p_r^{\alpha_r}
    \end{cases}
\end{equation*}
donde $n=p_1^{\alpha_1} \cdots p_r^{\alpha_r}$ es la factorización de $n>1$ en primos.
\end{definition}

\begin{definition}
Una función aritmética $f$ se dirá \emph{separable} si $f(n)=f(n_*)$, para cada $n \in \mathbb{N}$.
\end{definition}

\begin{lemma}
Una función multiplicativa es separable si y sólo si $(\mu * f)(n)=0$ para todo $n$ no libre de cuadrado.
\end{lemma}
\begin{proof}
Sea $F=\mu * f$. Entonces $F*\mathbf{1}=f$, es decir,
\begin{equation*}
    \sum_{d \mid n} F(d) = f(n), \forall n \in \mathbb{N}.
\end{equation*}
Si $F(n)=0$ para cada $n$ no libre de cuadrado, entonces
\begin{equation*}
    f(n) = \sum_{d \mid n} F(d) = \sum_{d \mid n_*} F(d) = f(n_*),
\end{equation*}
es decir, $f$ es separable.
\bigskip

Supóngase ahora que $f$ es separable. Se tiene que para cada primo $p$ y para cada $m>1$,
\begin{align*}
    F(p^m) = \sum_{d \mid p^m} \mu(d) f \left( \frac{p^m}{d} \right) &= \mu(1)f(p^m) + \mu(p)f(p^{m-1}) \\
                                                                     &= f(p^m) - f(p^{m-1}) = f(p) - f(p) = 0.
\end{align*}
Además como $f$ es multiplicativa, entonces $F$ también lo es. Si $n$ es un entero positivo no libre de cuadrado, entonces existen un primo $p$ y enteros positivos $q$ y $m>1$ tales que $n=p^m q$ y $(p^m,q)=1$. Luego $F(n)=F(p^m)F(q)=0 \cdot F(q)=0$.
\end{proof}

\begin{lemma}\label{lem:par0}
Si $f$ es multiplicativa y separable, entonces para cualesquiera $a,b \in \mathbb{N}$ se tiene:
\begin{enumerate}[label=\textnormal{(\roman*)}]
\item $f(a)f(b)=f(a b)f((a,b))$.
\item $\displaystyle f(a) = f((a,b)) \sum_{\substack{d \mid a \\ (d,b) = 1}} (\mu * f)(d)$
\end{enumerate}
\end{lemma}

\begin{proof}
(\textsc{\romannumeral 1}) Nótese que si $p$ es un primo y $m,n > 1$ entonces
\begin{equation*}
    f(p^m)f(p^n) = f(p) f(p) = f(p^{m+n})f((p^m,p^n)),
\end{equation*}
pues $(p^m,p^n)=p^i$, con $i=\min \{ m,n \}$. Sean $a,b \in \mathbb{N}$ y escríbase sin pérdida de generalidad $a=p_1^{\alpha_1} \cdots p_r^{\alpha_r}$ y $b=p_1^{\beta_1} \cdots p_r^{\beta_r}$, $0 \le \alpha_i, \beta_i$. Entonces, como $f$ es multiplicativa,
\begin{align*}
    f(a b)f((a,b)) &= f(p_1^{\alpha_1+\beta_1} \cdots p_r^{\alpha_r+\beta_r}) f(p_1^{\min \{ \alpha_1,\beta_1 \}} \cdots p_r^{\min \{ \alpha_r,\beta_r \}}) \\
                   &= f(p_1^{\alpha_1+\beta_1}) \cdots f(p_r^{\alpha_r+\beta_r}) f(p_1^{\min \{ \alpha_1,\beta_1 \}}) \cdots f(p_r^{\min \{ \alpha_r,\beta_r \}}) \\
                   &= f(p_1^{\alpha_1+\beta_1}) \cdots f(p_r^{\alpha_r+\beta_r}) f((p_1^{\alpha_1},p_1^{\beta_1})) \cdots f((p_r^{\alpha_r},p_r^{\beta_r})) \\
                   &= f(p_1^{\alpha_1})f(p_1^{\beta_1}) \cdots f(p_r^{\alpha_r})f(p_r^{\beta_r}) \\
                   &= f(p_1^{\alpha_1} \cdots p_r^{\alpha_r})f(p_1^{\beta_1} \cdots p_r^{\beta_r}) \\
                   &= f(a)f(b)
\end{align*}
(\textsc{\romannumeral 2}) Al igual que en la demostración anterior, si $F=\mu * f$, entonces
\begin{equation*}
    \sum_{d \mid n} F(d) = f(n), \forall n \in \mathbb{N}.
\end{equation*}

Se verá primero que los conjuntos $\{ d \in \mathbb{N} \std d \mid a_* \text{ y } (d,b)=1\}$ y $\{ d \in \mathbb{N} \std d \mid a_* / (a,b)_* \}$ son iguales.
\bigskip

Para empezar, se tiene que $a_*/(a,b)_*$ es un entero. Si $(a,b)_*=1$ esto es claro. Si $(a,b)_*>1$ se puede escribir $(a,b)_*=q_1 \cdots q_s$, donde todos los primos son distintos. Luego $q_i \mid (a,b)_*$, pero $(a,b)_* \mid (a,b)$ y $(a,b) \mid a$, por tanto $q_i \mid a$ y por tanto $q_i \mid a_*$. Como $i \in \{ 1,\ldots,s \}$ fue arbitrario y todos los primos $q_i$ son distintos, entonces $q_1 \cdots q_s = (a,b)_* \mid a_*$, que es lo que se quería probar.
\bigskip

Procedamos a probar la igualdad de los conjuntos. Supóngase primero que $d \mid a_*$ y $(d,b)=1$. Entonces existe $c \in \mathbb{N}$ tal que $a_*=d c$. Por otro lado, se tiene que $(a,b) \mid b$ y por tanto $((a,b),d)=1$, más aún, como $(a,b)_* \mid (a,b)$ entonces también $((a,b)_*,d)=1$ y como $(a,b)_* \mid a_* = d c$, por el lema de Euclides se debe tener que $(a,b)_* \mid c$ es decir, $a_* = (a,b)_* d q$, para algún $q \in \mathbb{N}$, luego $d \mid a_* / (a,b)_*$.
\bigskip

Recíprocamente, supóngase que $d \mid a_*/(a,b)_*$. Se debe tener que
\begin{equation}\label{eq:mcd2}
    \left( \frac{a_*}{(a,b)_*},b \right) = 1.
\end{equation}
Pues en caso contrario, es decir, si este máximo común divisor fuera mayor que uno, existiría un primo $p$ tal que $p \mid b$ y $p \mid a_*/(a,b)_*$, pero $a_*/(a,b)_* \mid a_*$, luego $p \mid a_*$ y por tanto $p \mid a$. En consecuencia, $p \mid (a,b)$ y por tanto $p \mid (a,b)_*$. Se puede escribir entonces $a_*=p p_1 \cdots p_r$, $(a,b)_*= p q_1 \cdots q_s$, donde todos los primos son distintos. Además, como $a_*=(a,b)_* n$ para algún $n \in \mathbb{N}$, se tiene que
$p p_1 \cdots p_r = p q_1 \cdots q_s r_1 \cdots r_t$, con $n=r_1 \cdots r_t$, y $r_i$ números primos, no necesariamente distintos. Luego $p_1 \cdots p_r = q_1 \cdots q_s r_1 \cdots r_t$ y dado que ningúno de los primos $p_i$ son iguales a $p$, entonces ninguno de los primos $r_j$ puede ser igual a $p$, es decir $p$ no divide a $n=a_*/(a,b)_*$, lo cual es absurdo.
\bigskip

Esto prueba la igualdad de dichos conjuntos. Ahora es fácil calcular la siguiente suma,
\begin{equation*}
    \sum_{\substack{d \mid a \\ (d,b)=1}} F(d) = \sum_{\substack{ d \mid a_* \\ (d,b)=1}} F(d) = \sum_{d \mid a_* / (a,b)_*} F(d) = \sum_{d \mid (a_*/(a,b)_*)} (\mu * f)(d) = f(a_*/(a,b)_*).
\end{equation*}
Además, por una demostración similar a la de la \cref{eq:mcd2}, se tiene que
\begin{equation*}
    \left( (a,b)_*,\frac{a_*}{(a,b)_*} \right) = 1.
\end{equation*}
Finalmente, como $f$ es multiplicativa,
\begin{equation*}
    f(a) = f(a_*) = f((a,b)_*)f(a_*/(a,b)_*) = f((a,b)) \sum_{\substack{d \mid a \\ (d,b)=1}} (\mu * f)(d).
\end{equation*}
\end{proof}

\begin{example}
La función $\overline{\varphi}=\varphi(n)/n$ es separable. Nótese que para cualquier primo $p$ y $m>0$ se tiene $\varphi(p^m)=p^m-p^{m-1}$, luego $\varphi(p^m)/p^m=1-p^{-1}$ y también $\varphi(p)/p=1-p^{-1}$. Ahora, si $n=p_1^{\alpha_1} \cdots p_r^{\alpha_r}$ entonces, como $\varphi$ es multiplicativa,
\begin{equation*}
    \frac{\varphi(n)}{n} = \frac{\varphi(p_1^{\alpha_1})}{p_1^{\alpha_1}} \cdots \frac{\varphi(p_r^{\alpha_r})}{p_r^{\alpha^r}} = \left( 1-\frac{1}{p_1} \right) \cdots \left( 1-\frac{1}{p_r} \right) = \frac{\varphi(p_1)}{p_1} \cdots \frac{\varphi(p_r)}{p_r} = \frac{\varphi(n_*)}{n_*}
\end{equation*}
\end{example}

\begin{lemma}[Fórmula de Hölder]\label{lem:holder}
Para cada $n \in \mathbb{N}$ se tiene
\begin{equation*}
    c_r(n) = \frac{\varphi(r) \mu \left( \displaystyle \frac{r}{(n,r)} \right)}{\varphi \left( \displaystyle \frac{r}{(n,r)} \right)}
\end{equation*}
\end{lemma}

\begin{proof}
Se tiene
\begin{equation}\label{eq:par1}
    c_r (n) = \sum_{d \mid (n,r)} \mu \left( \frac{r}{d} \right) d = \sum_{\substack{d \mid (n,r) \\ \left( \frac{r}{(n,r)}, \frac{(n,r)}{d} \right) > 1}} \mu \left( \frac{r}{d} \right) d + \sum_{\substack{d \mid (n,r) \\ \left( \frac{r}{(n,r)}, \frac{(n,r)}{d} \right) = 1}} \mu \left( \frac{r}{d} \right) d,
\end{equation}
pero si $(r/(n,r),(n,r)/d)>1$ entonces $r/d$ debe tener un factor cuadrado, pues en este caso existe un primo $p$ tal que $p \mid r/(n,r)$ y $p \mid (n,r)/d$, luego $r=p (n,r) q_1$ y $(n,r)=p d q_2$ para algunos enteros $q_1$ y $q_2$, luego $r=p^2 d q_1 q_2$ y por tanto $p^2 \mid r/d$, así que $\mu(r/d)=0$. Luego la ecuación \eqref{eq:par1} es igual a
\begin{equation}\label{eq:par2}
\begin{split}
    \sum_{\substack{d \mid (n,r) \\ \left( \frac{r}{(n,r)}, \frac{(n,r)}{d} \right) = 1}} \mu \left( \frac{r}{d} \right) d & = \sum_{\substack{d \mid (n,r) \\ \left( \frac{r}{(n,r)}, \frac{(n,r)}{d} \right) = 1}} \mu \left( \frac{r}{(n,r)} \right) \mu \left( \frac{(n,r)}{d} \right) d \\
    & = \mu \left( \frac{r}{(n,r)} \right) \sum_{\substack{d \mid (n,r) \\ \left( \frac{r}{(n,r)}, \frac{(n,r)}{d} \right) = 1}} \mu \left( \frac{(n,r)}{d} \right) d \\
    & = \mu \left( \frac{r}{(n,r)} \right) \sum_{\substack{d \mid (n,r) \\ \left( \frac{r}{(n,r)},d \right)=1}} \mu(d) \frac{(n,r)}{d} \\
    & = (n,r) \mu \left( \frac{r}{(n,r)} \right) \sum_{\substack{d \mid (n,r) \\ \left( \frac{r}{(n,r),d} \right)=1}} \frac{\mu(d)}{d}
\end{split}
\end{equation}
pues $\mu$ es multiplicativa. Sea ahora $\Phi=\mu * \overline{\varphi}$, donde $\overline{\varphi}(s)=\varphi(s)/s$ para cada $s \in \mathbb{N}$. Se tiene entonces que
\begin{equation}\label{eq:par3}
\begin{split}
    \Phi(s) = \sum_{d \mid s} \mu(d) \overline{\varphi} \left( \frac{s}{d} \right) & = \sum_{d \mid s} \mu(d) \varphi \left( \frac{s}{d} \right) \frac{1}{s/d} = \sum_{d \mid s} \mu(d) \sum_{e \mid s/d} \mu(e) \frac{s/d}{e} \frac{1}{s/d} \\
            & = \sum_{d \mid s} \mu(d) \sum_{e \mid s/d} \frac{\mu(e)}{e}  = \sum_{e \mid s} \frac{\mu(e)}{e} \sum_{d \mid s/e} \mu(d) = \frac{\mu(s)}{s}
\end{split}
\end{equation}
pues si $d \mid s$ y $c \mid s/d$, entonces $d/s$ es un entero y $s/d=e q$ para algún entero $q$, luego $s= d e q$ y por tanto $e \mid s$ y $d \mid s/e$. El recíproco es similar. Además, todos los términos en la penúltima suma son cero excepto aquel para el cual $s/e = 1$, es decir, $s=e$. Luego la suma \eqref{eq:par2} es igual a
\begin{align*}
    (n,r) \mu \left( \frac{r}{(n,r)} \right) \sum_{\substack{d \mid (n,r) \\ \left( \frac{r}{(n,r)},d \right)=1}} \Phi(d) & = (n,r) \mu \left( \frac{r}{(n,r)} \right) \sum_{\substack{d \mid (n,r) \\ \left( \frac{r}{(n,r)},d \right)=1}} (\mu * \overline{\varphi}(d)) \\
    & = (n,r) \mu \left( \frac{r}{(n,r)} \right) \frac{\overline{\varphi}((n,r))}{\overline{\varphi} \left( (n,r), \frac{r}{(n,r)} \right)} \\
    & = (n,r) \mu \left( \frac{r}{(n,r)} \right) \frac{\overline{\varphi}(r)\overline{\varphi}((n,r))}{\overline{\varphi}((n,r))\overline{\varphi} \left( \frac{r}{(n,r)} \right)} \\
    & = (n,r) \mu \left( \frac{r}{(n,r)} \right) \frac{\overline{\varphi}(r)}{\overline{\varphi}\left( \frac{r}{(n,r)} \right)} \\
    & = (n,r)\mu \left( \frac{r}{(n,r)} \right) \frac{r \varphi(r)}{(n,r) r \varphi \left( \frac{r}{(n,r)} \right)} \\
    & = \frac{\mu \left( \frac{r}{(n,r)} \right) \varphi(r)}{\varphi \left( \frac{r}{(n,r)} \right)}
\end{align*}
donde la primera igualdad se cumple por definición de $\Phi$ y la ecuación \eqref{eq:par3}, la segunda por ser $\overline{\varphi}$ multiplicativa, separable y por el \Cref{lem:par0} (\textsc{\romannumeral 2}), la tercera por el \Cref{lem:par0} (\textsc{\romannumeral 1}) y la quinta por definición de $\overline{\varphi}$.
\end{proof}

\begin{theorem}\label{thm:fou1}
Toda función $f$ par $\Mod{r}$ tiene una expansión de la forma
\begin{equation}\label{eq:ram3}
    f(n) = \sum_{d \mid r} \alpha(d) c_d(n),
\end{equation}
y recíprocamente, toda función aritmética de esta forma es par $\Mod{r}$. Los coeficientes $\alpha(d)$ están dados por
\begin{equation}\label{eq:ram6}
    \alpha(d) = \frac{1}{r} \sum_{e \mid r} f \left( \frac{r}{e} \right) c_e \left( \frac{r}{d} \right),
\end{equation}
o por la fórmula equivalente,
\begin{equation*}
    \alpha(d) = \frac{1}{r \varphi(d)} \sum_{m=1}^{r} f(m) c_d(m).
\end{equation*}
A los coeficientes $\alpha$ se les llamará \textbf{coeficientes de Fourier} de la función par $f$.
\end{theorem}
\begin{proof}
Es claro que toda función de la forma \eqref{eq:ram3} es par $\Mod{r}$, pues por el lema anterior si $d \mid r$ entonces $c_d(n)=c_d((n,r))$. Nótese que
\begin{equation*}
\begin{split}
    \sum_{d \mid r} \alpha(d) c_d(n) &= \sum_{d \mid r} \left( \frac{1}{r} \sum_{e \mid r} f \left( \frac{r}{e} \right) c_e \left( \frac{r}{d} \right) \right) c_d(n) \\
                                     &= \frac{1}{r} \sum_{e \mid r} f \left( \frac{r}{e} \right) \sum_{d \mid r} c_e \left( \frac{r}{d} \right) c_d(n) \\
                                     &= \frac{1}{r} \sum_{e \mid r} f \left( \frac{r}{e} \right) \sum_{d \mid r} c_e \left( \frac{r}{d} \right) c_d((n,r)) \\
                                     &= \frac{1}{r} f \left( \frac{r}{q} \right) r = f((n,r)) = f(n),
\end{split}
\end{equation*}
por el \Cref{lem:holder}, donde $r=(n,r) q$, para algún $q \in \mathbb{N}$ y donde la última igualdad se cumple por ser $f$ par $\Mod{r}$.
\bigskip

Por otro lado, de la demostración de \Cref{thm:gauss} se puede rescatar el hecho de que el conjunto $\{ 1,2,\ldots,r \}$ es igual a $\bigcup_{e \mid r} \{ rx/e \std (x,e)=1, 1 \le x \le e \}$ y todos los conjuntos son disjuntos a pares, por tanto
\begin{equation*}
\begin{split}
    \frac{1}{r \varphi(d)}\sum_{m=1}^{r} f(m) c_d(m) &= \frac{1}{r \varphi(d)}\sum_{e \mid r} \sum_{\substack{(x,e)=1 \\ 1 \le x \le e}} f \left( \frac{rx}{e} \right) c_d \left( \frac{rx}{e} \right) \\
                               &= \frac{1}{r \varphi(d)}\sum_{e \mid r} \sum_{\substack{(x,e)=1 \\ 1 \le x \le e}} f \left( \left( \frac{rx}{e},r \right) \right) c_d \left( \left( \frac{rx}{e},r \right) \right) \\
                               &= \frac{1}{r \varphi(d)}\sum_{e \mid r} \sum_{\substack{(x,e)=1 \\ 1 \le x \le e}} f \left( \frac{r}{e} \right) c_d \left( \frac{r}{e} \right) \\
                               &= \frac{1}{r \varphi(d)}\sum_{e \mid r} f \left( \frac{r}{e} \right) c_d \left( \frac{r}{e} \right) \varphi(e) \\
                               &= \frac{1}{r \varphi(d)}\sum_{e \mid r} f \left( \frac{r}{e} \right) c_e \left( \frac{r}{d} \right) \varphi(d) \\
                               &= \frac{1}{r}\sum_{e \mid r} \left( \frac{r}{e} \right) c_e \left( \frac{r}{d} \right),
\end{split}
\end{equation*}
donde la segunda igualdad se cumple por ser $f$ par $\Mod{r}$, la tercera por ser $(rx/e,r)=r/e$, pues $(x,e)=1$ implica que $(r/e)(x,e)=r/e$, y como $r/e$ es un entero positivo, entonces $(rx/e,r)=r/e$. La cuarta por definición de $\varphi$, y la penúltima igualdad se cumple por la fórmula de Hölder (\Cref{lem:holder}) y el \Cref{cor:mcd1}, pues $e$ y $d$ dividen a $d$, así que
\begin{equation}\label{eq:holder}
    c_d \left( \frac{r}{e} \right) \varphi(e) = \frac{\varphi(d)\mu \left( \displaystyle \frac{d}{(r/e,d)} \right)}{\varphi \left( \displaystyle \frac{d}{(r/e,d)} \right)} = \frac{\varphi(d)\mu \left( \displaystyle \frac{e}{(r/d,e)} \right)}{\varphi \left( \displaystyle \frac{e}{(r/d,e)} \right)} \varphi(e) = \varphi(d) c_e \left( \frac{r}{d} \right)
\end{equation}
\end{proof}

\begin{corollary}
Si $f$ y $f'$ son funciones pares $\Mod{r}$, entonces se verifican las siguientes implicaciones
\begin{equation*}
    f(n) = \sum_{d \mid r} f'(d) c_d(n), \forall n \in \mathbb{N} \implies f'(\delta) = \frac{1}{r \varphi(\delta)} \sum_{m=1}^{r} f(m) c_d(m), \forall \delta \mid r,
\end{equation*}
\begin{equation*}
    f'(\delta) = \sum_{m=1}^{r} f(m) c_{\delta} (m), \forall d \mid r \implies f(n) = \frac{1}{r} \sum_{d \mid r} \frac{f'(d)}{\varphi(d)} c_d(n), \forall n \in \mathbb{N}.
\end{equation*}
\end{corollary}
\begin{proof}
En efecto, como $f$ es par, entonces $f$ tiene una única representación de la forma \eqref{eq:ram3}, por tanto se verifica la primera implicación. Para la segunda implicación se tiene que, de nuevo por la \cref{eq:ram3},
\begin{equation*}
    f(n) = \sum_{d \mid r} \left( \frac{1}{r \varphi(d)} \sum_{m=1}^{r} f(m) c_d(m) \right) c_d(n) = \frac{1}{r} \sum_{d \mid r} \frac{f'(d)}{\varphi(d)} c_d(n).
\end{equation*}
\end{proof}

Considerando los casos $\delta=1$ y $n=r$ se tiene el siguiente corolario.

\begin{corollary}
Si $f$ y $f'$ son funciones pares $\Mod{r}$, entonces
\begin{equation*}
    f(n) = \sum_{d \mid r} f'(d) c_d(n), \forall n \in \mathbb{N} \implies f'(1) = \frac{1}{r} \sum_{m=1}^{r} f(m),
\end{equation*}
\begin{equation*}
    f'(\delta) = \sum_{m=1}^{r} f(m) c_{\delta}(m), \forall d \mid r \implies f(r) = \frac{1}{r} \sum_{d \mid r} f'(d).
\end{equation*}
\end{corollary}

La \Cref{prop:ram0} sugiere considerar una clase más general de funciones, aquellas que se pueden escribir como
\begin{equation*}
    f(n) = \sum_{\delta \mid (n,r)} g(\delta), \forall n \in \mathbb{N},
\end{equation*}
donde $g$ es una función aritmética.
\bigskip

La siguiente proposición muestra que esta generalización preserva la modularidad respecto a $r$. Más aún, el \Cref{thm:fou1} permite caracterizar a las funciones pares módulo $r$ de esta forma.

\begin{theorem}
Toda función $f$ par módulo $r$ se puede expresar como
\begin{equation}\label{eq:fou2}
    f(n) = \sum_{\delta \mid (n,r)} g(\delta), \forall n \in \mathbb{N}.
\end{equation}
Y recíprocamente, toda función aritmética que tenga esta forma es par módulo $r$.
\end{theorem}
\begin{proof}
Dado que $(n,r)=((n,r),r)$, es claro que toda función arimética que tenga dicha forma es par $\Mod{r}$. Supóngase que $f$ es par $\Mod{r}$. Por el \Cref{thm:fou1}, se puede escribir
\begin{align*}
    f(n) & = \sum_{d \mid r} \alpha(d) c_d(n) = \sum_{d \mid r} \alpha(d) \sum_{\delta \mid (n,d)} \mu \left( \frac{d}{\delta} \right) \delta \\
         & = \sum_{\delta \mid (n,d)} \sum_{d \mid r} \alpha(d) \mu \left( \frac{d}{\delta} \right) \delta = \sum_{\delta \mid (n,r)} \delta \sum_{d \mid r} \mu \left( \frac{d}{\delta} \right) \delta = \sum_{\delta \mid (n,r)} g(\delta) 
\end{align*}
\end{proof}

En vista de que toda función par tiene al menos dos representaciones, la del teorema anterior y la del \Cref{thm:fou1}, cabe preguntarse si existe alguna relación entre ellas. El siguiente teorema da respuesta a esta inquietud.

\begin{theorem}
Si $f$ es par módulo $r$, entonces sus respectivas expansiones \eqref{eq:ram3} y \eqref{eq:fou2} están relacionadas mediante la fórmula
\begin{equation*}
    \alpha(d) = \frac{1}{r} \sum_{e \mid r/d} g \left( \frac{r}{e} \right) e.
\end{equation*}
\end{theorem}
\begin{proof}
Se tiene que
\begin{align*}
    \alpha(d) = \frac{1}{r} \sum_{\delta \mid r} f \left( \frac{r}{\delta} \right) c_{\delta} \left( \frac{r}{d} \right) & = \frac{1}{r} \sum_{\delta \mid r} c_{\delta} \left( \frac{r}{d} \right) \sum_{e \mid (r/\delta,\delta)} g(e) = \frac{1}{r} \sum_{\delta \mid r} c_{\delta} \left( \frac{r}{d} \right) \sum_{e \mid r/\delta} g(e) \\
              & = \frac{1}{r} \sum_{\delta \mid r} c_{\delta} \left( \frac{r}{d} \right) \sum_{e \mid r/\delta} g(e) = \frac{1}{r} \sum_{e \mid r} g \left( \frac{r}{e} \right) \sum_{\delta \mid e} c_{\delta} \left( \frac{r}{d} \right) \\
              & = \frac{1}{r} \sum_{e \mid r} g \left( \frac{r}{e} \right) \cdot \begin{cases}
                  \hfil e & \text{si } e \mid r/d \\
                  \hfil 0 & \text{en otro caso} \\
              \end{cases} = \frac{1}{r} \sum_{e \mid r/d} g \left( \frac{r}{e} \right) e.
\end{align*}
por el \cref{cor:ram5}.
\end{proof}

\subsection{El subespacio de funciones pares}

Recuérdese del \cref{cor:est1} que el anillo $(\mathcal{A}, +, *)$ es un álgebra conmutativa con identidad. Se tiene que el conjunto de funciones pares es un subespacio de $\mathcal{A}$, pero no es un subanillo, pues en general $(f*g)(n) \ne (f*g)((n,r))$ aún cuando $f$ y $g$ sean pares $\Mod{r}$, tome por ejemplo dos funciones constantes. Se denotará como $\mathcal{A}_r$ al conjunto de funciones pares módulo $r$.

\begin{proposition}
El conjunto $\mathcal{A}_r$ es una subespacio de $\mathcal{A}$.
\end{proposition}
\begin{proof}
Basta verificar las siguientes condiciones:
\begin{enumerate}[label=\textnormal{(\roman*)}]
\item $\mathcal{A}_r \ne \emptyset$
\item $f, g \in \mathcal{A}_r$ implica $f+g \in \mathcal{A}_r$
\item $c \in \mathbb{C}$ y $f \in \mathcal{A}_r$ implica $c f \in \mathcal{A}$.
\end{enumerate}
Claro que $\mathbf{0} \in \mathcal{A}_r$. Sea $n \in \mathbb{N}$ y supóngase que $f, g \in \mathcal{A}_r$. Se tiene que $(c f)(n) = c f(n) = c f((n,r)) = (c f)((n,r))$, luego $c f \in \mathcal{A}_r$.
\bigskip

Además, $(f+g)(n)=f(n)+g(n)=f((n,r))+g((n,r))=(f+g)((n,r))$, así que $f+g \in \mathcal{A}_r$.
\bigskip
\end{proof}

El \cref{cor:ind} afirma que las sumas de Ramanujan $\mathcal{B}_r = \{ c_d \}_{d \mid r}$ son linealmente independientes respecto a la suma sobre los divisores de $r$, además el \Cref{thm:fou1} nos permite expresar cualquier función par como una suma de este tipo. En otras palabras, el conjunto $\mathcal{B}_r$ es una base del espacio vectorial $\mathcal{A}_r$.

\begin{corollary}
El espacio vectorial $\mathcal{A}_r$ tiene dimensión $d(r)$.
\end{corollary}

En lo que sigue de esta sección se verá que los coeficientes de la expansión \eqref{eq:ram3} pueden ser derivados de un producto interno en este espacio de funciones.

\begin{proposition}
La operación $\langle \phantom{f},\phantom{g} \rangle : \mathcal{A}_r\times \mathcal{A}_r \longrightarrow \mathbb{C}$ definida como
\begin{equation*}
    \langle f,g \rangle = \sum_{d \mid r} \varphi(d) f \left( \frac{r}{d} \right) \overline{g \left( \frac{r}{d} \right)}
\end{equation*}
es un producto interno en $\mathcal{A}_r$.
\end{proposition}
\begin{proof}
Sean $c \in \mathbb{C}$ y $f,g,h \in \mathcal{A}_r$. Escríbase $f(n)=f_1(n)+ i f_2(n)$ para todo $n \in \mathbb{N}$, donde $f_1,f_2 \in \mathcal{A}_{\mathbb{R}}$. Entonces
\begin{enumerate}[label=\textnormal{(\roman*)}]
\item \begin{align*}
        \langle f,f \rangle & = \sum_{d \mid r} \varphi(d) \left[ f_1 \left( \frac{r}{d} \right) + i f_2 \left( \frac{r}{d} \right) \right] \overline{\left[ f_1 \left( \frac{r}{d} \right) + i f_2 \left( \frac{r}{d} \right) \right]} \\
                            & = \sum_{d \mid r} \varphi(d) \left[ f_1^2 \left( \frac{r}{d} \right) + f_2^2 \left( \frac{r}{d} \right) \right] \ge 0
\end{align*}
Además, si $\langle f,f \rangle=0$, como todos los términos de la suma anterior son positivos, se debe tener que $f_1(d)=f_2(d)=0$ para todo $d$ divisor de $r$. Si $n \in \mathbb{N}$ entonces $f_1(n)=f((n,r))=0$, pues $f$ es par $\Mod{r}$ y $(n,r) \mid r$. Análogamente se tiene $f_2(n)=0$. Luego $f(n)=f_1(n)+i f_2(n)=0$ para cada $n \in \mathbb{N}$, es decir $f=\mathbf{0}$.
\item \begin{align*}
        \langle f+g,h \rangle & = \sum_{d \mid r} \varphi(d) (f+g) \left( \frac{r}{d} \right) \overline{g \left( \frac{r}{d} \right)} = \sum_{d \mid r} \varphi(d) \left[ f \left( \frac{r}{d} \right) \overline{h \left( \frac{r}{d} \right)} + g \left( \frac{r}{d} \right) \overline{h \left( \frac{r}{d} \right)} \right] \\
                              & = \sum_{d \mid r} \varphi(d) f \left( \frac{r}{d} \right) \overline{h \left( \frac{r}{d} \right)} + \sum_{d \mid r} g \left( \frac{r}{d} \right) \overline{h \left( \frac{r}{d} \right)} = \langle f,h \rangle + \langle g,h \rangle
\end{align*}
\item \begin{equation*}
    \langle c f,g \rangle = \sum_{d \mid r} \varphi(d) c f \left( \frac{r}{d} \right) \overline{g \left( \frac{r}{d} \right)} = c \sum_{d \mid r} \varphi(d) f \left( \frac{r}{d} \right) \overline{g\left( \frac{r}{d} \right)} = c \langle f,g \rangle
\end{equation*}
\item \begin{equation*}
    \overline{\langle g,f \rangle} = \overline{\sum_{d \mid r} \varphi(d) g \left( \frac{r}{d} \right) \overline{f \left( \frac{r}{d} \right)}} = \sum_{d \mid r} \varphi(d) f \left( \frac{r}{d} \right) \overline{g \left( \frac{r}{d} \right)} = \langle f,g \rangle
\end{equation*}
\end{enumerate}
\end{proof}

El producto interno así definido podría haberse escrito sin el factor $\varphi(d)$ dentro de la suma. Su uso se justifica al usar la ecuación \eqref{eq:holder} para probar que este producto interno hace de $\mathcal{B}_r$ una base ortogonal de $\mathcal{A}_r$.

\begin{proposition}
Si $i,j$ son divisores positivos de $r$, entonces
\begin{equation*}
    \langle c_i,c_j \rangle = \begin{cases}
        \hfil r \varphi(j) & \text{si } i=j \\
        \hfil 0 & \text{en otro caso}.
    \end{cases}
\end{equation*}
\end{proposition}
\begin{proof}
En efecto,
\begin{equation*}
    \langle c_i,c_j \rangle = \sum_{d \mid r} \varphi(d) c_i \left( \frac{r}{d} \right) c_j \left( \frac{r}{d} \right) = \sum_{d \mid r} c_i \left( \frac{r}{d} \right) \varphi(j) c_d \left( \frac{r}{j} \right) = 
    \begin{cases}
        \hfil r \varphi(j) & \text{si } i=j \\
        \hfil 0 & \text{en otro caso},
    \end{cases}
\end{equation*}
donde la segunda igualdad se cumple por la \cref{eq:holder} y la tercera por el \Cref{lem:holder}.
\end{proof}

\begin{corollary}
El conjunto
\begin{equation*}
    \mathcal{B}'_r = \left\{ c'_d = \frac{1}{\sqrt{r \varphi(d)}} c_d \right\}_{d \mid r}
\end{equation*}
es una base ortonormal de $\mathcal{A}_r$.
\end{corollary}

Como consecuencia del corolario anterior, toda función $f$ par $\Mod{r}$ debe tener una expansión de la forma $\sum_{e \mid r} \beta(e) c'_e$. Además, si $d$ es un divisor arbitrario de $r$, entonces
\begin{equation*}
    \left\langle f, c'_d \right\rangle = \left\langle \sum_{e \mid r} \beta(e) c'_e,c'_d \right\rangle = \sum_{e \mid r} \beta(e) \left\langle c'_e,c'_d \right\rangle = \beta(d),
\end{equation*}
pero $f$ también tiene una única representación de la forma $\sum_{e \mid r} \alpha(d) c_d$, en consecuencia, $\alpha(d) = \beta(d)/\sqrt{r \varphi(d)}$ para cada $d$ divisor de $r$. Por tanto,
\begin{align*}
    \alpha(d) & = \frac{\beta(d)}{\sqrt{r \varphi(d)}} = \frac{1}{\sqrt{r \varphi(d)}} \left\langle f, c'_d \right\rangle \\
                                                     & = \frac{1}{\sqrt{r \varphi(d)}} \sum_{e \mid r} \varphi(e) f \left( \frac{r}{e} \right) c'_d \left( \frac{r}{e} \right) \\
                                                     & = \frac{1}{\sqrt{r \varphi(d)}} \sum_{e \mid r} \varphi(e) f \left( \frac{r}{e} \right) \frac{1}{\sqrt{r \varphi(d)}} c_d \left( \frac{r}{e} \right) \\
                                                     & = \frac{1}{r \varphi(d)} \sum_{e \mid r} f \left( \frac{r}{e} \right) \varphi(d) c_e \left( \frac{r}{d} \right) \\
                                                     & = \frac{1}{r} \sum_{e \mid r} f \left( \frac{r}{e} \right) c_e \left( \frac{r}{d} \right),
\end{align*}
donde la cuarta igualdad se cumple nuevamente por \eqref{eq:holder}. Esta es la \cref{eq:ram6}. Ahora se pueden llamar propiamente \emph{coeficientes de Fourier} a los coeficientes $\alpha$ del \Cref{thm:fou1} por ser derivados de un producto interno.
